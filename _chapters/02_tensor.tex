% ══════════════════════════════════════════════════════════════════════════════════════════════════
% !TeX root = ../Draft.tex
% !TeX spellcheck = fa_IR
% ══════════════════════════════════════════════════════════════════════════════════════════════════
\chapter{تانسورها}

\section{مقدمه‌ای بر تانسورها}

تانسورها ابزارهای ریاضی بنیادین و قدرتمندی هستند که در قرن نوزدهم توسط ریاضیدانانی نظیر گرگوریو ریچی-کوربسترو و تولیو لوی-چیویتا توسعه یافتند و سپس توسط انیشتین در نظریه‌ی نسبیت عام به شهرت جهانی رسیدند. این ابجکت‌های ریاضی نه تنها ابزاری برای توصیف کمیت‌های فیزیکی در مکانیک پیوسته، فیزیک نظری، و مهندسی هستند، بلکه زبان ریاضی یکپارچه‌ای را برای بیان قوانین طبیعت فراهم می‌کنند.

قدرت اصلی تانسورها در قابلیت نمایش روابط پیچیده‌ی چندبعدی بین متغیرهای مختلف نهفته است، به طوری که این روابط تحت تبدیلات مختصات شکل ناوردای خود را حفظ می‌کنند. این ویژگی بنیادین باعث می‌شود که قوانین فیزیکی مستقل از انتخاب سیستم مختصات خاص باقی بمانند، که اساس اصل نسبیت و یکی از ارکان اصلی فیزیک مدرن محسوب می‌شود.

در مکانیک پیوسته، تانسورها امکان توصیف دقیق و جامع پدیده‌هایی نظیر تنش، کرنش، نرخ کرنش، گرادیان سرعت، و خصوصیات مادی را فراهم می‌کنند. برخلاف کمیت‌های اسکالر که تنها یک مقدار عددی دارند، یا بردارها که جهت و بزرگی را توصیف می‌کنند، تانسورها قادر به نمایش روابط پیچیده‌ای هستند که در آنها چندین جهت و مؤلفه به طور همزمان نقش دارند.

\section{ابزارهای ریاضی}

\subsection{تانسورها}

\paragraph{ناورداری} مفهوم ناورداری یکی از بنیادین‌ترین اصول در ریاضیات و فیزیک محسوب می‌شود. یک سیستم مختصات در حقیقت نمایانگر یک ناظر یا مشاهده‌گر است - به شرطی که فاصله‌ی زمانی بین رویدادها غیرمرتبط باشد. این مفهوم ریشه در این واقعیت دارد که طبیعت مستقل از روش‌هایی است که ما برای توصیف آن انتخاب می‌کنیم.

با معرفی یک سیستم مختصات، امکان تخصیص مختصات عددی به موجودیت‌های فیزیکی و هندسی فراهم می‌شود. این مختصات در واقع نمایانگر موقعیت، جهت، یا خصوصیات یک شیء در آن سیستم مختصات خاص هستند. اما نکته‌ی حائز اهمیت این است که تشکیل سیستم‌های مختصات مختلف، ناظر را با نمایش‌های متفاوت از همان شیء فیزیکی روبرو می‌کند.

برای مثال، یک نیروی اعمالی به جسم ممکن است در سیستم مختصات کارتزین مؤلفه‌هایی داشته باشد که کاملاً متفاوت از مؤلفه‌های همان نیرو در سیستم مختصات قطبی باشد، اما نیروی فیزیکی واقعی تغییری نکرده است. از آنجا که همه‌ی این نمایش‌های مختلف به همان شیء فیزیکی اشاره دارند، آنها در جوهر خود معادل هستند - یا به تعبیر دقیق‌تر، شیء فیزیکی اصلی \textit{ناوردا} است و مستقل از روش توصیف ما باقی می‌ماند.

\paragraph{قوانین تبدیل} قوانین تبدیل، قلب ریاضی مفهوم تانسور را تشکیل می‌دهند و تعیین می‌کنند که چگونه مؤلفه‌های یک تانسور هنگام تغییر سیستم مختصات تبدیل می‌شوند. تانسورها برای توصیف کمیت‌های فیزیکی بنیادین نظیر تنش، کرنش، مومنتوم، میدان الکترومغناطیسی، و متریک فضا-زمان استفاده می‌شوند. نمایش ریاضی هر تانسور، ترکیبی هماهنگ از پایه‌های سیستم مختصات و مؤلفه‌های عددی تانسور است.

این دوگانگی اساسی باعث می‌شود که هنگام تغییر سیستم مختصات، هم پایه‌های مختصات و هم مؤلفه‌های عددی تانسور دستخوش تغییر شوند، اما به گونه‌ای که کمیت فیزیکی کلی بدون تغییر باقی بماند. مطالبه‌ی ناورداری از یک تانسور با اعمال تساوی ریاضی بین دو نمایش مختلف آن از همان کمیت فیزیکی انجام می‌شود.

نتیجه‌ی این فرآیند، استخراج \textit{قوانین تبدیل} است که به وسیله‌ی آنها ساختارهای ریاضی ناوردا تشکیل و حفظ می‌شوند. این قوانین بر اساس ماتریس‌های ژاکوبین تبدیل مختصات تعریف می‌شوند. به طور کلی‌تر، ناورداری کمیت‌های فیزیکی به دلیل رابطه‌ی ریاضی معکوس بین ژاکوبین‌ها در تبدیل مؤلفه‌های کوواریانت (که با شاخص‌های پایین نمایش داده می‌شوند) و کنتراواریانت (که با شاخص‌های بالا نمایش داده می‌شوند) تانسور حاصل می‌شود. این مکانیسم ریاضی تضمین می‌کند که ضرب داخلی تانسورها مستقل از انتخاب سیستم مختصات باقی بماند.

\paragraph{تعریف شهودی} اگرچه مفهوم ناورداری ممکن است به عنوان انگیزه‌ی بنیادی و اصلی برای توسعه و استفاده از تانسورها در نظر گرفته شود، خصوصیات ریاضی و هندسی دیگر تانسور نیز در تعریف جامع و کامل تانسورها نقش بسزایی دارند. در ادبیات علمی و کتب تخصصی، رویکردهای مختلفی برای تعریف تانسورها اتخاذ شده است.

برخی از ریاضیدانان و فیزیک‌دانان سعی کرده‌اند تعریف جامعی ارائه دهند با تمرکز بر توضیح دقیق رفتار تانسور تحت تبدیلات پایه و قوانین تبدیل مؤلفه‌ها، در حالی که گروه دیگری بیشتر بر ماهیت نگاشت خطی تانسورها و عملکرد تابعی آنها متمرکز شده‌اند. این دو رویکرد مکمل یکدیگر هستند و هر کدام جنبه‌های مهمی از طبیعت تانسورها را روشن می‌کنند.

از منظر شهودی و کاربردی، تانسور را می‌توان به عنوان یک شیء ریاضی چندوجهی تصور کرد که قابلیت ذخیره، پردازش، و انتقال اطلاعات پیچیده‌ی هندسی و فیزیکی را دارد. این شیء ریاضی اطلاعات زیر را به طور یکپارچه و هماهنگ در خود جای می‌دهد:

\begin{enumerate}
    \item دستورالعمل‌هایی برای یک نگاشت خطی،
    \item مؤلفه‌های مرتبط بر حسب بردارهای پایه‌ی خاص، و
    \item کیفیت ناوردا بودن تحت تبدیل پایه.
\end{enumerate}

مؤلفه‌های یک تانسور با انتخاب بردارهای پایه تعیین می‌شوند و آنها طبق قوانین تبدیل خاصی از یک مجموعه پایه به پایه‌ی دیگر تغییر می‌کنند. بنابراین، مؤلفه‌های یک تانسور با پایه‌هایش مرتبط هستند، یعنی آنها در یک مجموعه پایه‌ی خاص مؤلفه‌های منحصر به فرد دارند. این مؤلفه‌ها اگر بردارهای پایه تغییر کنند تغییر خواهند کرد. با این وجود، مؤلفه‌های یک تانسور همراه با پایه‌هایش ناوردا هستند.

\begin{definition}[تانسور]\label{def:tensors}
    تانسور $\ten{T}$ از مرتبه‌ی $n$-ام در فضای $d$-بعدی به عنوان یک تابع خطی اسکالر-مقدار از $n$ بردار تعریف می‌شود:
    \begin{equation}
        \begin{aligned}
            \ten{T}: & \underbrace{\tena{u}\in\Rset^d\times\tena{v}\in\Rset^d\times\ldots  \times\tena{w}\in\Rset^d}_n   \mapsto \alpha\in\Rset \\
                     & \ten{T}[\tena{u},\tena{v},\ldots,\tena{w}]=\alpha,
        \end{aligned}
    \end{equation}
    که در آن $d$ بعد فضا، $\alpha$ یک اسکالر، و $n$ مرتبه (یا رتبه) تانسور است که با $\order{\ten{T}}=n$ نمایش داده می‌شود. بعد و مرتبه‌ی یک تانسور با هم به صورت $\class{n}{d}$ بیان می‌شوند که \textit{کلاس} یک تانسور را نشان می‌دهد، یعنی مجموعه‌ی همه‌ی تانسورهای $n$-ام مرتبه و $d$-بعدی. توجه کنید که یک کلاس عمومی‌تر (مجموعه) تنها با نشان دادن مرتبه‌ی تانسور مانند $\class{n}{}$ ساخته می‌شود.

    توجه کنید که هدف‌گیری یک مؤلفه‌ی خاص از تانسور به روش‌های مختلف نمایش داده می‌شود:
    \begin{equation}
        T_{ij\ldots k}\equiv[\ten{T}]_{ij\ldots k}\equiv (\ten{T})_{ij\ldots k}.
    \end{equation}
\end{definition}

برای تأکید بر خطی بودن تانسورها، می‌توان خط زیر را برای تکمیل تعریف اضافه کرد:
\begin{equation}
    \forall a,b,\ldots ,z\in \Rset:\qquad\ten{T}[a\tena{u},b\tena{v},\ldots,z\tena{w}]=ab\ldots z\,\ten{T}[\tena{u},\tena{v},\ldots,\tena{w}].
\end{equation}

\section{عملیات تانسوری}

\subsection{ضرب داخلی}
در جبر خطی، ضرب داخلی یکی از بنیادین‌ترین و مهم‌ترین عملیات ریاضی محسوب می‌شود که به عنوان یک نگاشت دوخطی (یا چندخطی) حقیقی-مقدار از دو بردار (یا تانسور) تعریف می‌شود. ضرب نقطه‌ای معمولی که در جبر برداری مقدماتی آموخته می‌شود، انتخاب استاندارد و رایج‌ترین مثال از ضرب داخلی محسوب می‌شود، اما مفهوم ضرب داخلی فراتر از این کاربرد ساده است.

ضرب داخلی - به عنوان یک مفهوم بنیادی ریاضی و ابزاری برای تعریف نرم - نقش کلیدی در تنظیم و ساختار دادن به فضاهای برداری دارد. با تعریف ضرب داخلی در یک فضای برداری، امکان تعریف مفاهیم هندسی ثانویه و مشتق‌شده‌ای نظیر طول (یا بزرگی) بردارها، زاویه بین دو بردار، عمود بودن، و فاصله هندسی فراهم می‌شود. فضای برداری حاصل از این فرآیند، یک فضای اقلیدسی نامیده می‌شود که در حقیقت چیزی جز فضای حقیقی مجهز به ساختار غنی‌شده‌ای شامل مفاهیم ضرب داخلی، نرم، و متریک نیست.

این ساختار ریاضی پیشرفته امکان تعمیم بسیاری از مفاهیم هندسی آشنا از فضای سه‌بعدی معمولی به فضاهای با ابعاد دلخواه (حتی بی‌نهایت بعد) را فراهم می‌کند و پایه‌ی محکمی برای تحلیل تانسوری و کاربردهای آن در فیزیک و مهندسی ایجاد می‌کند.

\begin{definition}[ضرب داخلی]
    در تحلیل تانسوری، ضرب داخلی یک عملیات دودویی روی دو تانسور است که اغلب با نماد نقطه ($\scp$) نشان داده می‌شود. برای مثال، تانسور $\ten{T}$ به طور متوالی روی چندین بردار اعمال می‌شود تا آنها را به یک اسکالر نگاشت کند:
    \begin{equation}\label{eq:contr}
        \ten{T}[\tena{v},\ldots,\tena{w}] \equiv \ten{T} \odot (\tena{v} \dyad  \ldots \dyad \tena{w})=\mathcal{T}_{i\ldots j}v_{i} \ldots w_j.
    \end{equation}
    قدرت یک تانسور توسط ضرب داخلی محاسبه می‌شود:
    \begin{equation}
        \ten{T}^n=\underbrace{\ten{T}\scp\ldots\scp\ten{T}}_{n-1\ \text{ضرب داخلی}},\qquad\qquad\forall n\in\Iset^+.
    \end{equation}
\end{definition}

\subsection{ضرب خارجی}
ضرب خارجی برای ایجاد موجودیت‌های مرتبه‌ی بالاتر استفاده می‌شود. برای مثال، یک دیاد از ضرب خارجی دو بردار ساخته می‌شود.

\begin{definition}[ضرب خارجی]
    ضرب دو واریانت منجر به واریانت دیگری می‌شود. چنین ضربی \textit{ضرب خارجی} یا \textit{ضرب تانسوری} نامیده می‌شود:
    \begin{equation}
        \ten{A}=\ten{B}\dyad\ten{C}.
    \end{equation}
    در نتیجه‌ی ضرب تانسوری، مرتبه‌ی نتیجه افزایش می‌یابد، یعنی $\order{\ten{A}}=\order{\ten{B}}+\order{\ten{C}}$.

    قدرت تانسوری یک تانسور توسط ضرب خارجی محاسبه می‌شود:
    \begin{equation}
        \ten{T}^\tenpow{n}=\underbrace{\ten{T}\dyad\ldots\dyad\ten{T}}_{n-1\ \text{ضرب خارجی}},\qquad\qquad\forall n\in\Iset^+.
    \end{equation}
\end{definition}

\subsection{تبدیل}
عملیات تبدیل ($\op^\tran$) یا مزدوج (برای تانسورهای مختلط) عملیاتی است که ترتیب دیادیک یک تانسور را تغییر می‌دهد. در حالی که تبدیل تانسورهای مرتبه‌ی اول غیرمرتبط است، تنها یک نوع تبدیل برای تانسور مرتبه‌ی دوم $\tenb{T}$ قابل تعریف است:
\begin{equation}
    \forall \tena{u},\tena{v}\in \class{1}{n}:\qquad \tena{u}\scp\tenb{T}\scp\tena{v}=\tena{v}\scp\tenb{T}^\tran\scp\tena{u}.
\end{equation}

\section{کاربردهای تانسورها در مکانیک پیوسته}

تانسورها در مکانیک پیوسته نه تنها نقش اساسی و بنیادی دارند، بلکه در حقیقت زبان ریاضی اصلی و غیرقابل جایگزین این شاخه از علوم مهندسی محسوب می‌شوند. این کاربرد گسترده و عمیق ریشه در ماهیت چندبعدی و پیچیده‌ی پدیده‌های مکانیک پیوسته دارد که نیازمند ابزارهای ریاضی قدرتمند برای توصیف دقیق و تحلیل جامع هستند.

تانسورها در مکانیک پیوسته برای توصیف و مدل‌سازی انواع مختلفی از کمیت‌های فیزیکی و مهندسی به کار می‌روند:

\begin{itemize}
    \item \textbf{تنش و کرنش در مواد}: تانسورهای تنش و کرنش که حالت مکانیکی نقاط مختلف یک پیوسته را توصیف می‌کنند و شامل اطلاعات جامعی درباره نیروهای داخلی، تغییرات شکل، و پاسخ مکانیکی مواد هستند.

    \item \textbf{خصوصیات مکانیکی و ترمودینامیکی مواد}: تانسورهای مرتبه‌ی بالا که روابط پیچیده‌ی بین متغیرهای مختلف حالت (نظیر تنش، کرنش، دما، و تغییرات حجم) را در قوانین تشکیل‌دهنده مواد مدل می‌کنند.

    \item \textbf{میدان‌های حرکتی}: شامل میدان‌های سرعت، شتاب، گرادیان سرعت، و تانسور نرخ کرنش که توصیف‌کننده‌ی جنبش و تغییرات حرکتی در پیوسته‌ها هستند.

    \item \textbf{قوانین تشکیل‌دهنده و رفتاری مواد}: تانسورهای پیچیده‌ای که روابط بین علت و معلول در رفتار مواد (نظیر الاستیسیته، ویسکوالاستیسیته، پلاستیسیته، و آسیب) را مدل‌سازی می‌کنند.

    \item \textbf{انتقال حرارت و جرم}: تانسورهای هدایت حرارتی، نفوذپذیری، و ضرایب انتقال که خصوصیات انتقال انرژی و جرم در محیط‌های ناهمگن و ناهمسان را توصیف می‌کنند.
\end{itemize}

قابلیت بنیادی ناورداری تانسورها تحت تبدیلات مختصات، آنها را برای توصیف دقیق و قابل اعتماد پدیده‌های فیزیکی که ذاتاً باید مستقل از سیستم مختصات انتخابی باشند، نه تنها مناسب بلکه ضروری می‌سازد. این ویژگی تضمین می‌کند که قوانین فیزیکی و روابط مهندسی بیان‌شده با تانسورها، صرف‌نظر از چگونگی انتخاب سیستم مرجع یا روش اندازه‌گیری، همواره معتبر و قابل استناد باقی بمانند.

\subsection{ضرب برداری}
جایگزینی یک ضرب خارجی در یک دیادیک، مرتبه‌ی تانسور را یک واحد کاهش می‌دهد. با این حال، خاصیت مهم‌تر ضرب برداری در ایجاد یک ناوردای برداری است که ارتباط نزدیکی با مفهوم شبه-تانسور دارد.

\begin{definition}[ضرب برداری]
    ضرب برداری دو بردار ($\tena{u}$ و $\tena{v}$) را به بردار سومی نگاشت می‌کند:
    \begin{equation}
        \tena{\omega}=\tena{u}\times\tena{v},
    \end{equation}
    که در آن نتیجه عمود بر صفحه‌ی بردارهای اصلی است و جهت آن بر اساس چپ یا راست بودن سیستم مختصات تعیین می‌شود.
\end{definition}

یک تانسور متقارن-مایل $\skw{\tenb{A}}\in\class{2}{d}$ تنها $d$ تعداد مؤلفه‌ی مستقل دارد و بنابراین می‌تواند توسط شبه-بردار $\tena{\omega}\in\class{1}{d}$ نمایش داده شود - که \textit{ناوردای برداری} مرتبط با تانسور (یا به سادگی بردار مرتبط) نامیده می‌شود. ناوردای برداری از طریق رابطه زیر بدست می‌آید:
\begin{equation}
    \tena{\omega}=-\frac{1}{2}\levi\dscp\skw{\tenb{A}},\label{eq:skw}
\end{equation}

بنابراین، بردار محوری یک دیادیک متقارن-مایل مرتبه‌ی دوم، ضرب برداری دیادهای مربوطه است:
\begin{equation}
    -\frac{1}{2}\gibbs{(\tena{u}\dyad\tena{v}-\tena{v}\dyad\tena{u})}=\tena{v}\cross\tena{u},
\end{equation}
که می‌تواند برای جایگزینی یک ضرب برداری با دیادیک متقارن-مایل آن استفاده شود یا از طرف دیگر، تانسور اصلی می‌تواند با ناوردای برداری‌اش جایگزین شود:
\begin{alignat}{2}
     & \forall \tena{x}\in\class{1}{d}: & \qquad\skw{\tenb{A}}\scp\tena{x}=\tena{\omega}\times \tena{x},
\end{alignat}




\subsection{تقارن}
تقارن یک تانسور چیزی جز ناورداری آن تحت عملیات تبدیل نیست. رایج‌ترین شکل آن تحت تبدیل اصلی است:
% \begin{subequations}
\begin{alignat}{4}
    \tenb{T}=\tenb{T}^{\tran} & \qquad \text{یا}\quad & T_{ij}   & =T_{ji},    \label{eq:tran2} \\
    \tenc{T}=\tenc{T}^{\tran} & \qquad \text{یا}\quad & T_{ijk}  & =T_{kji},  \label{eq:tran3}  \\
    \tend{C}=\tend{C}^{\tran} & \qquad \text{یا}\quad & C_{ijkl} & =C_{klij},\label{eq:tran4}
\end{alignat}
% \end{subequations}

برای تانسورهای مرتبه‌ی چهارم، انواع مختلف تقارن قابل تعریف است:
\begin{subequations}
    \begin{alignat}{2}
        \tend{C}^\ltran\equimust\tend{C} & \qquad\text{زیرتقارن چپ},    \\
        \tend{C}^\rtran\equimust\tend{C} & \qquad\text{زیرتقارن راست},  \\
        \tend{C}^\mtran\equimust\tend{C} & \qquad\text{زیرتقارن میانی}, \\
        \tend{C}^\otran\equimust\tend{C} & \qquad\text{زیرتقارن خارجی}.
    \end{alignat}
\end{subequations}

\paragraph{نتیجه‌ی تقارن} تأثیر تقارن در یک تانسور کاهش مؤلفه‌های مستقل آن است. برای مثال، یک تانسور عمومی مرتبه‌ی چهارم دارای ۸۱ مؤلفه‌ی مستقل است که در صورت وجود هر دو زیرتقارن راست و چپ به ۳۶ کاهش می‌یابد. اضافه کردن زیرتقارن میانی منجر به تقارن کامل با ۲۱ مؤلفه‌ی مستقل می‌شود.

\section{قضیه‌ی تصویر}
مفیدترین قضایای مهندسی از نظر شهودی واضح اما از نظر ریاضی پیچیده‌تر هستند. \textit{قضیه‌ی تصویر} می‌گوید که یک بردار می‌تواند به یک بردار دیگر در امتداد جهت ترجیحی (تصویر) به علاوه‌ی آنچه باقی می‌ماند (رد) تجزیه شود. این موضوع در فضای برداری سه‌بعدی واضح است اما همچنین برای بسیاری از ابجکت‌های بردار-مانند دیگر مانند تانسورها، ماتریس‌ها و توابع قابل مشتق‌گیری قابل اعمال است.

در این زمینه، برخی کاربردهای قضیه‌ی تصویر عبارتند از:
\begin{itemize}
    \item یک تابع اسکالر می‌تواند به صورت مجموع قسمت‌های فرد و زوج نوشته شود، مثلاً در سری فوریه با استفاده از دو مجموعه تابع متعامد.
    \item یک ماتریس می‌تواند به مؤلفه‌های متقارن و متقارن-مایل تجزیه شود.
    \item یک تابع پیوسته قابل مشتق‌گیری می‌تواند به صورت سری تیلور بیان شود.
    \item اکثر قوانین تشکیل‌دهنده‌ی مواد می‌توانند بر حسب تصاویر در قالب مؤلفه‌های حجمی و انحرافی بیان شوند.
\end{itemize}

\begin{definition}[تانسور تصویر]
    تانسور متقارن $\tenb{P}$ یک تانسور تصویر و تانسور متقارن $\tenb{P}^*$ تانسور تصویر مکمل آن است اگر:
    \begin{subequations}
        \begin{alignat}{4}
            \forall n\in\Iset^+ & : & \tenb{P}^n             & =\tenb{P},   & \quad & \text{خاصیت توانی،} \\
            \forall n\in\Iset^+ & : & (\tenb{P}^*)^n         & =\tenb{P}^*, &       & \text{خاصیت توانی،} \\
                                & : & \tenb{P}+\tenb{P}^*    & =\unitb,     &       & \text{کاملیت،}      \\
                                & : & \tenb{P}\scp\tenb{P}^* & =\null,      &       & \text{متعامد بودن.}
        \end{alignat}
    \end{subequations}
    این تصاویر از کلاس $\class{2}{3}$ هستند اما مفهوم قابل تعمیم به کلاس‌های مرتبه‌ی بالاتر نیز است.
\end{definition}

\paragraph{تانسورهای واحد به عنوان تصویرگر} بردار واحد مرتبه‌ی اول (بردار جهت) با $\unita$ نمایش داده می‌شود و دارای طول واحد است ($\norm{\unita}=1$). تانسور واحد مرتبه‌ی دوم می‌تواند به عنوان یک تانسور تصویر پیش‌پا افتاده استفاده شود:
\begin{equation}\label{eq:unitb}
    \unitb\equidef\delta_{ij}\,\base_i\dyad \base_j,
\end{equation}
که در آن $\delta_{ij}$ دلتای کرونکر است.

\subsection{ضرب مزدوج}
این نوع ضرب تغییری از ضرب خارجی است و در طول جابه‌جایی شاخص برای متقارن‌سازی/ضد متقارن‌سازی تانسورها به کار می‌رود.

\begin{definition}[ضرب مزدوج]
    ضرب مزدوج دوگانه‌ی ضرب تانسوری است. در ادبیات دو شکل موجود است:
    \begin{enumerate}
        \item یک ضرب دیادیک به دنبال تبدیل میانی:
              \begin{equation}
                  (\tenb{A}\dyadu\tenb{B})\equidef(\tenb{A}\dyad\tenb{B})^\mtran.
              \end{equation}
              این ضرب همچنین «ضرب تانسوری تبدیلات» نامیده می‌شود، یعنی اعمال آن به ضرب خارجی دو بردار (یک تانسور مرتبه‌ی دوم) معادل تبدیل جداگانه‌ی هر یک از این بردارها به وسیله‌ی تانسورهای مرتبه‌ی دوم است:
              \begin{equation}
                  (\tenb{A}\dyadu\tenb{B})\dscp(\tena{u}\dyad\tena{v})= (\tenb{A}\scp\tena{u})
                  \dyad(\tenb{B}\scp\tena{v}),
              \end{equation}

        \item یک ضرب دیادیک به دنبال تبدیل راست و تبدیل میانی:
              \begin{equation}
                  (\tenb{A}\dyado\tenb{B})\equidef\big((\tenb{A}\dyad\tenb{B})^\rtran\big)^\mtran,
              \end{equation}
    \end{enumerate}
\end{definition}

\section{گروه‌های تقارن}
در ارتباط با مکانیک پیوسته، مفهوم تقارن در نظریه‌ی گروه در مسائل مستقیم یافتن گروه تقارن یک تانسور و همچنین مسائل معکوس یافتن تانسور بر اساس عناصر خاص تقارن استفاده می‌شود.

\begin{definition}[گروه تقارن کلاسیک]
    برای یک تانسور داده شده $\ten{T}$، گروه تانسورهای متعامد $\tenb{Q}$ می‌تواند با اعمال ضرب ریلی پیدا شود:
    \begin{alignat}{3}\label{eq:sym_group}
         & \tenb{Q} \Rayleigh \ten{T} \equimust\ten{T},
    \end{alignat}
    که \textit{گروه تقارن} تانسور $n$-ام مرتبه $\ten{T}$ نامیده می‌شود. توجه کنید که این تعریف تنها برای فضاهای برداری اقلیدسی (غیرجهت‌دار) معتبر است.
\end{definition}

\paragraph{تقارن در جهات اصلی} شایان ذکر است که هر نوع تقارن مادی در تانسور سختی/انعطاف با مقادیر متقارن و اغلب مؤلفه‌های صفر منعکس می‌شود. از آنجا که تانسور سختی/انعطاف می‌تواند در هر جهت دلخواهی تنظیم شود، تقارن عددی ممکن است در هر جهتی قابل مشاهده نباشد. یعنی تنها در امتداد جهت «اصلی» مادی، تأثیر جداسازی تقارن قابل احساس و فعال است - در غیر این صورت رفتار به طور کلی ناهمسان است.

\paragraph{مواد متقارن عرضی} برای حالت مواد متقارن عرضی (کریستال‌های شش‌گوشه)، ۵ پارامتر مادی به علاوه‌ی محور تقارن باید شناخته شود. برای آزمایش محور تقارن که با بردار جهت $\tena{d}$ نمایش داده می‌شود، گروه تقارن مربوطه به سادگی شامل همه‌ی تانسورهای تبدیل $\tenb{Q}$ با زاویه‌ی دلخواه است که از رابطه‌ی تبدیل ایجاد می‌شوند. توجه کنید که ناورداری از قبل توسط تعریف گروه تقارن اعمال شده، یعنی با مطالبه‌ی رابطه‌ی گروه تقارن.

\section{تجزیه‌ی طیفی و مقادیر ویژه}

\subsection{تجزیه‌ی مقادیر ویژه}
هر تانسور متقارن مرتبه‌ی دوم می‌تواند به صورت طیفی تجزیه شود. این تجزیه یکی از مهم‌ترین ابزارهای تحلیل تانسورهای در مکانیک پیوسته محسوب می‌شود.

\begin{definition}[تجزیه‌ی طیفی]
    برای یک تانسور متقارن $\sym{\tenb{A}}\in\class{2}{3}$، تجزیه‌ی طیفی به صورت زیر تعریف می‌شود:
    \begin{equation}
        \sym{\tenb{A}} = \sum_{i=1}^{3} \lambda_i \tena{n}_i \dyad \tena{n}_i = \lambda_1 \tena{n}_1 \dyad \tena{n}_1 + \lambda_2 \tena{n}_2 \dyad \tena{n}_2 + \lambda_3 \tena{n}_3 \dyad \tena{n}_3,
    \end{equation}
    که در آن $\lambda_i$ مقادیر ویژه و $\tena{n}_i$ بردارهای ویژه‌ی واحد متناظر هستند که رابطه‌ی $\tena{n}_i \scp \tena{n}_j = \delta_{ij}$ را برآورده می‌کنند.
\end{definition}

مقادیر ویژه از حل معادله‌ی مشخصه زیر بدست می‌آیند:
\begin{equation}
    \det(\sym{\tenb{A}} - \lambda \unitb) = 0,
\end{equation}
که منجر به یک معادله‌ی مکعبی می‌شود:
\begin{equation}
    \lambda^3 - I_1 \lambda^2 + I_2 \lambda - I_3 = 0,
\end{equation}
که در آن $I_1$، $I_2$ و $I_3$ ناوردهای اصلی تانسور هستند.

\subsection{ناوردهای تانسوری}
ناوردهای یک تانسور کمیت‌هایی هستند که تحت تبدیلات مختصات تغییر نمی‌کنند. برای تانسور متقارن مرتبه‌ی دوم، سه ناورد اصلی وجود دارد:

\begin{subequations}
    \begin{alignat}{2}
        I_1 & = \tr{\sym{\tenb{A}}} = \lambda_1 + \lambda_2 + \lambda_3,                                                                       \\
        I_2 & = \frac{1}{2}[(\tr{\sym{\tenb{A}}})^2 - \tr{(\sym{\tenb{A}})^2}] = \lambda_1\lambda_2 + \lambda_2\lambda_3 + \lambda_3\lambda_1, \\
        I_3 & = \det{\sym{\tenb{A}}} = \lambda_1 \lambda_2 \lambda_3.
    \end{alignat}
\end{subequations}

این ناوردها در تحلیل رفتار مواد و تعریف معیارهای شکست بسیار مهم هستند.

\section{تجزیه‌های خاص تانسورها}

\subsection{تجزیه‌ی حجمی-انحرافی}
هر تانسور مرتبه‌ی دوم می‌تواند به دو قسمت حجمی (کروی) و انحرافی تجزیه شود:
\begin{equation}
    \tenb{A} = \vol{\tenb{A}} + \dev{\tenb{A}},
\end{equation}
که در آن:
\begin{subequations}
    \begin{alignat}{2}
        \vol{\tenb{A}} & = \frac{1}{3}\tr{\tenb{A}} \unitb = \frac{1}{3}A_{kk} \unitb,             \\
        \dev{\tenb{A}} & = \tenb{A} - \vol{\tenb{A}} = \tenb{A} - \frac{1}{3}\tr{\tenb{A}} \unitb.
    \end{alignat}
\end{subequations}

این تجزیه در مکانیک خاک و سنگ و همچنین در پلاستیسیته کاربرد گسترده دارد.

\subsection{تجزیه‌ی متقارن-متقارن مایل}
هر تانسور مرتبه‌ی دوم می‌تواند منحصراً به دو قسمت متقارن و متقارن-مایل تجزیه شود:
\begin{equation}
    \tenb{A} = \sym{\tenb{A}} + \skw{\tenb{A}},
\end{equation}
که در آن:
\begin{subequations}
    \begin{alignat}{2}
        \sym{\tenb{A}} & = \frac{1}{2}(\tenb{A} + \tenb{A}^\tran), \\
        \skw{\tenb{A}} & = \frac{1}{2}(\tenb{A} - \tenb{A}^\tran).
    \end{alignat}
\end{subequations}

قسمت متقارن معمولاً مربوط به کرنش و قسمت متقارن-مایل مربوط به چرخش محلی است.

\section{کاربردهای خاص در مکانیک پیوسته}

\subsection{تانسور کرنش}
تانسور کرنش کوشی-گرین راست به صورت زیر تعریف می‌شود:
\begin{equation}
    \tenb{C} = \tenb{F}^\tran \scp \tenb{F},
\end{equation}
که در آن $\tenb{F}$ تانسور گرادیان تغییر شکل است. تانسور کرنش گرین-لاگرانژ نیز از طریق رابطه زیر تعریف می‌شود:
\begin{equation}
    \tenb{E} = \frac{1}{2}(\tenb{C} - \unitb).
\end{equation}

\subsection{تانسور تنش}
تانسور تنش کوشی $\boldsymbol{\sigma}$ رابطه‌ی بین نیروهای داخلی و سطوح داخلی یک پیوسته را نشان می‌دهد:
\begin{equation}
    \tena{t} = \boldsymbol{\sigma} \scp \tena{n},
\end{equation}
که در آن $\tena{t}$ بردار تنش روی سطح با بردار نرمال $\tena{n}$ است.

تانسور تنش می‌تواند به اجزای حجمی و انحرافی تجزیه شود:
\begin{subequations}
    \begin{alignat}{2}
        \boldsymbol{\sigma} & = p \unitb + \dev{\boldsymbol{\sigma}},                                                    \\
        p                   & = \frac{1}{3}\tr{\boldsymbol{\sigma}} = \frac{\sigma_{11} + \sigma_{22} + \sigma_{33}}{3},
    \end{alignat}
\end{subequations}
که در آن $p$ فشار هیدروستاتیک و $\dev{\boldsymbol{\sigma}}$ تانسور تنش انحرافی است.

\subsection{تانسور نرخ کرنش}
برای جریان‌های ویسکوز، تانسور نرخ کرنش به صورت زیر تعریف می‌شود:
\begin{equation}
    \tenb{D} = \frac{1}{2}(\nabla \tena{v} + (\nabla \tena{v})^\tran),
\end{equation}
که در آن $\tena{v}$ میدان سرعت است.

\section{معیارهای تسلیم و شکست}

\subsection{معیار فون میزس}
معیار تسلیم فون میزس بر اساس انرژی کرنش انحرافی تعریف می‌شود:
\begin{equation}
    \sigma_{\text{eq}} = \sqrt{\frac{3}{2} \dev{\boldsymbol{\sigma}} : \dev{\boldsymbol{\sigma}}} = \sqrt{3 J_2},
\end{equation}
که در آن $J_2$ دومین ناورد تانسور تنش انحرافی است:
\begin{equation}
    J_2 = \frac{1}{2} \dev{\boldsymbol{\sigma}} : \dev{\boldsymbol{\sigma}}.
\end{equation}

\subsection{معیار ترسکا}
معیار تسلیم ترسکا بر اساس حداکثر تنش برشی تعریف می‌شود:
\begin{equation}
    \tau_{\max} = \frac{1}{2}(\sigma_{\max} - \sigma_{\min}),
\end{equation}
که در آن $\sigma_{\max}$ و $\sigma_{\min}$ به ترتیب حداکثر و حداقل تنش‌های اصلی هستند.

\section{نتیجه‌گیری}
تانسورها ابزارهای قدرتمند ریاضی هستند که امکان توصیف دقیق پدیده‌های پیچیده‌ی فیزیکی را فراهم می‌کنند. خصوصیت ناورداری آنها تحت تبدیلات مختصات، آنها را برای استفاده در قوانین فیزیکی مناسب می‌سازد. درک عمیق مفاهیم تانسوری و عملیات مربوط به آنها برای هر مهندس مکانیک و فیزیکدانی که با مکانیک پیوسته سروکار دارد، ضروری است.

عملیات مختلف تانسوری نظیر ضرب داخلی، ضرب خارجی، تبدیل، و تجزیه‌های مختلف امکان تحلیل و حل مسائل پیچیده‌ی مهندسی را فراهم می‌کنند. همچنین، کاربرد تانسورها در تعریف تنش، کرنش، و معیارهای شکست نقش حیاتی در طراحی و تحلیل سازه‌ها و مواد دارد.

\section{تحلیل تانسوری}

\subsection{مشتق‌گیری از تانسورها}
مشتق‌گیری از تانسورها نسبت به متغیرهای مختلف یکی از ابزارهای اساسی در تحلیل تانسوری محسوب می‌شود. برای تانسور $\ten{T}(\tena{x})$ که تابعی از موقعیت $\tena{x}$ است، گرادیان به صورت زیر تعریف می‌شود:
\begin{equation}
    \nabla \ten{T} = \frac{\partial \ten{T}}{\partial x_i} \dyad \base_i.
\end{equation}

برای تانسور مرتبه‌ی دوم $\tenb{A}(\tena{x})$، گرادیان یک تانسور مرتبه‌ی سوم است:
\begin{equation}
    \nabla \tenb{A} = \frac{\partial A_{jk}}{\partial x_i} \base_i \dyad \base_j \dyad \base_k.
\end{equation}

\subsection{واگرایی و چرخش}
واگرایی یک تانسور مرتبه‌ی دوم به صورت زیر تعریف می‌شود:
\begin{equation}
    \text{div } \tenb{A} = \nabla \scp \tenb{A} = \frac{\partial A_{ij}}{\partial x_i} \base_j.
\end{equation}

برای میدان برداری $\tena{v}$، واگرایی و چرخش به ترتیب عبارتند از:
\begin{subequations}
    \begin{alignat}{2}
        \text{div } \tena{v}  & = \nabla \scp \tena{v} = \frac{\partial v_i}{\partial x_i},                          \\
        \text{curl } \tena{v} & = \nabla \times \tena{v} = \epsilon_{ijk} \frac{\partial v_k}{\partial x_j} \base_i.
    \end{alignat}
\end{subequations}

\subsection{قضایای انتگرال}
قضیه‌ی گاوس برای تانسورها به صورت زیر بیان می‌شود:
\begin{equation}
    \int_V \nabla \scp \tenb{A} \, dV = \int_{\partial V} \tenb{A} \scp \tena{n} \, dS,
\end{equation}
که در آن $V$ حجم، $\partial V$ سطح محدوده‌کننده، و $\tena{n}$ بردار نرمال خارجی است.

\section{تبدیلات مختصات}

\subsection{قوانین تبدیل}
تحت تبدیل مختصات $x'_i = Q_{ij} x_j$، مؤلفه‌های تانسورهای مختلف به صورت زیر تبدیل می‌شوند:

برای بردار (تانسور مرتبه‌ی اول):
\begin{equation}
    v'_i = Q_{ij} v_j.
\end{equation}

برای تانسور مرتبه‌ی دوم:
\begin{equation}
    T'_{ij} = Q_{ik} Q_{jl} T_{kl}.
\end{equation}

برای تانسور مرتبه‌ی چهارم:
\begin{equation}
    C'_{ijkl} = Q_{im} Q_{jn} Q_{ko} Q_{lp} C_{mnop}.
\end{equation}

\subsection{ماتریس تبدیل}
ماتریس تبدیل $\tenb{Q}$ باید خصوصیات زیر را داشته باشد:
\begin{subequations}
    \begin{alignat}{2}
        \tenb{Q} \scp \tenb{Q}^\tran & = \unitb, \quad \text{(متعامد بودن)} \\
        \det(\tenb{Q})               & = +1. \quad \text{(جهت‌دار بودن)}
    \end{alignat}
\end{subequations}

این شرایط تضمین می‌کنند که تبدیل یک چرخش خالص باشد و طول‌ها و زوایا حفظ شوند.

\section{کاربردهای پیشرفته}

\subsection{مکانیک شکست}
در علم مکانیک شکست، که یکی از شاخه‌های پیچیده و کاربردی مکانیک جامدات محسوب می‌شود، تانسورهای تنش و کرنش نقش محوری و تعیین‌کننده‌ای در درک و پیش‌بینی رفتار مواد در حضور عیوب و ترک ایفا می‌کنند. مکانیک شکست به طور خاص با تحلیل تمرکز تنش در اطراف نقاط بحرانی نظیر نوک ترک سروکار دارد، جایی که تانسور تنش رفتار بسیار پیچیده و مشخصه‌ای از خود نشان می‌دهد.

برای ترک‌های حالت I (حالت بازشدگی یا کششی)، که رایج‌ترین نوع بارگذاری در مهندسی عملی محسوب می‌شود، میدان تنش در نزدیکی نوک ترک دارای یک ساختار ریاضی مشخص و قابل پیش‌بینی است. این میدان تنش با استفاده از نظریه‌ی الاستیسیته خطی و تکنیک‌های تحلیل مختلط به صورت زیر بیان می‌شود:

\begin{equation}
    \boldsymbol{\sigma}_{ij} = \frac{K_I}{\sqrt{2\pi r}} f_{ij}(\theta) + \text{مرتبه‌های بالاتر},
\end{equation}

که در آن $K_I$ ضریب شدت تنش برای حالت I است که کمیتی مستقل از موقعیت و فاصله از نوک ترک بوده و تنها به هندسه قطعه، ابعاد ترک، و بارگذاری اعمالی بستگی دارد، $r$ فاصله‌ی شعاعی از نوک ترک، و $f_{ij}(\theta)$ توابع بدون بعد زاویه‌ای هستند که توزیع زاویه‌ای تنش را در اطراف نوک ترک تعیین می‌کنند. عبارت «مرتبه‌های بالاتر» نیز اثرات هندسه‌ی محلی و شرایط مرزی را در نظر می‌گیرد که برای فواصل خیلی نزدیک یا خیلی دور از نوک ترک اهمیت پیدا می‌کنند.

\subsection{پلاستیسیته}
در نظریه‌ی پلاستیسیته، تانسور نرخ کرنش پلاستیک به تانسور تنش انحرافی مربوط است:
\begin{equation}
    \tenb{D}^p = \dot{\lambda} \frac{\partial f}{\partial \boldsymbol{\sigma}},
\end{equation}
که در آن $f$ تابع تسلیم، $\dot{\lambda}$ ضریب پلاستیک، و $\tenb{D}^p$ تانسور نرخ کرنش پلاستیک است.

\subsection{الاستیسیته}
رابطه‌ی تشکیل‌دهنده‌ی الاستیک خطی به صورت زیر بیان می‌شود:
\begin{equation}
    \boldsymbol{\sigma} = \tend{C} : \boldsymbol{\varepsilon},
\end{equation}
که در آن $\tend{C}$ تانسور سختی الاستیک مرتبه‌ی چهارم است. برای مواد همسان، این تانسور تنها دو پارامتر مستقل دارد:
\begin{equation}
    C_{ijkl} = \lambda \delta_{ij} \delta_{kl} + \mu (\delta_{ik} \delta_{jl} + \delta_{il} \delta_{jk}),
\end{equation}
که در آن $\lambda$ و $\mu$ ثابت‌های لامه هستند.

\section{روش‌های عددی}

\subsection{تانسورها در روش اجزای محدود}
در روش اجزای محدود، تانسورهای تنش و کرنش در نقاط انتگرال‌گیری محاسبه می‌شوند. ماتریس سختی المان از تانسور سختی مادی بدست می‌آید:
\begin{equation}
    \tenb{K}_e = \int_{V_e} \tenb{B}^\tran \tenb{D} \tenb{B} \, dV,
\end{equation}
که در آن $\tenb{B}$ ماتریس کرنش-جابجایی و $\tenb{D}$ ماتریس سختی مادی است.

\subsection{تکنیک‌های انتگرال‌گیری}
برای انتگرال‌گیری دقیق تانسورهای مرتبه‌ی بالا، از روش‌های انتگرال‌گیری گاوسی استفاده می‌شود. برای تانسور مرتبه‌ی چهارم، تعداد نقاط انتگرال‌گیری مورد نیاز معمولاً زیاد است.

\section{کاربردهای محاسباتی}

\subsection{نمایش تانسورها در برنامه‌نویسی}
در پیاده‌سازی عددی، تانسورهای مرتبه‌ی بالا معمولاً به صورت آرایه‌های چندبعدی یا ماتریس‌هایی با شاخص‌گذاری خاص نمایش داده می‌شوند. برای تانسور مرتبه‌ی چهارم $C_{ijkl}$، نمایش ماتریسی با استفاده از نماد Voigt رایج است.

\subsection{بهینه‌سازی محاسبات}
محاسبات تانسوری می‌توانند بسیار پرهزینه باشند. استفاده از تقارن‌های تانسور و روش‌های بهینه‌سازی عددی می‌تواند زمان محاسبه را به طور قابل توجهی کاهش دهد.

\section{مثال‌های کاربردی}

\subsection{مثال ۱: تحلیل تنش دوبعدی}
برای حالت تنش مسطح، تانسور تنش به صورت زیر است:
\begin{equation}
    \boldsymbol{\sigma} = \begin{bmatrix}
        \sigma_{xx} & \sigma_{xy} & 0 \\
        \sigma_{xy} & \sigma_{yy} & 0 \\
        0           & 0           & 0
    \end{bmatrix}.
\end{equation}

تنش‌های اصلی از حل معادله‌ی مشخصه بدست می‌آیند:
\begin{equation}
    \sigma_{1,2} = \frac{\sigma_{xx} + \sigma_{yy}}{2} \pm \sqrt{\left(\frac{\sigma_{xx} - \sigma_{yy}}{2}\right)^2 + \sigma_{xy}^2}.
\end{equation}

\subsection{مثال ۲: کرنش در تغییرشکل بزرگ}
برای تغییرشکل‌های بزرگ، تانسور کرنش گرین-لاگرانژ استفاده می‌شود:
\begin{equation}
    E_{ij} = \frac{1}{2}(C_{ij} - \delta_{ij}) = \frac{1}{2}(F_{ki} F_{kj} - \delta_{ij}),
\end{equation}
که در آن $F_{ij}$ مؤلفه‌های تانسور گرادیان تغییرشکل هستند.


\section{تانسور آکوستیک و ناپایداری مادی}

\subsection{تانسور آکوستیک}
تانسور آکوستیک (که همچنین با نام‌های تانسور موضعی‌سازی، تانسور قطبش، یا تانسور سختی مشخصه شناخته می‌شود) یکی از مفاهیم پیشرفته و تخصصی در مکانیک پیوسته محسوب می‌شود که نقش حیاتی در تشخیص و پیش‌بینی ناپایداری‌های مادی و پدیده‌های موضعی‌سازی کرنش ایفا می‌کند. این تانسور ابزاری قدرتمند برای درک رفتار مواد در شرایط بحرانی و مرز پایداری است.

\begin{definition}[تانسور آکوستیک]
    دو تعریف اصلی برای تانسور آکوستیک وجود دارد~\autocite{Etse.1999,Ottosen.2005}:
    \begin{subequations}
        \begin{alignat}{4}
            \tenb{A} & \equidef \dir{n}\scp \tend{C}\scp\dir{n}
                     & \qquad\text{یا}\qquad                           &
            A_{jk}   & \equidef C_{ijkl}\hat{n}_i \hat{n}_l
            ,                                                            \\
            \tenb{A} & \equidef \dir{n}\scp \tend{C}^\rtran\scp\dir{n}
                     & \qquad\text{یا}\qquad                           &
            A_{jk}   & \equidef C_{ijkl}\hat{n}_i \hat{n}_k
            ,
        \end{alignat}
    \end{subequations}
    که در آن $\dir{n}$ بردار جهت یا بردار نرمال به سطح موضعی‌سازی، و $\tend{C}$ تانسور سختی مادی مرتبه‌ی چهارم است.
\end{definition}

تانسور آکوستیک در حقیقت نمایانگر سختی جهتی مادی در راستای مشخصی است و اطلاعات مهمی درباره رفتار مادی تحت بارگذاری‌های پیچیده فراهم می‌کند. این تانسور به طور خاص در تحلیل پدیده‌هایی نظیر شکل‌گیری نوارهای برشی، موضعی‌سازی کرنش، و گذار از رفتار یکنواخت به رفتار ناهمگن در مواد کاربرد دارد.

\subsection{معیارهای ناپایداری مادی}
ناپایداری مادی از طریق تکین بودن (singularity) تانسور آکوستیک تشخیص داده می‌شود، که این امر به معنای صفر شدن دترمینان یا مقادیر ویژه‌ی تانسور آکوستیک است. این پدیده نشان‌دهنده‌ی از دست رفتن یکتایی جواب در مسائل مقدار مرزی و امکان شکل‌گیری حالت‌های تغییرشکل موضعی است.

\paragraph{معیارهای تشخیص موضعی‌سازی} دو معیار اصلی برای تشخیص موضعی‌سازی و ناپایداری مادی عبارتند از~\autocite{Staber.2021}:

\begin{enumerate}
    \item \textbf{از دست رفتن بیضوی بودن} (معادل عدم تکین بودن و معروف به معیار رایس برای موضعی‌سازی): وجود هر گونه مقدار ویژه‌ی صفر برای تانسور آکوستیک. این شرط به صورت ریاضی به شکل زیر بیان می‌شود:
          \begin{equation}
              \det(\tenb{A}) = 0 \quad \text{یا} \quad \exists \lambda_i = 0
          \end{equation}

    \item \textbf{از دست رفتن بیضوی بودن قوی}: از دست رفتن مثبت معین بودن تانسور متقارن برای همه‌ی جهات. این شرط سخت‌گیرانه‌تر و محافظه‌کارانه‌تر است:
          \begin{equation}
              \tena{u} \scp \tenb{A} \scp \tena{u} \leq 0 \quad \text{برای برخی} \quad \tena{u} \neq \nulla
          \end{equation}
\end{enumerate}

این معیارها نقش بنیادی در پیش‌بینی شکست، شکل‌گیری ترک، و انتقال از رفتار پایدار به ناپایدار در مواد دارند.

\subsection{مثال: تانسور آکوستیک در الاستیسیته}
برای مواد الاستیک همسان، تانسور آکوستیک شکل ساده و قابل تحلیلی دارد که امکان بررسی دقیق معیارهای پایداری را فراهم می‌کند.

\begin{example}[تانسور آکوستیک در الاستیسیته همسان]
    تانسور آکوستیک در الاستیسیته همسان به صورت زیر بیان می‌شود~\autocite{Bigoni.2012}:
    \begin{equation}
        \tenb{A}(\dir{n}) =  (\lambda+\mu)  \dir{n}\dyad \dir{n} + \mu\unitb,
    \end{equation}
    که در آن $\lambda$ و $\mu$ ثابت‌های لامه هستند.

    معیارهای ناپایداری در این حالت عبارتند از:
    \begin{enumerate}
        \item \textbf{بیضوی بودن} (عدم تکین بودن): $\mu\neq 0 \wedge \lambda+2\mu\neq0$
        \item \textbf{بیضوی بودن قوی} (مثبت معین بودن): $\mu> 0 \wedge \lambda+2\mu>0$
    \end{enumerate}

    این شرایط نشان می‌دهند که پایداری مادی به هر دو ثابت الاستیک بستگی دارد و نقض هر یک از این شرایط می‌تواند منجر به ناپایداری و موضعی‌سازی شود.
\end{example}

\subsection{کاربردهای عملی تانسور آکوستیک}
تانسور آکوستیک در موارد متعددی در مهندسی عملی کاربرد دارد:

\begin{itemize}
    \item \textbf{تحلیل پایداری سازه‌ها}: تشخیص نقاط بحرانی در سازه‌ها که ممکن است دچار کمانش موضعی شوند
    \item \textbf{مکانیک خاک}: پیش‌بینی شکل‌گیری نوارهای برشی در خاک‌های چسبنده
    \item \textbf{مکانیک سنگ}: تحلیل ناپایداری‌های مادی در توده‌سنگ‌ها تحت تنش‌های برجا
    \item \textbf{شکل‌دهی فلزات}: تشخیص شروع موضعی‌سازی کرنش در فرآیندهای شکل‌دهی
    \item \textbf{مکانیک شکست}: درک مکانیسم‌های رشد ترک و انتشار آسیب
\end{itemize}

درک عمیق تانسور آکوستیک و کاربرد صحیح آن در تحلیل‌های مهندسی، امکان طراحی ایمن‌تر و پیش‌بینی دقیق‌تر رفتار مواد تحت شرایط بحرانی را فراهم می‌کند.

\subsection{تحلیل ریاضی پیشرفته تانسور آکوستیک}

از نظر ریاضی، تانسور آکوستیک رابطه‌ای مستقیم با خواص طیفی تانسور سختی مادی دارد. برای درک عمیق‌تر این مفهوم، باید به بررسی جنبه‌های مختلف آن پرداخت.

\paragraph{خواص طیفی تانسور آکوستیک}
مقادیر ویژه تانسور آکوستیک $\tenb{A}(\dir{n})$ معیار مستقیمی از پایداری مادی در جهت $\dir{n}$ محسوب می‌شوند. اگر $\lambda_1, \lambda_2, \lambda_3$ مقادیر ویژه این تانسور باشند، آنگاه:

\begin{itemize}
    \item اگر همه‌ی مقادیر ویژه مثبت باشند ($\lambda_i > 0$)، مادی در آن جهت پایدار است
    \item اگر حداقل یک مقدار ویژه صفر شود ($\lambda_i = 0$)، مادی در مرز ناپایداری قرار دارد
    \item اگر حداقل یک مقدار ویژه منفی شود ($\lambda_i < 0$)، مادی ناپایدار است
\end{itemize}

\paragraph{تعمیم به مواد ناهمسان}
برای مواد ناهمسان، تانسور آکوستیک پیچیده‌تر می‌شود و بستگی به جهت بردار نرمال دارد. در این حالت، تحلیل پایداری نیازمند بررسی همه‌ی جهات ممکن است:

\begin{equation}
    \min_{\|\dir{n}\|=1} \min_i \lambda_i[\tenb{A}(\dir{n})] > 0
\end{equation}

این شرط تضمین می‌کند که مادی در همه‌ی جهات پایدار باقی بماند.

\subsection{کاربردهای پیشرفته در تحلیل ناپایداری}

\paragraph{تحلیل حساسیت ناپایداری}
یکی از کاربردهای مهم تانسور آکوستیک، تحلیل حساسیت سیستم نسبت به تغییرات پارامترهای مادی است. با محاسبه‌ی مشتقات مقادیر ویژه نسبت به پارامترهای مادی، می‌توان نقاط حساس سیستم را شناسایی کرد:

\begin{equation}
    \frac{\partial \lambda_i}{\partial p} = \tena{v}_i \scp \frac{\partial \tenb{A}}{\partial p} \scp \tena{v}_i
\end{equation}

که در آن $p$ پارامتر مادی و $\tena{v}_i$ بردار ویژه متناظر با مقدار ویژه $\lambda_i$ است.

\paragraph{پیش‌بینی مسیر موضعی‌سازی}
جهت بردار ویژه متناظر با کمترین مقدار ویژه تانسور آکوستیک، جهت احتمالی شکل‌گیری نوار برشی یا موضعی‌سازی کرنش را نشان می‌دهد. این اطلاعات برای طراحی و پیش‌بینی الگوهای شکست در مواد و سازه‌ها حیاتی است.

\section{جنبه‌های محاسباتی و عددی}

\subsection{محاسبه‌ی عددی تانسورها}
در کاربردهای عملی مهندسی، محاسبه‌ی عددی تانسورها و عملیات روی آنها اهمیت بالایی دارد. الگوریتم‌های محاسباتی موثر برای کار با تانسورها شامل:

\paragraph{ذخیره‌سازی و نمایش}
تانسورهای مرتبه‌ی بالا نیازمند ساختارهای داده‌ای خاصی هستند. برای تانسور مرتبه‌ی چهارم در فضای سه‌بعدی، $3^4 = 81$ مؤلفه وجود دارد که با استفاده از تقارن‌ها می‌توان آن را به تعداد کمتری کاهش داد.

\paragraph{بهینه‌سازی محاسبات}
استفاده از ویژگی‌های تقارن تانسورها برای کاهش پیچیدگی محاسباتی:
\begin{itemize}
    \item تقارن کوچک: $C_{ijkl} = C_{jikl} = C_{ijlk}$
    \item تقارن بزرگ: $C_{ijkl} = C_{klij}$
    \item کاهش ابعاد: از 81 مؤلفه به 21 مؤلفه مستقل
\end{itemize}

\subsection{الگوریتم‌های عددی پیشرفته}
\paragraph{محاسبه‌ی مقادیر ویژه}
برای تحلیل پایداری، محاسبه‌ی مقادیر ویژه تانسور آکوستیک ضروری است. روش‌های عددی شامل:
\begin{itemize}
    \item الگوریتم QR برای تانسورهای متقارن
    \item روش‌های تکراری برای تانسورهای بزرگ
    \item تکنیک‌های موازی‌سازی برای بهبود کارایی
\end{itemize}

\paragraph{تحلیل حساسیت عددی}
ارزیابی تأثیر خطاهای عددی بر نتایج تحلیل پایداری:
\begin{equation}
    \text{شرط عددی} = \frac{\lambda_{\max}}{\lambda_{\min}}
\end{equation}
که نشان‌دهنده‌ی حساسیت سیستم به اختلالات عددی است.




\section{جمع‌بندی و چشم‌انداز}

این فصل طیف وسیعی از مفاهیم تانسوری را پوشش داده است، از مبانی ریاضی اولیه تا کاربردهای پیشرفته در تحلیل ناپایداری مادی. تانسورها نه تنها پایه‌ی ریاضی مکانیک پیوسته و بسیاری از شاخه‌های فیزیک و مهندسی را تشکیل می‌دهند، بلکه ابزارهای قدرتمندی برای درک و پیش‌بینی رفتار مواد در شرایط پیچیده و بحرانی نیز محسوب می‌شوند.

\subsection{نکات کلیدی}
مهم‌ترین مفاهیم ارائه شده در این فصل عبارتند از:

\begin{itemize}
    \item \textbf{مبانی ریاضی}: درک عمیق تعاریف، عملیات، و خواص تانسورها
    \item \textbf{کاربردهای فیزیکی}: نقش تانسورها در توصیف تنش، کرنش، و خواص مادی
    \item \textbf{تحلیل ناپایداری}: استفاده از تانسور آکوستیک برای پیش‌بینی شکست و موضعی‌سازی
    \item \textbf{جنبه‌های محاسباتی}: الگوریتم‌ها و روش‌های عددی برای محاسبه‌ی موثر تانسورها
\end{itemize}

\subsection{توسعه‌های آینده}
پیشرفت‌های آینده در محاسبات تانسوری و کاربردهای آن شامل موارد زیر خواهد بود:

\paragraph{هوش مصنوعی و یادگیری ماشین}
\begin{itemize}
    \item استفاده از شبکه‌های عصبی برای بهینه‌سازی محاسبات تانسوری
    \item یادگیری الگوهای شکست از داده‌های تجربی
    \item پیش‌بینی رفتار مواد با استفاده از مدل‌های یادگیری عمیق
\end{itemize}

\paragraph{محاسبات موازی و ابرمحاسبه}
\begin{itemize}
    \item توسعه‌ی الگوریتم‌های موازی برای تانسورهای مرتبه‌ی بالا
    \item استفاده از واحدهای پردازش گرافیکی (GPU) برای تسریع محاسبات
    \item پیاده‌سازی روش‌های توزیع شده برای مسائل بزرگ‌مقیاس
\end{itemize}

\paragraph{کاربردهای نوظهور}
\begin{itemize}
    \item متامواد و ساختارهای دوره‌ای با خواص غیرمعمول
    \item نانومواد و مقیاس‌های کوچک با اثرات اندازه
    \item مواد هوشمند و سازه‌های تطبیقی
    \item مواد چندفازی و کامپوزیت‌های پیشرفته
\end{itemize}

درک عمیق مفاهیم تانسوری و توانایی کار با آنها برای هر متخصصی که در زمینه‌های مکانیک جامدات، مکانیک سیالات، یا علم مواد فعالیت می‌کند، ضروری است. تسلط بر این مفاهیم نه تنها درک بهتری از پدیده‌های فیزیکی ارائه می‌دهد، بلکه راه را برای توسعه‌ی روش‌های جدید تحلیل و طراحی در مهندسی هموار می‌کند و امکان پیشرفت در مرزهای دانش فنی را فراهم می‌آورد.