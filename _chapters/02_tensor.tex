% ══════════════════════════════════════════════════════════════════════════════════════════════════
% !TeX root = ../Draft.tex
% !TeX spellcheck = fa_IR
% ══════════════════════════════════════════════════════════════════════════════════════════════════

\chapter{تانسورها}

\section{مقدمه‌ای بر تانسورها}

تانسورها ابزارهای ریاضی قدرتمندی هستند که برای توصیف کمیت‌های فیزیکی در مکانیک پیوسته و فیزیک استفاده می‌شوند. این ابجکت‌های ریاضی قابلیت نمایش روابط پیچیده‌ی بین متغیرهای مختلف را دارند و در عین حال تحت تبدیلات مختصات ناوردا باقی می‌مانند.

\section{ابزارهای ریاضی}

\subsection{تانسورها}

\paragraph{ناورداری} یک سیستم مختصات نمایانگر یک ناظر است - به شرطی که فاصله‌ی زمانی بین رویدادها غیرمرتبط باشد. با معرفی یک سیستم مختصات، امکان تخصیص مختصات به موجودیت‌ها فراهم می‌شود. مختصات نمایانگر یک شیء در سیستم مختصات هستند. تشکیل سیستم‌های مختصات مختلف، ناظر را با نمایش‌های متفاوت از همان شیء روبرو می‌کند. از آنجا که همه‌ی نمایش‌ها به همان شیء فیزیکی اشاره دارند، آنها به نوعی معادل هستند - یا به تعبیر بهتر، شیء \textit{ناوردا} است.

\paragraph{قوانین تبدیل} تانسورها برای توصیف کمیت‌های فیزیکی استفاده می‌شوند. نمایش تانسوری ترکیبی از پایه‌ی سیستم مختصات و مؤلفه‌های تانسور است. بنابراین، تغییر سیستم مختصات هم پایه‌های مختصات و هم مؤلفه‌های تانسور را تغییر خواهد داد. مطالبه‌ی ناورداری از یک تانسور با اعمال تساوی بین دو نمایش آن از همان کمیت انجام می‌شود. نتیجه، \textit{قوانین تبدیل} است که به وسیله‌ی آنها ساختارهای ناوردا تشکیل می‌شوند. به طور کلی‌تر، ناورداری به دلیل رابطه‌ی معکوس بین ژاکوبین‌ها در تبدیل مؤلفه‌های کوواریانت و کنتراواریانت تانسور حاصل می‌شود.

\paragraph{تعریف شهودی} اگرچه مفهوم ناورداری ممکن است به عنوان انگیزه‌ی اساسی برای استفاده از تانسورها در نظر گرفته شود، خصوصیات دیگر تانسور نیز در تعریف تانسورها دخیل هستند. در ادبیات، برخی سعی کرده‌اند تعریفی ارائه دهند با توضیح رفتار تانسور تحت تبدیلات پایه، در حالی که دیگران بیشتر بر عملکرد نگاشت آن متمرکز شده‌اند. به طور شهودی، تانسور یک شیء ریاضی است که اطلاعات زیر را در بر می‌گیرد:

\begin{enumerate}
    \item دستورالعمل‌هایی برای یک نگاشت خطی،
    \item مؤلفه‌های مرتبط بر حسب بردارهای پایه‌ی خاص، و
    \item کیفیت ناوردا بودن تحت تبدیل پایه.
\end{enumerate}

مؤلفه‌های یک تانسور با انتخاب بردارهای پایه تعیین می‌شوند و آنها طبق قوانین تبدیل خاصی از یک مجموعه پایه به پایه‌ی دیگر تغییر می‌کنند. بنابراین، مؤلفه‌های یک تانسور با پایه‌هایش مرتبط هستند، یعنی آنها در یک مجموعه پایه‌ی خاص مؤلفه‌های منحصر به فرد دارند. این مؤلفه‌ها اگر بردارهای پایه تغییر کنند تغییر خواهند کرد. با این وجود، مؤلفه‌های یک تانسور همراه با پایه‌هایش ناوردا هستند.

\begin{definition}[تانسور]\label{def:tensors}
    تانسور $\ten{T}$ از مرتبه‌ی $n$-ام در فضای $d$-بعدی به عنوان یک تابع خطی اسکالر-مقدار از $n$ بردار تعریف می‌شود:
    \begin{equation}
        \begin{aligned}
            \ten{T}: & \underbrace{\tena{u}\in\Rset^d\times\tena{v}\in\Rset^d\times\ldots  \times\tena{w}\in\Rset^d}_n   \mapsto \alpha\in\Rset \\
                     & \ten{T}[\tena{u},\tena{v},\ldots,\tena{w}]=\alpha,
        \end{aligned}
    \end{equation}
    که در آن $d$ بعد فضا، $\alpha$ یک اسکالر، و $n$ مرتبه (یا رتبه) تانسور است که با $\order{\ten{T}}=n$ نمایش داده می‌شود. بعد و مرتبه‌ی یک تانسور با هم به صورت $\class{n}{d}$ بیان می‌شوند که \textit{کلاس} یک تانسور را نشان می‌دهد، یعنی مجموعه‌ی همه‌ی تانسورهای $n$-ام مرتبه و $d$-بعدی. توجه کنید که یک کلاس عمومی‌تر (مجموعه) تنها با نشان دادن مرتبه‌ی تانسور مانند $\class{n}{}$ ساخته می‌شود.

    توجه کنید که هدف‌گیری یک مؤلفه‌ی خاص از تانسور به روش‌های مختلف نمایش داده می‌شود:
    \begin{equation}
        T_{ij\ldots k}\equiv[\ten{T}]_{ij\ldots k}\equiv (\ten{T})_{ij\ldots k}.
    \end{equation}
\end{definition}

برای تأکید بر خطی بودن تانسورها، می‌توان خط زیر را برای تکمیل تعریف اضافه کرد:
\begin{equation}
    \forall a,b,\ldots ,z\in \Rset:\qquad\ten{T}[a\tena{u},b\tena{v},\ldots,z\tena{w}]=ab\ldots z\,\ten{T}[\tena{u},\tena{v},\ldots,\tena{w}].
\end{equation}

\section{عملیات تانسوری}

\subsection{ضرب داخلی}
در جبر خطی، ضرب داخلی یک نگاشت دوخطی حقیقی-مقدار از دو بردار است که ضرب نقطه‌ای انتخاب استاندارد آن محسوب می‌شود. ضرب داخلی - به عنوان یک مفهوم اولیه و یک نرم - برای تنظیم فضا استفاده می‌شود. بدین ترتیب، مفاهیم ثانویه‌ی طول و زاویه تعریف می‌شوند. فضای حاصل یک فضای اقلیدسی است که چیزی جز فضای حقیقی مجهز به مفاهیم ضرب داخلی و نرم نیست.

\begin{definition}[ضرب داخلی]
    در تحلیل تانسوری، ضرب داخلی یک عملیات دودویی روی دو تانسور است که اغلب با نماد نقطه ($\scp$) نشان داده می‌شود. برای مثال، تانسور $\ten{T}$ به طور متوالی روی چندین بردار اعمال می‌شود تا آنها را به یک اسکالر نگاشت کند:
    \begin{equation}\label{eq:contr}
        \ten{T}[\tena{v},\ldots,\tena{w}] \equiv \ten{T} \odot (\tena{v} \dyad  \ldots \dyad \tena{w})=\mathcal{T}_{i\ldots j}v_{i} \ldots w_j.
    \end{equation}
    قدرت یک تانسور توسط ضرب داخلی محاسبه می‌شود:
    \begin{equation}
        \ten{T}^n=\underbrace{\ten{T}\scp\ldots\scp\ten{T}}_{n-1\ \text{ضرب داخلی}},\qquad\qquad\forall n\in\Iset^+.
    \end{equation}
\end{definition}

\subsection{ضرب خارجی}
ضرب خارجی برای ایجاد موجودیت‌های مرتبه‌ی بالاتر استفاده می‌شود. برای مثال، یک دیاد از ضرب خارجی دو بردار ساخته می‌شود.

\begin{definition}[ضرب خارجی]
    ضرب دو واریانت منجر به واریانت دیگری می‌شود. چنین ضربی \textit{ضرب خارجی} یا \textit{ضرب تانسوری} نامیده می‌شود:
    \begin{equation}
        \ten{A}=\ten{B}\dyad\ten{C}.
    \end{equation}
    در نتیجه‌ی ضرب تانسوری، مرتبه‌ی نتیجه افزایش می‌یابد، یعنی $\order{\ten{A}}=\order{\ten{B}}+\order{\ten{C}}$.

    قدرت تانسوری یک تانسور توسط ضرب خارجی محاسبه می‌شود:
    \begin{equation}
        \ten{T}^\tenpow{n}=\underbrace{\ten{T}\dyad\ldots\dyad\ten{T}}_{n-1\ \text{ضرب خارجی}},\qquad\qquad\forall n\in\Iset^+.
    \end{equation}
\end{definition}

\subsection{تبدیل}
عملیات تبدیل ($\op^\tran$) یا مزدوج (برای تانسورهای مختلط) عملیاتی است که ترتیب دیادیک یک تانسور را تغییر می‌دهد. در حالی که تبدیل تانسورهای مرتبه‌ی اول غیرمرتبط است، تنها یک نوع تبدیل برای تانسور مرتبه‌ی دوم $\tenb{T}$ قابل تعریف است:
\begin{equation}
    \forall \tena{u},\tena{v}\in \class{1}{n}:\qquad \tena{u}\scp\tenb{T}\scp\tena{v}=\tena{v}\scp\tenb{T}^\tran\scp\tena{u}.
\end{equation}

\section{کاربردهای تانسورها در مکانیک پیوسته}

تانسورها در مکانیک پیوسته نقش اساسی دارند و برای توصیف:
\begin{itemize}
    \item تنش و کرنش در مواد
    \item خصوصیات مکانیکی مواد
    \item میدان‌های سرعت و شتاب
    \item قوانین تشکیل‌دهنده مواد
\end{itemize}

استفاده می‌شوند. قابلیت ناورداری تانسورها آنها را برای توصیف پدیده‌های فیزیکی که باید مستقل از سیستم مختصات انتخابی باشند، مناسب می‌سازد.