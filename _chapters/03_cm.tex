\chapter{مرور مکانیک محیط های پیوسته}

\section{مقدمه‌}

\subsection{مفهوم پیوستار}

مکانیک پیوسته چارچوب ریاضی قدرتمندی برای مدل‌سازی رفتار مواد فراهم می‌کند که در آن ماده به‌جای در نظر گیری ساختار گسسته اتمی و مولکولی، به‌صورت پیوسته در سراسر نواحی اشغال شده توزیع می‌شود~\autocite{Abeyaratne.1987}.

فرضیه‌ی پیوستار نمایانگر فرض بنیادی است که خواص مادی می‌توانند به‌عنوان توابع پیوسته‌ی مکان و زمان در نظر گرفته شوند. این فرض امکان به‌کارگیری حساب دیفرانسیل و انتگرال را برای تحلیل رفتار مادی فراهم می‌کند.

\begin{keypoint}
	فرضیه‌ی پیوستار سنگ بنای مکانیک پیوسته است که امکان جایگزینی ساختار مولکولی گسسته‌ی ماده با محیط پیوسته‌ای را فراهم می‌کند که با توابع میدان هموار توصیف می‌شود.
\end{keypoint}

\subsection{اعتبار رویکرد پیوستار}

فرض پیوستار زمانی معتبر است که:
\begin{itemize}
	\item مقیاس مشخصه‌ی طول مسئله به‌مراتب بزرگ‌تر از فواصل بین‌مولکولی باشد
	\item تعداد مولکول‌ها در عنصر حجمی نماینده برای میانگین‌گیری آماری کافی باشد
	\item اثرات سطحی بر رفتار کلی غالب نباشند
\end{itemize}

کاربردهای معمول از مقیاس نانومتری (>۱۰ نانومتر) تا ابعاد ماکروسکوپی را شامل می‌شود.

\subsection{عناصر بنیادی مکانیک پیوسته}

توسعه‌ی کامل مکانیک پیوسته نیازمند چهار مؤلفه‌ی ضروری است:

\begin{enumerate}
	\item \textbf{سینماتیک}: توصیف ریاضی حرکت و تغییرشکل بدون ارجاع به نیروهای باعث آن‌ها. این شاخه شامل تعریف پیکربندی‌های مرجع و کنونی، نگاشت حرکت $\mathbf{x} = \boldsymbol{\chi}(\mathbf{X}, t)$، و اندازه‌گیری‌های مختلف کرنش مانند تانسورهای کرنش گرین-لاگرانژ و المانسی-اویلری می‌شود. سینماتیک همچنین شامل بررسی سازگاری کرنش و مفهوم تغییرات عناصر خط، سطح و حجم است.

	\item \textbf{تحلیل تنش}: مشخصه‌سازی نیروهای داخلی و توزیع آن‌ها در ماده. بر اساس اصل تنش کوشی، بردار تنش روی هر سطح با نرمال واحد $\mathbf{n}$ به‌صورت $\mathbf{t}^{(\mathbf{n})} = \boldsymbol{\sigma} \cdot \mathbf{n}$ تعریف می‌شود. تانسور تنش کوشی $\boldsymbol{\sigma}$ متقارن بوده و دارای تنش‌های اصلی و جهات اصلی مشخص است.

	\item \textbf{قوانین بقا}: اصول تعادل جهانی برای جرم، تکانه‌ی خطی، تکانه‌ی زاویه‌ای، و انرژی. این قوانین شامل معادله‌ی پیوستگی برای بقای جرم، معادلات حرکت کوشی برای بقای تکانه، و قانون اول ترمودینامیک برای بقای انرژی هستند. قضیه‌ی انتقال رینولدز پایه‌ی ریاضی برای تبدیل بیانیه‌های بقای سراسری به محلی فراهم می‌کند.

	\item \textbf{روابط ساختاری}: معادلات مخصوص مواد که تنش، کرنش، دما، و سایر متغیرهای میدان را به هم مربوط می‌کنند. این روابط باید اصول عینیت، تقارن مادی، و سازگاری ترمودینامیکی را رعایت کنند. نمونه‌هایی شامل قانون تعمیم‌یافته‌ی هوک برای الاستیسیته‌ی خطی، مدل‌های ویسکوالاستیک، و روابط پیچیده‌تر برای رفتارهای غیرخطی هستند.
\end{enumerate}

این مؤلفه‌ها چارچوب سیستماتیکی برای فرمول‌بندی مسائل مقدار مرزی در مکانیک پیوسته فراهم می‌کنند.

\begin{keypoint}
	چهار رکن مکانیک پیوسته—سینماتیک، تحلیل تنش، قوانین بقا، و روابط ساختاری—همگی باید برای توصیف کامل رفتار مادی حضور داشته باشند.
\end{keypoint}

\section{ابزارهای ریاضی}

\subsection{تحلیل تانسوری}

توصیف ریاضی پدیده‌های سه‌بعدی نیازمند تحلیل تانسوری است. تانسورها ابزارهای ریاضی قدرتمندی هستند که برای توصیف کمیت‌های فیزیکی در فضاهای چندبعدی و تحلیل رفتار آن‌ها تحت تغییرات مختصات استفاده می‌شوند. کمیت‌های فیزیکی به‌صورت زیر طبقه‌بندی می‌شوند:

\begin{itemize}
	\item \textbf{اسکالرها (تانسورهای مرتبه صفر)}: کمیت‌هایی که فقط بزرگی داشته و تحت تغییر مختصات ثابت می‌مانند. نمونه‌ها: دما $\theta$، چگالی $\rho$، انرژی $E$، و فشار $p$.

	\item \textbf{بردارها (تانسورهای مرتبه اول)}: کمیت‌هایی که علاوه بر بزرگی، جهت نیز داشته و قانون تبدیل خاصی تحت تغییر مختصات دارند:
	      \begin{equation}
		      \tena{u} = u_i \dir{e}_i, \quad \tena{v} = v_i \dir{e}_i, \quad \tena{F} = F_i \dir{e}_i
	      \end{equation}
	      نمونه‌ها: جابه‌جایی، سرعت، شتاب، نیرو.

	\item \textbf{تانسورهای مرتبه دوم}: این تانسورها برای توصیف کمیت‌هایی استفاده می‌شوند که ارتباط بین دو بردار را نشان می‌دهند یا ماتریس‌های $3 \times 3$ را نمایندگی می‌کنند:
	      \begin{equation}
		      \tenb{T} = T_{ij} \dir{e}_i \otimes \dir{e}_j
	      \end{equation}
	      نمونه‌ها: تانسور تنش $\tenb{\sigma}$، تانسور کرنش $\tenb{\varepsilon}$، تانسور ممان اینرسی، گرادیان تغییرشکل $\tenb{F}$.

	\item \textbf{تانسورهای مرتبه بالاتر}: این تانسورها برای توصیف خواص پیچیده‌تر مواد استفاده می‌شوند:
	      \begin{equation}
		      \tend{C} = C_{ijkl} \dir{e}_i \otimes \dir{e}_j \otimes \dir{e}_k \otimes \dir{e}_l
	      \end{equation}
	      نمونه‌ها: تانسور سختی الاستیک $\tend{C}$، ضرایب پیزوالکتریک، مدول‌های الاستیسیته.
\end{itemize}

\subsection{عملیات تانسوری اساسی}

عملیات‌های اساسی بین تانسورها شامل موارد زیر هستند:

\textbf{جمع و تفریق تانسورها:} تانسورهای هم‌مرتبه را می‌توان جمع یا تفریق کرد:
\begin{equation}
	(\tenb{A} + \tenb{B})_{ij} = A_{ij} + B_{ij}
\end{equation}

\textbf{ضرب داخلی (انقباض):} این عملیات منجر به کاهش مرتبه تانسور می‌شود:
\begin{equation}
	(\tenb{A} \scp \tena{b})_i = A_{ij} b_j, \quad \tenb{A} : \tenb{B} = A_{ij} B_{ij}
\end{equation}

\textbf{ضرب خارجی (تانسوری):} این عملیات منجر به افزایش مرتبه تانسور می‌شود:
\begin{equation}
	(\tena{a} \otimes \tena{b})_{ij} = a_i b_j
\end{equation}

\textbf{ردِ تانسور:} برای تانسورهای مرتبه دوم، رد به‌صورت زیر تعریف می‌شود:
\begin{equation}
	\text{tr}(\tenb{A}) = A_{ii} = A_{11} + A_{22} + A_{33}
\end{equation}

\textbf{دترمینان تانسورهای مرتبه دوم:} برای تانسور $\tenb{A}$:
\begin{equation}
	\det(\tenb{A}) = \varepsilon_{ijk} A_{1i} A_{2j} A_{3k}
\end{equation}

\subsection{نماد شاخص و قرارداد جمع اینشتین}

نماد شاخص روش کارآمدی برای کار با عبارات تانسوری فراهم می‌کند. بر اساس قرارداد جمع اینشتین، وقتی شاخصی در یک عبارت تکرار شود، جمع روی آن شاخص از $1$ تا $3$ (در فضای سه‌بعدی) در نظر گرفته می‌شود:

\begin{equation}
	a_i b_i = a_1 b_1 + a_2 b_2 + a_3 b_3 = \sum_{i=1}^{3} a_i b_i
\end{equation}

نمادهای مهم تانسوری عبارتند از:

\begin{itemize}
	\item \textbf{دلتای کرونکر}:
	      \begin{equation}
		      \delta_{ij} = \begin{cases} 1 & \text{اگر } i = j \\ 0 & \text{اگر } i \neq j \end{cases}
	      \end{equation}

	\item \textbf{تانسور جایگشت لوی-سیویتا}:
	      \begin{equation}
		      \varepsilon_{ijk} = \begin{cases}
			      +1 & \text{اگر } (i,j,k) \text{ جایگشت زوج } (1,2,3) \text{ باشد} \\
			      -1 & \text{اگر } (i,j,k) \text{ جایگشت فرد } (1,2,3) \text{ باشد} \\
			      0  & \text{اگر دو یا سه شاخص برابر باشند}
		      \end{cases}
	      \end{equation}

	\item \textbf{تانسور واحد مرتبه دوم}: $\unitb = \delta_{ij} \dir{e}_i \otimes \dir{e}_j$
\end{itemize}

\subsection{عمل‌گرهای دیفرانسیل در تحلیل تانسوری}

عمل‌گرهای دیفرانسیل ابزارهای اساسی برای تجزیه و تحلیل میدان‌های پیوسته هستند:

\textbf{گرادیان:} برای میدان اسکالری $\phi$ و میدان برداری $\tena{u}$:
\begin{equation}
	\nabla \phi = \frac{\partial \phi}{\partial x_i} \dir{e}_i, \quad \nabla \tena{u} = \frac{\partial u_j}{\partial x_i} \dir{e}_i \otimes \dir{e}_j
\end{equation}

\textbf{واگرایی:} برای میدان برداری $\tena{u}$ و میدان تانسوری $\tenb{T}$:
\begin{equation}
	\nabla \scp \tena{u} = \frac{\partial u_i}{\partial x_i}, \quad \nabla \scp \tenb{T} = \frac{\partial T_{ij}}{\partial x_i} \dir{e}_j
\end{equation}

\textbf{روتور:} برای میدان برداری $\tena{u}$:
\begin{equation}
	\nabla \times \tena{u} = \varepsilon_{ijk} \frac{\partial u_k}{\partial x_j} \dir{e}_i
\end{equation}

\textbf{لاپلاسین:} برای میدان اسکالری $\phi$:
\begin{equation}
	\nabla^2 \phi = \frac{\partial^2 \phi}{\partial x_i \partial x_i}
\end{equation}

\begin{keypoint}
	عمل‌گرهای دیفرانسیل ابزارهای اساسی برای فرمول‌بندی قوانین بقا و تعادل در مکانیک پیوسته هستند و پایه‌ی ریاضی معادلات تعادل، معادلات حرکت، و روابط سازگاری را تشکیل می‌دهند.
\end{keypoint}

\section{سینماتیک حرکت و تغییرشکل}

\subsection{پیکربندی و توصیف حرکت}

توصیف سینماتیکی پایه‌ی مکانیک پیوسته را با فراهم کردن ابزارهای ریاضی برای توصیف حرکت و تغییرشکل بدون ارجاع به نیروهای باعث آن‌ها تشکیل می‌دهد.

\begin{keypoint}
	سینماتیک توصیف کاملاً هندسی از حرکت و تغییرشکل، مستقل از نیروها و خواص مادی، ارائه می‌دهد که پایه‌ی ضروری برای تحلیل تنش و مدل‌سازی ساختاری را تشکیل می‌دهد.
\end{keypoint}

\subsection{جسم مادی و پیکربندی‌ها}

جسم مادی $\mathcal{B}$ را به‌عنوان مجموعه‌ای پیوسته از ذرات یا نقاط مادی $\mathbf{X}$ تعریف می‌کنیم. این ذرات نقاط جرمی گسسته مانند مکانیک نیوتونی نیستند، بلکه بخش‌های بی‌نهایت کوچک یک محیط پیوسته با چگالی جرمی قابل تعریف هستند. برای هر یک از این ذرات، نگاشت یک‌به‌یکی به نقاط فضایی $\mathbf{X}$ در فضای اقلیدسی سه‌بعدی که ذرات در لحظه‌ای معین $t_0$ اشغال می‌کنند، تعریف می‌کنیم.

\textbf{پیکربندی مرجع} $\kappa_0$: پیکربندی انتخاب شده (معمولاً بدون تنش یا پیکربندی اولیه در $t = 0$) که در آن ذرات مادی با بردارهای موقعیت $\tena{X}$ شناسایی می‌شوند. انتخاب پیکربندی مرجع کاملاً اختیاری است.

\textbf{پیکربندی کنونی} $\kappa_t$: پیکربندی در زمان $t$ که ذرات موقعیت‌های $\tena{x}$ را اشغال می‌کنند.

\subsection{نگاشت حرکت}

حرکت پیوستار با نگاشت زیر توصیف می‌شود:
\begin{equation}
	\tena{x} = \boldsymbol{\chi}(\tena{X}, t)
\end{equation}

بنابراین، ذره‌ی $\mathbf{X}$ در موقعیت $\mathbf{X}$ در پیکربندی مرجع به موقعیت جدید $\mathbf{x}$ در پیکربندی کنونی در زمان $t$ منتقل می‌شود. زمانی که $t = t_0$، رابطه فوق $\mathbf{X} = \boldsymbol{\chi}(\mathbf{X}, t_0)$ را می‌دهد.

این تابع باید شرایط زیر را برآورده کند:
\begin{itemize}
	\item پیوستگی: ذرات همسایه، همسایه باقی می‌مانند (عدم نفوذپذیری ماده)
	\item معکوس‌پذیری: تناظر یک‌به‌یک بین پیکربندی‌ها، به‌طوری که حرکت معکوس $\mathbf{X} = \boldsymbol{\chi}^{-1}(\mathbf{x}, t)$ وجود داشته باشد
	\item مشتق‌پذیری: حرکت و معکوس آن توابع پیوسته و مشتق‌پذیر باشند
\end{itemize}

تحت این شرایط، دترمینان ژاکوبین $J = \det(\partial \mathbf{x}/\partial \mathbf{X})$ نمی‌تواند صفر شود و در واقع فرض می‌کنیم:
\begin{equation}
	0 < \det\left(\frac{\partial \mathbf{x}}{\partial \mathbf{X}}\right) < \infty
\end{equation}

\begin{keypoint}
	نگاشت تغییرشکل $\tena{x} = \boldsymbol{\chi}(\tena{X}, t)$ باید پیوسته، معکوس‌پذیر، و مشتق‌پذیر باشد، با $J > 0$ تا اطمینان حاصل شود که ماده به خود نفوذ نمی‌کند.
\end{keypoint}

\section{توصیف‌های لاگرانژی و اویلری}

معادله‌ی حرکت $x_i = \chi_i(X_1, X_2, X_3, t)$ می‌تواند به‌عنوان رابطه‌ای بین مختصات مرجع یا مادی $X_i$ و مختصات کنونی یا فضایی $x_i$ در نظر گرفته شود. بنابراین، اگر مختصات مادی یک ذره‌ی مشخص را بدانیم، این رابطه اجازه می‌دهد موقعیت ذره در پیکربندی کنونی را تعیین کنیم. به همین ترتیب، معادله‌ی معکوس $X_i = \chi_i^{-1}(x_1, x_2, x_3, t)$ رابطه‌ی مخالف را ارائه می‌دهد.

همه‌ی متغیرهای میدان فضایی در مکانیک پیوسته (چگالی، دما، جابه‌جایی، کرنش، تنش و غیره) می‌توانند بر حسب مختصات مادی $X_i$ یا مختصات فضایی $x_i$ توصیف شوند.

\subsection{توصیف لاگرانژی (مادی)}

ذرات مادی منفرد را در طول حرکتشان دنبال می‌کند:
\begin{itemize}
	\item متغیرهای مستقل: $(\tena{X}, t)$
	\item تمرکز: چه بر ذره‌ی مشخصی می‌گذرد
	\item مناسب برای: کاربردهای مکانیک جامدات
\end{itemize}

با دنبال کردن ذرات، می‌توانیم کمیت‌های تانسوری را به‌عنوان توابعی که با مختصات مادی $(X_1, X_2, X_3)$ شناسایی می‌شوند، بیان کنیم. چنین توصیفی به‌عنوان توصیف لاگرانژی، مادی یا مرجع شناخته می‌شود.

\subsection{توصیف اویلری (فضایی)}

نقاط ثابت فضا را مشاهده می‌کند:
\begin{itemize}
	\item متغیرهای مستقل: $(\tena{x}, t)$
	\item تمرکز: چه در موقعیت ثابت می‌گذرد
	\item مناسب برای: کاربردهای مکانیک سیالات
\end{itemize}

با استفاده از طرح دیگر، می‌توانیم تغییرات را در موقعیت‌های ثابت مشاهده کنیم و بنابراین کمیت‌های تانسوری را به‌عنوان توابع مختصات موقعیت $(x_1, x_2, x_3)$ بیان کنیم. چنین توصیفی به‌عنوان توصیف اویلری یا فضایی شناخته می‌شود. توجه کنید که با گذشت زمان، ذرات مختلف موقعیت فضایی یکسانی را اشغال خواهند کرد، و بنابراین توصیف فضایی اطلاعات مشخصی در مورد خواص ذرات در طول حرکت ارائه نمی‌دهد.

\subsection{مشتق مادی}

نرخ تغییر زمانی یک کمیت تانسوری که ذره‌ی مادی را دنبال می‌کند، به‌عنوان مشتق زمانی مادی شناخته شده و معمولاً با $D/Dt$ نشان داده می‌شود. زمانی که توصیف مادی یک میدان تانسوری مشخص $T$ استفاده می‌شود، چنین مشتقی به روش مستقیم محاسبه می‌شود:
\begin{equation}
	\frac{DT}{Dt} = \frac{\partial T(X_1, X_2, X_3, t)}{\partial t}\bigg|_{X_i \text{ ثابت}}
\end{equation}

با این حال، زمانی که توصیف فضایی برای تانسور $T$ استفاده می‌شود، مشتق زمانی کمی پیچیده‌تر است زیرا مختصات فضایی خود اکنون توابعی از زمان هستند. این امر نیازمند استفاده از قانون زنجیره‌ای است:
\begin{equation}
	\frac{D\phi}{Dt} = \frac{\partial \phi}{\partial t} + \tena{v} \scp \nabla \phi
\end{equation}

که در آن $\tena{v}$ بردار سرعت است و $\nabla\phi$ گرادیان فضایی $\phi$ است.

\section{اندازه‌گیری تغییرشکل}

\subsection{گرادیان تغییرشکل}

تانسور گرادیان تغییرشکل $\tenb{F}$ کمیت کلیدی در سینماتیک محدود است که اطلاعات کاملی در مورد تغییرشکل محلی ارائه می‌دهد:
\begin{equation}
	\tenb{F} = \frac{\partial \tena{x}}{\partial \tena{X}} = \nabla_0 \tena{x}
\end{equation}

این تانسور نشان‌دهنده‌ی چگونگی نگاشت عناصر خط بی‌نهایت کوچک از پیکربندی مرجع به پیکربندی کنونی است. اگر $d\tena{X}$ عنصر خط بی‌نهایت کوچک در پیکربندی مرجع باشد، آنگاه عنصر خط متناظر در پیکربندی کنونی $d\tena{x} = \tenb{F} \cdot d\tena{X}$ خواهد بود.

دترمینان گرادیان تغییرشکل $J = \det(\tenb{F})$ نسبت تغییر حجم محلی را نمایندگی می‌کند. برای ماده‌ی غیرقابل تراکم، $J = 1$ و برای مواد قابل تراکم، $J > 0$ باید باشد تا از نفوذ ماده جلوگیری شود.

\subsection{اندازه‌های کرنش}

\subsubsection{تانسور کرنش گرین-لاگرانژ}

تانسور کرنش گرین-لاگرانژ اندازه‌ی لاگرانژی کرنش است که نسبت به پیکربندی مرجع تعریف می‌شود:
\begin{equation}
	\tenb{E} = \frac{1}{2}(\tenb{F}^\tran \scp \tenb{F} - \unitb) = \frac{1}{2}(\tenb{C} - \unitb)
\end{equation}

که $\tenb{C} = \tenb{F}^\tran \scp \tenb{F}$ تانسور تغییرشکل راست کوشی-گرین است.

این تانسور دارای خواص مهم زیر است:
\begin{itemize}
	\item تانسور متقارن است: $E_{ij} = E_{ji}$
	\item در صورت عدم تغییرشکل، $\tenb{E} = \mathbf{0}$
	\item برای چرخش صلب خالص، $\tenb{E} = \mathbf{0}$
	\item مؤلفه‌های قطری $E_{ii}$ نشان‌دهنده‌ی کرنش‌های نرمال هستند
	\item مؤلفه‌های غیرقطری $E_{ij}$ ($i \neq j$) نشان‌دهنده‌ی کرنش‌های برشی هستند
\end{itemize}

برای تغییرشکل‌های کوچک، این تانسور به تانسور کرنش خطی تبدیل می‌شود.

\subsubsection{تانسور کرنش المانسی-اویلری}

تانسور کرنش المانسی-اویلری اندازه‌ی اویلری کرنش است که نسبت به پیکربندی کنونی تعریف می‌شود:
\begin{equation}
	\tenb{e} = \frac{1}{2}(\unitb - \tenb{b}^{-1})
\end{equation}

که $\tenb{b} = \tenb{F} \scp \tenb{F}^\tran$ تانسور تغییرشکل چپ کوشی-گرین است.

تفاوت‌های اساسی با تانسور گرین-لاگرانژ:
\begin{itemize}
	\item در چارچوب فضایی تعریف می‌شود
	\item برای تحلیل‌های اویلری مناسب‌تر است
	\item برای سیالات و مواد با تغییرشکل‌های بزرگ به‌کار می‌رود
	\item رابطه‌ی تبدیل: $\tenb{e} = \tenb{F}^{-\tran} \cdot \tenb{E} \cdot \tenb{F}^{-1}$
\end{itemize}

هر دو تانسور در حد تغییرشکل‌های کوچک به تانسور کرنش خطی همگرا می‌شوند، اما برای تغییرشکل‌های بزرگ تفاوت‌های قابل توجهی دارند.

\subsubsection{تانسور کرنش بی‌نهایت کوچک}
برای تغییرشکل‌های کوچک:
\begin{equation}
	\boldsymbol{\varepsilon} = \frac{1}{2}(\nabla \tena{u} + (\nabla \tena{u})^\tran)
\end{equation}

که $\tena{u} = \tena{x} - \tena{X}$ بردار جابه‌جایی است.

\subsection{تجزیه‌ی قطبی}

قضیه‌ی تجزیه‌ی قطبی یکی از نتایج بنیادی در نظریه‌ی ماتریس است که در مکانیک پیوسته کاربرد مهمی دارد. طبق این قضیه، هر تانسور گرادیان تغییرشکل $\tenb{F}$ قابل تجزیه‌ی یکتا به صورت:
\begin{equation}
	\tenb{F} = \tenb{R} \scp \tenb{U} = \tenb{V} \scp \tenb{R}
\end{equation}

که در آن:
\begin{itemize}
	\item $\tenb{R}$: تانسور چرخش (متعامد با $\det(\tenb{R}) = +1$)
	\item $\tenb{U}$: تانسور کشش راست (متقارن مثبت معین)
	\item $\tenb{V}$: تانسور کشش چپ (متقارن مثبت معین)
\end{itemize}

\textbf{تفسیر فیزیکی}: تجزیه‌ی $\tenb{F} = \tenb{R} \scp \tenb{U}$ نشان می‌دهد که تغییرشکل می‌تواند به‌عنوان کشش خالص توسط $\tenb{U}$ در امتداد محورهای اصلی‌اش، و سپس چرخش صلب توسط $\tenb{R}$ تفسیر شود. تجزیه‌ی $\tenb{F} = \tenb{V} \scp \tenb{R}$ اول چرخش صلب و سپس کشش را انجام می‌دهد.

\textbf{روابط بین تانسورهای کشش}:
\begin{align}
	\tenb{U} & = \sqrt{\tenb{C}} = \sqrt{\tenb{F}^\tran \cdot \tenb{F}} \\
	\tenb{V} & = \sqrt{\tenb{b}} = \sqrt{\tenb{F} \cdot \tenb{F}^\tran} \\
	\tenb{V} & = \tenb{R} \cdot \tenb{U} \cdot \tenb{R}^\tran
\end{align}

مقادیر ویژه‌ی $\tenb{U}$ و $\tenb{V}$ یکسان بوده و کشش‌های اصلی $\lambda_i$ نامیده می‌شوند.

\section{نیرو و اندازه‌های تنش}

\subsection{نیروهای بدنی و سطحی}

قبل از ورود به مفهوم تنش، باید نیروهای داخلی محیط‌های پیوسته را بررسی کنیم. در چارچوب مکانیک پیوسته، نیروهای داخلی به دو دسته‌ی اصلی تقسیم می‌شوند:

\textbf{نیروهای بدنی} (Body Forces): این نیروها متناسب با جرم جسم بوده و با عوامل خارجی واکنش می‌دهند. مثال‌هایی از این نیروها عبارتند از:
\begin{itemize}
	\item نیروی وزن ناشی از جاذبه زمین
	\item نیروهای مغناطیسی
	\item نیروهای اینرسی در مراجع غیراینرسی
\end{itemize}

چگالی نیروی بدنی $\tena{b}(\tena{x}, t)$ به‌صورت نیرو در واحد جرم تعریف می‌شود، به‌طوری که کل نیروی بدنی $\tena{F}_R$ روی جسم $\mathcal{B}$ عبارت است از:
\begin{equation}
	\tena{F}_R = \int_{\mathcal{B}} \rho \tena{b}(\tena{x}, t) \, d\mathcal{V}
\end{equation}

\textbf{نیروهای سطحی} (Surface Forces): این نیروها همواره روی سطح عمل کرده و نتیجه‌ی تماس فیزیکی با اجسام دیگر هستند. کل نیروی سطحی $\tena{F}_S$ روی سطح $S$ به‌صورت زیر محاسبه می‌شود:
\begin{equation}
	\tena{F}_S = \int_{S} \tena{t}(\tena{x}, t) \, dS
\end{equation}

که در آن $\tena{t}(\tena{x}, t)$ چگالی نیروی سطحی یا بردار تنش (traction vector) نامیده می‌شود.

\subsection{اصل تنش کوشی}

اصل بنیادی تنش کوشی که پایه‌ی نظریه‌ی تنش مدرن را تشکیل می‌دهد، بیان می‌کند که بردار تنش در هر نقطه از ماده وابسته به موقعیت مکانی و جهت نرمال سطح است. برای تعیین کمی این رابطه، سطح کوچکی با مساحت $\Delta a$ و بردار نرمال واحد $\tena{n}$ در نظر می‌گیریم. نیروی کل $\Delta \tena{F}$ عمل‌کننده روی این سطح در حد $\Delta a \to 0$ به‌صورت زیر تعریف می‌شود:

\begin{equation}
	\tena{t}(\tena{x}, t, \tena{n}) = \lim_{\Delta a \to 0} \frac{\Delta \tena{F}}{\Delta a}
\end{equation}

رابطه‌ی بنیادی اصل تنش کوشی عبارت است از:
\begin{equation}
	\tena{t}^{(\dir{n})} = \tenb{\sigma} \scp \dir{n}
\end{equation}

که در آن $\tena{t}^{(\dir{n})}$ بردار تنش روی سطح با نرمال $\dir{n}$ و $\tenb{\sigma}$ تانسور تنش کوشی است.

همچنین، اصل عکس‌العمل (قانون سوم نیوتون) نیز به‌صورت زیر بیان می‌شود:
\begin{equation}
	\tena{t}(\tena{x}, \tena{n}) = -\tena{t}(\tena{x}, -\tena{n})
\end{equation}

\begin{keypoint}
	اصل تنش کوشی بیان می‌کند که بردار تنش در هر نقطه بر روی سطحی با نرمال واحد $\dir{n}$ به‌صورت خطی به نرمال سطح وابسته است و امکان تعیین حالت تنش در نقطه با تعداد محدودی مؤلفه فراهم می‌کند.
\end{keypoint}

\subsection{تانسور تنش کوشی}

برای درک کامل حالت تنش در یک نقطه، مؤلفه‌های تانسور تنش کوشی روی سطوح مختصاتی تعریف می‌شوند. در صورتی که سطح کوچک $\Delta a$ با هر یک از سه صفحه‌ی مختصاتی منطبق شود، بردارهای تنش روی هر وجه به‌صورت زیر بیان می‌شوند:

\begin{align}
	\tena{t}(\tena{x}, \tena{n} = \dir{e}_1) & = T_{11} \dir{e}_1 + T_{12} \dir{e}_2 + T_{13} \dir{e}_3 \\
	\tena{t}(\tena{x}, \tena{n} = \dir{e}_2) & = T_{21} \dir{e}_1 + T_{22} \dir{e}_2 + T_{23} \dir{e}_3 \\
	\tena{t}(\tena{x}, \tena{n} = \dir{e}_3) & = T_{31} \dir{e}_1 + T_{32} \dir{e}_2 + T_{33} \dir{e}_3
\end{align}

نه مؤلفه‌ی $T_{ij}$ تانسور تنش کوشی نامیده می‌شوند که در قالب ماتریسی به‌صورت زیر نمایش داده می‌شوند:
\begin{equation}
	[\tenb{T}] = \begin{bmatrix}
		T_{11} & T_{12} & T_{13} \\
		T_{21} & T_{22} & T_{23} \\
		T_{31} & T_{32} & T_{33}
	\end{bmatrix}
\end{equation}

مؤلفه‌های قطری $T_{11}$، $T_{22}$، و $T_{33}$ \textbf{تنش‌های نرمال} نامیده شده و مؤلفه‌های غیرقطری $T_{12}$، $T_{21}$، $T_{23}$، $T_{32}$، $T_{31}$، و $T_{13}$ \textbf{تنش‌های برشی} هستند.

\subsubsection{فرمول تنش کوشی}

برای تعیین بردار تنش روی سطح مایل با نرمال دلخواه، از تعادل المان چهاروجهی که توسط سطح مایل و سه صفحه‌ی مختصاتی محدود شده است، استفاده می‌کنیم. با اعمال قانون دوم نیوتون و در نظر گیری حد $\Delta V \to 0$، رابطه‌ی زیر به دست می‌آید:

\begin{equation}
	t_i = T_{ji} n_j \quad \text{یا به‌صورت برداری:} \quad \tena{t} = \tenb{T}^T \cdot \tena{n}
\end{equation}

این رابطه که \textbf{فرمول تنش کوشی} نامیده می‌شود، روشی مستقیم و ساده برای محاسبه‌ی نیروها روی سطوح مایل فراهم می‌کند.

\subsubsection{اثبات تانسوری بودن تنش کوشی}

برای اثبات اینکه تنش کوشی تانسور مرتبه‌ی دوم است، از این حقیقت استفاده می‌کنیم که بردارهای تنش و نرمال تانسورهای مرتبه‌ی اول هستند. با شروع از فرمول کوشی در چارچوب مرجع جدید:
\begin{equation}
	t'_i = T'_{ji} n'_j
\end{equation}

و با استفاده از قوانین تبدیل تانسوری، نشان داده می‌شود که:
\begin{equation}
	T'_{ij} = Q_{im} Q_{jn} T_{mn}
\end{equation}

که قانون تبدیل استاندارد برای تانسورهای مرتبه‌ی دوم است.

\subsection{خواص تانسور تنش}

\subsubsection{تقارن تانسور تنش}
از تعادل تکانه‌ی زاویه‌ای (که در فصل بعد اثبات خواهد شد):
\begin{equation}
	\sigma_{ij} = \sigma_{ji}
\end{equation}

این تقارن تعداد مؤلفه‌های مستقل تانسور تنش را از 9 به 6 کاهش می‌دهد.

\subsubsection{تنش‌های اصلی و جهات اصلی}
مقادیر ویژه تانسور تنش که \textbf{تنش‌های اصلی} نامیده می‌شوند، از حل معادله‌ی مشخصه زیر به دست می‌آیند:
\begin{equation}
	\det(\tenb{\sigma} - \sigma \unitb) = 0
\end{equation}

که معادل معادله‌ی مکعبی زیر است:
\begin{equation}
	-\sigma^3 + I_T \sigma^2 - II_T \sigma + III_T = 0
\end{equation}

که در آن ناورداهای بنیادی تنش عبارتند از:
\begin{align}
	I_T   & = T_{ii} = T_{11} + T_{22} + T_{33}                                                                                                                                                                                                                   \\
	II_T  & = \frac{1}{2}(T_{ii} T_{jj} - T_{ij} T_{ij}) = \begin{vmatrix} T_{11} & T_{12} \\ T_{21} & T_{22} \end{vmatrix} + \begin{vmatrix} T_{22} & T_{23} \\ T_{32} & T_{33} \end{vmatrix} + \begin{vmatrix} T_{11} & T_{13} \\ T_{31} & T_{33} \end{vmatrix} \\
	III_T & = \det \tenb{T}
\end{align}

در سیستم مختصات اصلی، تانسور تنش به فرم قطری زیر درمی‌آید:
\begin{equation}
	[\tenb{T}] = \begin{bmatrix}
		T_1 & 0   & 0   \\
		0   & T_2 & 0   \\
		0   & 0   & T_3
	\end{bmatrix}
\end{equation}

\subsubsection{تجزیه‌ی کروی-انحرافی}
تانسور تنش می‌تواند به دو بخش کروی (هیدرواستاتیک) و انحرافی تجزیه شود:
\begin{align}
	\tenb{\sigma} & = \sigma_m \unitb + \tenb{s}                                                         \\
	\sigma_m      & = \frac{1}{3} \text{tr}(\tenb{\sigma}) = \frac{1}{3}(\sigma_1 + \sigma_2 + \sigma_3) \\
	\tenb{s}      & = \tenb{\sigma} - \sigma_m \unitb
\end{align}

بخش کروی مسئول تغییر حجم و بخش انحرافی مسئول تغییر شکل (بدون تغییر حجم) است.

\subsection{دایره‌های موهر و تحلیل تنش}

برای تحلیل حالت‌های تنش دوبعدی، روش دایره‌های موهر ابزار قدرتمندی فراهم می‌کند. با در نظر گیری تنش‌های اصلی مرتب‌شده $T_1 > T_2 > T_3$، مؤلفه‌های نرمال و برشی بردار تنش روی سطح دلخواه با نرمال $\tena{n}$ به‌صورت زیر بیان می‌شوند:

\begin{align}
	N   & = T_1 n_1^2 + T_2 n_2^2 + T_3 n_3^2             \\
	S^2 & = T_1^2 n_1^2 + T_2^2 n_2^2 + T_3^2 n_3^2 - N^2
\end{align}

با شرط $n_1^2 + n_2^2 + n_3^2 = 1$، سه دایره موهر با معادلات زیر تعریف می‌شوند:

\begin{align}
	S^2 + (N - T_2)(N - T_3) & \geq 0 \\
	S^2 + (N - T_3)(N - T_1) & \leq 0 \\
	S^2 + (N - T_1)(N - T_2) & \geq 0
\end{align}

بیشینه تنش برشی برابر با $S_{\max} = \frac{1}{2}|T_1 - T_3|$ است که شعاع بزرگ‌ترین دایره موهر را تشکیل می‌دهد.

\subsection{تنش‌های ویژه}

\subsubsection{تنش هشت‌وجهی}

صفحه‌ی هشت‌وجهی صفحه‌ای است که نرمال آن زوایای مساوی با سه محور اصلی می‌سازد. مؤلفه‌های نرمال آن $n_i = \pm(1,1,1)/\sqrt{3}$ هستند. تنش‌های نرمال و برشی هشت‌وجهی عبارتند از:

\begin{align}
	\sigma_{\text{oct}} & = \frac{1}{3}(T_1 + T_2 + T_3) = \frac{1}{3} I_T                  \\
	\tau_{\text{oct}}   & = \frac{1}{3}\sqrt{(T_1 - T_2)^2 + (T_2 - T_3)^2 + (T_3 - T_1)^2}
\end{align}

تنش برشی هشت‌وجهی مستقیماً با انرژی کرنش اعوجاجی مرتبط است که در نظریه‌های شکست مواد نرم کاربرد دارد.

\subsubsection{تنش مؤثر فون میزس}

تنش مؤثر یا فون میزس که معیار مهمی در نظریه‌های خرابی محسوب می‌شود، به‌صورت زیر تعریف می‌شود:

\begin{align}
	\sigma_e = \sigma_{\text{von Mises}} & = \sqrt{\frac{3}{2} \tenb{\hat{s}} : \tenb{\hat{s}}}                     \\
	                                     & = \frac{1}{\sqrt{2}}\sqrt{(T_1 - T_2)^2 + (T_2 - T_3)^2 + (T_3 - T_1)^2} \\
	                                     & = \sqrt{3} \tau_{\text{oct}}
\end{align}

که در آن $\tenb{\hat{s}}$ تانسور تنش انحرافی است. این تنش با تنش برشی هشت‌وجهی از رابطه‌ی $\sigma_e = \sqrt{3/2} \tau_{\text{oct}}$ مرتبط است.

\subsection{توزیع‌های تنش و خطوط کانتور}

تجسم و درک طبیعت توزیع تنش در جامدات از اهمیت بالایی برخوردار است. روش‌های مختلفی برای این منظور توسعه یافته‌اند:

\textbf{خطوط هم‌رنگ} (Isochromatic lines): خطوطی که اختلاف تنش‌های اصلی روی آن‌ها ثابت است: $T_1 - T_2 = \text{ثابت}$

\textbf{خطوط هم‌میل} (Isoclinic lines): خطوطی که جهت تنش‌های اصلی روی آن‌ها ثابت است.

\textbf{خطوط هم‌پاک} (Isopachic lines): خطوطی که مجموع تنش‌های نرمال روی آن‌ها ثابت است: $T_{11} + T_{22} = T_1 + T_2 = \text{ثابت}$

\textbf{مسیرهای تنش} (Stress trajectories): خطوطی که در جهت تنش‌های اصلی قرار دارند و دو خانواده متعامد تشکیل می‌دهند. این خطوط برای درک مسیرهای انتقال بار از نقاط اعمال نیرو به نقاط تکیه‌گاه بسیار مفیدند.

برای مسیرهای تنش دوبعدی، زاویه‌ی جهت $\theta_p$ نسبت به محور $x_1$ از رابطه‌ی زیر محاسبه می‌شود:
\begin{equation}
	\tan 2\theta_p = \frac{2T_{12}}{T_{11} - T_{22}}
\end{equation}

معادله‌ی دیفرانسیل مسیرهای تنش به‌صورت زیر است:
\begin{equation}
	\frac{dy}{dx} = \frac{T_{11} - T_{22}}{2T_{12}} \pm \sqrt{\left(\frac{T_{11} - T_{22}}{2T_{12}}\right)^2 + 1}
\end{equation}

\subsection{تانسورهای تنش پیولا-کیرشهف}

در تغییرشکل‌های محدود، تمایز بین مساحت‌های مرجع و کنونی اهمیت پیدا می‌کند. تانسور تنش کوشی که قبلاً مطرح شد، تنش واقعی در پیکربندی کنونی را نمایندگی می‌کند. اما در برخی کاربردها، فرمول‌بندی مسئله در پیکربندی مرجع مزایایی دارد که به تعریف تانسورهای تنش جدید منجر می‌شود.

\subsubsection{تانسور تنش پیولا-کیرشهف اول}

در پیکربندی مرجع، سطح $dA$ با نرمال واحد $\tena{N}$ وجود دارد که در پیکربندی کنونی به سطح $da$ با نرمال $\tena{n}$ تبدیل می‌شود. اصل تعادل نیرو بیان می‌کند که:
\begin{equation}
	\tena{T}^R dA = \tena{t} da
\end{equation}

که در آن $\tena{T}^R$ بردار تنش کاذب (pseudo-traction) در پیکربندی مرجع است. تانسور تنش پیولا-کیرشهف اول $\tenb{T}^o$ از رابطه‌ی زیر تعریف می‌شود:
\begin{equation}
	\tena{T}^R = \tenb{T}^o \cdot \tena{N}
\end{equation}

با استفاده از فرمول نانسون برای تبدیل سطح و ترکیب روابط، رابطه‌ی زیر حاصل می‌شود:
\begin{equation}
	\tenb{T}^o = J \tenb{T} \cdot \tenb{F}^{-T}
\end{equation}

که در آن $J = \det(\tenb{F})$ ژاکوبین تغییرشکل است. رابطه‌ی معکوس نیز به‌صورت زیر است:
\begin{equation}
	\tenb{T} = J^{-1} \tenb{T}^o \cdot \tenb{F}^T
\end{equation}

\textbf{ویژگی مهم}: تانسور پیولا-کیرشهف اول در حالت کلی متقارن نیست، حتی اگر تانسور کوشی متقارن باشد.

\subsubsection{تانسور تنش پیولا-کیرشهف دوم}

تانسور تنش پیولا-کیرشهف دوم $\tenb{S}$ از طریق عملیات pull-back تعریف می‌شود:
\begin{equation}
	\tenb{S} \tena{N} dA = \tenb{F}^{-1} \tena{t} da
\end{equation}

که منجر به روابط زیر می‌شود:
\begin{equation}
	\tenb{S} = J \tenb{F}^{-1} \cdot \tenb{T} \cdot \tenb{F}^{-T}
\end{equation}

رابطه‌ی معکوس:
\begin{equation}
	\tenb{T} = J^{-1} \tenb{F} \cdot \tenb{S} \cdot \tenb{F}^T
\end{equation}

\textbf{ویژگی مهم}: تانسور پیولا-کیرشهف دوم متقارن است اگر تانسور کوشی متقارن باشد.

\subsubsection{رابطه بین تانسورهای پیولا-کیرشهف}

دو تانسور پیولا-کیرشهف از رابطه‌ی زیر به هم مربوط هستند:
\begin{equation}
	\tenb{T}^o = \tenb{F} \cdot \tenb{S}
\end{equation}

\subsubsection{حد تغییرشکل‌های کوچک}

برای تغییرشکل‌های کوچک که $\nabla \tena{u} = \mathcal{O}(\varepsilon)$ با $\varepsilon \ll 1$، شرایط زیر برقرار است:
\begin{align}
	\tenb{F}      & \approx \unitb + \nabla \tena{u}            \\
	\tenb{F}^{-1} & \approx \unitb - \nabla \tena{u}            \\
	J             & \approx 1 + \nabla \cdot \tena{u} \approx 1
\end{align}

در این حالت، هر سه تانسور تنش (کوشی، PK1، و PK2) به هم همگرا می‌شوند:
\begin{equation}
	\tenb{T} \approx \tenb{T}^o \approx \tenb{S}
\end{equation}

\subsection{سایر تانسورهای تنش}

\subsubsection{تانسور تنش کیرشهف}
\begin{equation}
	\boldsymbol{\tau} = J \tenb{T}
\end{equation}

این تانسور گاهی در نظریه‌های پلاستیسیته استفاده می‌شود.

\subsubsection{تانسور تنش بیوت}
\begin{equation}
	\tenb{T}^B = \tenb{R}^T \cdot \tenb{T}^o = \tenb{U} \cdot \tenb{S}
\end{equation}

که در آن $\tenb{R}$ تانسور چرخش از تجزیه‌ی قطبی $\tenb{F} = \tenb{R} \cdot \tenb{U}$ است.

\subsubsection{عینیت تانسورهای تنش}

بررسی عینیت تانسورهای مختلف تنش نتایج زیر را می‌دهد:

\begin{itemize}
	\item \textbf{تانسور کوشی $\tenb{T}$}: عینی است و قانون تبدیل $\tenb{T}^* = \tenb{Q} \cdot \tenb{T} \cdot \tenb{Q}^T$ را دنبال می‌کند.
	\item \textbf{تانسور PK1 $\tenb{T}^o$}: برای تانسورهای مرتبه‌ی اول عینی است اما برای مرتبه‌ی دوم عینی نیست.
	\item \textbf{تانسور PK2 $\tenb{S}$}: عینی نیست و در هر دو چارچوب مرجع یکسان باقی می‌ماند.
	\item \textbf{تانسور کیرشهف $\boldsymbol{\tau}$}: عینی است.
	\item \textbf{تانسور بیوت $\tenb{T}^B$}: عینی نیست.
\end{itemize}

\section{قوانین بقا و تعادل}

قوانین بقا یا تعادل اصول بنیادی فیزیکی هستند که رفتار همه‌ی مواد پیوسته را تحت تأثیر قرار می‌دهند. این قوانین که از سال‌ها تحقیق توسعه یافته‌اند، برای تغییرشکل‌ها، بارگذاری‌ها، و اثرات نرخی که در کاربردهای مهندسی معمول یافت می‌شوند، مناسب هستند. این روابط معمولاً نوعی اصل بقا را نمایندگی می‌کنند و بر همه‌ی مواد پیوسته اعمال می‌شوند، صرف‌نظر از اینکه جامد، سیال، الاستیک، یا پلاستیک باشند.

\begin{keypoint}
	قوانین بقا اصول جهانی هستند که بر همه‌ی مواد پیوسته اعمال می‌شوند و پایه‌ی ریاضی معادلات میدان در مکانیک پیوسته را تشکیل می‌دهند.
\end{keypoint}

\subsection{اصول کلی بقا و قضیه‌ی انتقال رینولدز}

اساساً اصول تعادل به‌عنوان اصولی شروع می‌شوند که شامل روابط انتگرالی بر روی پیکربندی‌های جسم مادی هستند. ابتدا نرخ تغییرات زمانی انتگرال‌های خاصی را در نظر می‌گیریم.

برای ناحیه‌ی ثابت فضا $R$ و با $G(\tena{x}, t)$ به‌عنوان میدان تانسوری دلخواه:
\begin{equation}
	\frac{\partial}{\partial t} \int_R G \, dv = \int_R \frac{\partial G}{\partial t} \, dv
\end{equation}

مشتق زمانی از علامت انتگرال حجمی عبور می‌کند زیرا حدود انتگرال‌گیری مستقل از زمان هستند.

با این حال، ما می‌خواهیم این نرخ تغییر را بر روی گروه ثابتی از ذرات پیوستار که ناحیه‌ی فضایی $R_m$ را در نقطه‌ای خاص از زمان اشغال می‌کنند، در نظر بگیریم. برای این حالت، نه تنها انتگرال‌گیرنده با زمان تغییر می‌کند، بلکه حجم فضایی که انتگرال بر روی آن گرفته می‌شود نیز تغییر می‌کند. بنابراین می‌خواهیم مشتق زمانی مادی انتگرال حجمی را به‌گونه‌ای تعریف کنیم که نرخ تغییر کل مقدار کمیتی را که توسط سیستم جرمی داده شده در $R_m$ حمل می‌شود، اندازه‌گیری کند:

\begin{equation}
	\frac{D}{Dt} \int_{R_m} G \, dV
\end{equation}

با استفاده از روابط ژاکوبین و تبدیل از مختصات مادی به فضایی، می‌توانیم بنویسیم:
\begin{equation}
	\frac{D}{Dt} \int_{R_m} G \, dv = \int_R \left( \frac{DG}{Dt} + G v_{k,k} \right) dv
\end{equation}

با استفاده از قضیه‌ی واگرایی در انتگرال دوم:
\begin{equation}
	\frac{D}{Dt} \int_{R_m} G \, dV = \int_R \frac{\partial G}{\partial t} \, dv + \int_{\partial R} G v_k n_k \, ds
\end{equation}

که در آن $n_k$ بردار نرمال واحد خروجی به سطح $\partial R$ است. نتیجه‌ی (\ref{eq:reynolds}) اغلب \textbf{قضیه‌ی انتقال رینولدز} نامیده می‌شود.

\begin{keypoint}
	قضیه‌ی انتقال رینولدز پایه‌ی ریاضی برای تبدیل بیانیه‌های بقای سراسری به شکل محلی است و نشان می‌دهد که نرخ تغییر مادی شامل دو جمله است: یکی مربوط به نرخ تغییر ساده درون ناحیه و دیگری مربوط به مقدار ورودی (شار) از مرز.
\end{keypoint}

بیشتر قوانین فیزیکی ما ابتدا در قالب معادله‌ی کلی بقا یا حفظ به‌صورت زیر بیان می‌شوند:
\begin{equation}
	\frac{D}{Dt} \int_{R_m} \rho \Psi \, dV = -\int_{\partial R} \tena{I} \scp \tena{n} \, ds + \int_R \rho S \, dv
\end{equation}

که در آن $\Psi$ میدان تانسوری، $\tena{I}$ جمله‌ی ورودی ناشی از انتقال از مرز $\partial R$، $S$ جمله‌ی منبع داخلی، و علامت منفی به این دلیل لازم است که $\tena{n}$ نرمال خروجی است.

\subsection{بقای جرم}

با در نظر گیری بخش دلخواهی از مادی $R_m$، اصل کلی بقای جرم به سادگی بیان می‌شود که کل جرم در $R_m$ باید در همه‌ی زمان‌ها ثابت باقی بماند:
\begin{equation}
	\frac{D}{Dt} \int_{R_m} \rho \, dV = 0
\end{equation}

با ترکیب قضیه‌ی انتقال رینولدز:
\begin{equation}
	\int_R \left( \frac{\partial \rho}{\partial t} + (\rho v_k)_{,k} \right) dv = 0
\end{equation}

با استفاده از قضیه‌ی محلی‌سازی، انتگرال‌گیرنده مشترک باید صفر باشد:
\begin{equation}
	\frac{\partial \rho}{\partial t} + (\rho v_k)_{,k} = 0
\end{equation}

یا در شکل بردار:
\begin{equation}
	\frac{\partial \rho}{\partial t} + \nabla \scp (\rho \tena{v}) = 0
\end{equation}

رابطه‌ی (\ref{eq:continuity}) به‌عنوان \textbf{بیانیه‌ی دیفرانسیل بقای جرم} شناخته شده و رابطه‌ای نقطه‌ای است که در همه‌ی نقاط پیوستار درون $R$ به‌کار می‌رود. این رابطه اغلب به‌عنوان \textbf{معادله‌ی پیوستگی} در مکانیک سیالات شناخته می‌شود.

این رابطه را می‌توان در شکل جایگزین زیر نوشت:
\begin{equation}
	\frac{D\rho}{Dt} + \rho v_{k,k} = 0 \quad \text{یا} \quad \frac{D\rho}{Dt} + \rho \nabla \scp \tena{v} = 0
\end{equation}

توجه کنید که اگر ماده غیرقابل تراکم باشد، آنگاه $D\rho/Dt = 0$ و بنابراین:
\begin{equation}
	v_{k,k} = \nabla \scp \tena{v} = 0
\end{equation}

\subsection{بقای تکانه‌ی خطی}

اصل بقای تکانه‌ی خطی اساساً بیانیه‌ای از قانون دوم نیوتون برای مجموعه‌ای از ذرات است. می‌تواند بیان شود که نرخ تغییر زمانی کل تکانه‌ی خطی گروه داده شده‌ای از ذرات پیوستار برابر با مجموع همه‌ی نیروهای خارجی عمل‌کننده بر گروه است.

با در نظر گیری گروه ثابتی از ذرات پیوستار که لحظه‌ای ناحیه‌ی فضایی $R$ را اشغال می‌کنند و تحت نیروهای سطحی $\tena{t}$ و نیروهای بدنی $\tena{b}$ قرار دارند:
\begin{equation}
	\frac{D}{Dt} \int_{R_m} \rho \tena{v} \, dv = \int_{\partial R} \tena{t} \, ds + \int_R \rho \tena{b} \, dv
\end{equation}

با تبدیل به نماد شاخص و معرفی تانسور تنش کوشی:
\begin{equation}
	\frac{D}{Dt} \int_{R_m} \rho v_i \, dv = \int_{\partial R} T_{ji} n_j \, ds + \int_R \rho b_i \, dv
\end{equation}

با استفاده از قضیه‌ی واگرایی:
\begin{equation}
	\frac{D}{Dt} \int_{R_m} \rho v_i \, dv = \int_R T_{ji,j} \, dv + \int_R \rho b_i \, dv
\end{equation}

با استفاده از قضیه‌ی انتقال رینولدز و بقای جرم:
\begin{equation}
	\int_R \rho \frac{Dv_i}{Dt} \, dv = \int_R (T_{ji,j} + \rho b_i) \, dv
\end{equation}

با استفاده از قضیه‌ی محلی‌سازی:
\begin{equation}
	T_{ji,j} + \rho b_i = \rho a_i
\end{equation}

یا در شکل بردار:
\begin{equation}
	\nabla \scp \tenb{T} + \rho \tena{b} = \rho \tena{a}
\end{equation}

که در آن $\tena{a} = D\tena{v}/Dt$ میدان شتاب است. روابط (\ref{eq:cauchy-motion}) به‌عنوان \textbf{معادلات حرکت کوشی} شناخته می‌شوند که شکل دیفرانسیل بقای تکانه‌ی خطی هستند.

برای مورد تغییرشکل‌های کوچک، عبارت شتاب $\tena{a}(\tena{x}, t) = \partial^2 \tena{u}/\partial t^2$ در معادلات حرکت استفاده می‌شود. وقتی مسئله در تعادل استاتیکی است که شتاب ناچیز یا صفر است، معادلات (\ref{eq:cauchy-motion}) به معادلات تعادل کاهش می‌یابند:
\begin{equation}
	T_{ji,j} + \rho b_i = 0
\end{equation}

\subsection{بقای تکانه‌ی زاویه‌ای}

اصل تکانه‌ی زاویه‌ای یا گشتاور تکانه شکل دیگری از قانون دوم نیوتون است. برای کاربرد ما بیان می‌کند که نرخ تغییر زمانی کل تکانه‌ی زاویه‌ای گروهی از ذرات باید برابر با مجموع همه‌ی گشتاورهای خارجی عمل‌کننده بر سیستم نسبت به نقطه‌ی دلخواهی در فضا باشد.

با در نظر گیری گروه ثابتی از ذرات پیوستار که لحظه‌ای ناحیه‌ی فضایی $R$ را اشغال می‌کنند و تحت نیروهای سطحی $\tena{t}$ و نیروهای بدنی $\tena{b}$ قرار دارند، و با نادیده گرفتن جفت‌های توزیع شده‌ی بدنی یا سطحی:
\begin{equation}
	\frac{D}{Dt} \int_{R_m} \tena{r} \times \rho \tena{v} \, dv = \int_{\partial R} \tena{r} \times \tena{t} \, ds + \int_R \tena{r} \times \rho \tena{b} \, dv
\end{equation}

که در آن $\tena{r}$ بردار مکان نسبت به نقطه‌ی دلخواهی است.

پس از انجام محاسبات تانسوری دقیق و استفاده از معادلات حرکت، نتیجه‌ی نهایی این است:
\begin{equation}
	\varepsilon_{ijk} T_{jk} = 0
\end{equation}

از آنجا که $\varepsilon_{ijk}$ پادمتقارن است، $T_{jk}$ باید متقارن باشد:
\begin{equation}
	T_{ij} = T_{ji}
\end{equation}

رابطه‌ی (\ref{eq:stress-symmetry}) گاهی \textbf{قانون دوم حرکت کوشی} نامیده می‌شود که نتیجه‌ی تعادل تکانه‌ی زاویه‌ای است. این نتیجه به فرض عدم وجود جفت‌های توزیع شده‌ی بدنی یا سطحی متکی است.

\subsection{بقای انرژی}

اصل بقای انرژی که معمولاً به‌عنوان \textbf{قانون اول ترمودینامیک} شناخته می‌شود، بیان می‌کند که نرخ تغییر انرژی جنبشی و درونی گروه داده شده‌ای از ذرات پیوستار برابر با مجموع نرخ تغییر کار انجام شده توسط نیروهای خارجی و انرژی ورودی به سیستم از مرز است.

انرژی جنبشی گروه:
\begin{equation}
	K = \int_R \frac{1}{2} \rho v^2 \, dv
\end{equation}

انرژی درونی:
\begin{equation}
	E = \int_R \rho \varepsilon \, dv
\end{equation}

که در آن $\varepsilon = \varepsilon(\tena{x}, t)$ چگالی انرژی درونی در واحد جرم است.

نرخ کار نیروهای خارجی (قدرت مکانیکی خارجی):
\begin{equation}
	P_{ext} = \int_{\partial R} \tena{t} \scp \tena{v} \, ds + \int_R \rho \tena{b} \scp \tena{v} \, dv
\end{equation}

این رابطه را می‌توان در شکل‌های جایگزین بازنویسی کرد. با کار بر روی جمله‌ی انتگرال سطحی و استفاده از فرمول کوشی و قضیه‌ی واگرایی:
\begin{equation}
	P_{ext} = \int_R [(T_{ji,j} + \rho b_i) v_i + T_{ij} D_{ij}] \, dv
\end{equation}

با استفاده از معادلات حرکت:
\begin{equation}
	P_{ext} = \int_R [\rho a_i v_i + T_{ij} D_{ij}] \, dv
\end{equation}

جمله‌ی $\int_R T_{ij} D_{ij} \, dv$ معمولاً به‌عنوان \textbf{قدرت تنش} شناخته می‌شود. دو متغیر $T_{ij}$ و $D_{ij}$ اغلب \textbf{مزدوج‌های انرژی} نامیده می‌شوند.

برای مطالعه‌ی ترمومکانیکی ما، انرژی ورودی به سیستم $R$ هم از مرز $\partial R$ و هم از منابع داخلی را می‌توان به‌صورت زیر نوشت:
\begin{equation}
	Q = -\int_{\partial R} \tena{q} \scp \tena{n} \, ds + \int_R \rho h \, dv
\end{equation}

که در آن $\tena{q}$ نرخ شار حرارت در واحد مساحت و $h$ منبع انرژی ویژه (تأمین) در واحد جرم است.

بیانیه‌ی کلی تعادل انرژی:
\begin{equation}
	\dot{K} + \dot{E} = P_{ext} + Q
\end{equation}

پس از جایگذاری نتایج مشخص برای قطعات مختلف انرژی و استفاده از قضایای انتقال رینولدز و واگرایی، و نهایتاً استفاده از قضیه‌ی محلی‌سازی:
\begin{equation}
	\rho \dot{\varepsilon} - T_{ij} D_{ij} + q_{i,i} - \rho h = 0
\end{equation}

یا در نماد بردار:
\begin{equation}
	\rho \dot{\varepsilon} - \text{tr}(\tenb{T} \tenb{D}) + \nabla \scp \tena{q} - \rho h = 0
\end{equation}

که شکل دیفرانسیل معادله‌ی تعادل انرژی برای مکانیک پیوستار ترمومکانیکی است.

\subsection{قانون دوم ترمودینامیک - نابرابری آنتروپی}

قانون اول ترمودینامیک که در بخش قبل توسعه یافت، می‌تواند به‌عنوان اندازه‌گیری تبدیل‌پذیری گرما و کار در حین حفظ تعادل انرژی مناسب تلقی شود. با این حال، این عبارت هیچ محدودیتی در جهت چنین فرآیندهای تبدیل‌پذیری ارائه نمی‌دهد.

آنتروپی به‌عنوان متغیری تعریف می‌شود که می‌تواند به‌عنوان اندازه‌گیری بی‌نظمی میکروسکوپی یا اختلال سیستم پیوستار تفسیر شود. در ترمودینامیک کلاسیک، معمولاً به‌عنوان تابع حالتی مرتبط با انتقال حرارت تعریف می‌شود.

برای فرآیند برگشت‌پذیر، آنتروپی در واحد جرم $s(\tena{x}, t)$ معمولاً از رابطه‌ی زیر تعریف می‌شود:
\begin{equation}
	ds = \left( \frac{\delta q}{\theta} \right)_{rev}
\end{equation}

که در آن $\theta$ دمای مطلق (مقیاس کلوین، همیشه مثبت) و $\delta q$ ورودی حرارت در واحد جرم بر روی فرآیند برگشت‌پذیر است.

برای فرآیندهای برگشت‌ناپذیر (دنیای واقعی)، مشاهدات نشان می‌دهند که:
\begin{equation}
	\oint \left( \frac{\delta q}{\theta} \right)_{irrev} < 0
\end{equation}

از آنجا که $\delta q / \theta$ را به‌عنوان ورودی آنتروپی از ورودی حرارت $\delta q$ تفسیر می‌کنیم، نتیجه می‌گیریم که بر روی یک چرخه‌ی برگشت‌ناپذیر، ورودی آنتروپی خالص منفی است. این به معنای آن است که فرآیندهای اتلافی برگشت‌ناپذیر تولید آنتروپی داخلی مثبت ایجاد می‌کنند.

برای استفاده در مکانیک پیوسته، قانون دوم معمولاً در شکل متفاوتی بازنویسی می‌شود. برای گروه ثابتی از ذرات پیوستار $R_m$ که ناحیه‌ی فضایی $R$ را اشغال می‌کنند:

نرخ ورودی آنتروپی:
\begin{equation}
	\int_R \frac{\rho h}{\theta} \, dv - \int_{\partial R} \frac{\tena{q} \scp \tena{n}}{\theta} \, ds
\end{equation}

طبق رابطه قانون دوم، نرخ افزایش آنتروپی در $R$ باید بزرگ‌تر یا برابر (برای حالت برگشت‌پذیر) با نرخ ورودی آنتروپی باشد:
\begin{equation}
	\frac{D}{Dt} \int_{R_m} \rho s \, dv \geq \int_R \frac{\rho h}{\theta} \, dv - \int_{\partial R} \frac{\tena{q} \scp \tena{n}}{\theta} \, ds
\end{equation}

با استفاده از روش‌های معمول روی فرمول‌بندی‌های انتگرالی و قضیه‌ی واگرایی:
\begin{equation}
	\int_R \left[ \rho \dot{s} - \frac{\rho h}{\theta} + \frac{1}{\theta} \left( \frac{q_i}{\theta} \right)_{,i} \right] dv \geq 0
\end{equation}

که منجر به:
\begin{equation}
	\rho \dot{s} \geq \frac{\rho h}{\theta} - \frac{1}{\theta^2} q_i \theta_{,i}
\end{equation}

روابط (\ref{eq:clausius-duhem-integral}) و (\ref{eq:clausius-duhem-differential}) به‌عنوان شکل‌های انتگرالی و دیفرانسیل \textbf{نابرابری کلاؤزیوس-دوهم} شناخته می‌شوند که اشکالی از قانون دوم ترمودینامیک برای کاربردهای مکانیک پیوسته هستند.

با استفاده از معادله‌ی انرژی، نابرابری آنتروپی را می‌توان به‌صورت زیر بیان کرد:
\begin{equation}
	\rho(\theta \dot{s} - \dot{\varepsilon}) + T_{ij} D_{ij} - \frac{1}{\theta} q_i \theta_{,i} \geq 0
\end{equation}

که گاهی به‌عنوان \textbf{نابرابری کلاؤزیوس-دوهم کاهش یافته} یا \textbf{نابرابری اتلاف} شناخته می‌شود.
