\chapter{مرور مکانیک محیط های پیوسته}

\section{مقدمه‌}

\subsection{مفهوم پیوستار}

مکانیک پیوسته چارچوب ریاضی قدرتمندی برای مدل‌سازی رفتار مواد فراهم می‌کند که در آن ماده به‌جای در نظر گیری ساختار گسسته اتمی و مولکولی، به‌صورت پیوسته در سراسر نواحی اشغال شده توزیع می‌شود~\autocite{Abeyaratne.1987}.

فرضیه‌ی پیوستار نمایانگر فرض بنیادی است که خواص مادی می‌توانند به‌عنوان توابع پیوسته‌ی مکان و زمان در نظر گرفته شوند. این فرض امکان به‌کارگیری حساب دیفرانسیل و انتگرال را برای تحلیل رفتار مادی فراهم می‌کند.

\begin{keypoint}
	فرضیه‌ی پیوستار سنگ بنای مکانیک پیوسته است که امکان جایگزینی ساختار مولکولی گسسته‌ی ماده با محیط پیوسته‌ای را فراهم می‌کند که با توابع میدان هموار توصیف می‌شود.
\end{keypoint}

\subsection{اعتبار رویکرد پیوستار}

فرض پیوستار زمانی معتبر است که:
\begin{itemize}
	\item مقیاس مشخصه‌ی طول مسئله به‌مراتب بزرگ‌تر از فواصل بین‌مولکولی باشد
	\item تعداد مولکول‌ها در عنصر حجمی نماینده برای میانگین‌گیری آماری کافی باشد
	\item اثرات سطحی بر رفتار کلی غالب نباشند
\end{itemize}

کاربردهای معمول از مقیاس نانومتری (>۱۰ نانومتر) تا ابعاد ماکروسکوپی را شامل می‌شود.

\subsection{عناصر بنیادی مکانیک پیوسته}

توسعه‌ی کامل مکانیک پیوسته نیازمند چهار مؤلفه‌ی ضروری است:

\begin{enumerate}
	\item \textbf{سینماتیک}: توصیف ریاضی حرکت و تغییرشکل بدون ارجاع به نیروهای باعث آن‌ها. این شاخه شامل تعریف پیکربندی‌های مرجع و کنونی، نگاشت حرکت $\mathbf{x} = \boldsymbol{\chi}(\mathbf{X}, t)$، و اندازه‌گیری‌های مختلف کرنش مانند تانسورهای کرنش گرین-لاگرانژ و المانسی-اویلری می‌شود. سینماتیک همچنین شامل بررسی سازگاری کرنش و مفهوم تغییرات عناصر خط، سطح و حجم است.

	\item \textbf{تحلیل تنش}: مشخصه‌سازی نیروهای داخلی و توزیع آن‌ها در ماده. بر اساس اصل تنش کوشی، بردار تنش روی هر سطح با نرمال واحد $\mathbf{n}$ به‌صورت $\mathbf{t}^{(\mathbf{n})} = \boldsymbol{\sigma} \cdot \mathbf{n}$ تعریف می‌شود. تانسور تنش کوشی $\boldsymbol{\sigma}$ متقارن بوده و دارای تنش‌های اصلی و جهات اصلی مشخص است.

	\item \textbf{قوانین بقا}: اصول تعادل جهانی برای جرم، تکانه‌ی خطی، تکانه‌ی زاویه‌ای، و انرژی. این قوانین شامل معادله‌ی پیوستگی برای بقای جرم، معادلات حرکت کوشی برای بقای تکانه، و قانون اول ترمودینامیک برای بقای انرژی هستند. قضیه‌ی انتقال رینولدز پایه‌ی ریاضی برای تبدیل بیانیه‌های بقای سراسری به محلی فراهم می‌کند.

	\item \textbf{روابط ساختاری}: معادلات مخصوص مواد که تنش، کرنش، دما، و سایر متغیرهای میدان را به هم مربوط می‌کنند. این روابط باید اصول عینیت، تقارن مادی، و سازگاری ترمودینامیکی را رعایت کنند. نمونه‌هایی شامل قانون تعمیم‌یافته‌ی هوک برای الاستیسیته‌ی خطی، مدل‌های ویسکوالاستیک، و روابط پیچیده‌تر برای رفتارهای غیرخطی هستند.
\end{enumerate}

این مؤلفه‌ها چارچوب سیستماتیکی برای فرمول‌بندی مسائل مقدار مرزی در مکانیک پیوسته فراهم می‌کنند.

\begin{keypoint}
	چهار رکن مکانیک پیوسته—سینماتیک، تحلیل تنش، قوانین بقا، و روابط ساختاری—همگی باید برای توصیف کامل رفتار مادی حضور داشته باشند.
\end{keypoint}

\section{ابزارهای ریاضی}

\subsection{تحلیل تانسوری}

توصیف ریاضی پدیده‌های سه‌بعدی نیازمند تحلیل تانسوری است. تانسورها ابزارهای ریاضی قدرتمندی هستند که برای توصیف کمیت‌های فیزیکی در فضاهای چندبعدی و تحلیل رفتار آن‌ها تحت تغییرات مختصات استفاده می‌شوند. کمیت‌های فیزیکی به‌صورت زیر طبقه‌بندی می‌شوند:

\begin{itemize}
	\item \textbf{اسکالرها (تانسورهای مرتبه صفر)}: کمیت‌هایی که فقط بزرگی داشته و تحت تغییر مختصات ثابت می‌مانند. نمونه‌ها: دما $\theta$، چگالی $\rho$، انرژی $E$، و فشار $p$.
	
	\item \textbf{بردارها (تانسورهای مرتبه اول)}: کمیت‌هایی که علاوه بر بزرگی، جهت نیز داشته و قانون تبدیل خاصی تحت تغییر مختصات دارند:
	\begin{equation}
		\tena{u} = u_i \dir{e}_i, \quad \tena{v} = v_i \dir{e}_i, \quad \tena{F} = F_i \dir{e}_i
	\end{equation}
	نمونه‌ها: جابه‌جایی، سرعت، شتاب، نیرو.
	
	\item \textbf{تانسورهای مرتبه دوم}: این تانسورها برای توصیف کمیت‌هایی استفاده می‌شوند که ارتباط بین دو بردار را نشان می‌دهند یا ماتریس‌های $3 \times 3$ را نمایندگی می‌کنند:
	\begin{equation}
		\tenb{T} = T_{ij} \dir{e}_i \otimes \dir{e}_j
	\end{equation}
	نمونه‌ها: تانسور تنش $\tenb{\sigma}$، تانسور کرنش $\tenb{\varepsilon}$، تانسور ممان اینرسی، گرادیان تغییرشکل $\tenb{F}$.
	
	\item \textbf{تانسورهای مرتبه بالاتر}: این تانسورها برای توصیف خواص پیچیده‌تر مواد استفاده می‌شوند:
	\begin{equation}
		\tend{C} = C_{ijkl} \dir{e}_i \otimes \dir{e}_j \otimes \dir{e}_k \otimes \dir{e}_l
	\end{equation}
	نمونه‌ها: تانسور سختی الاستیک $\tend{C}$، ضرایب پیزوالکتریک، مدول‌های الاستیسیته.
\end{itemize}

\subsection{عملیات تانسوری اساسی}

عملیات‌های اساسی بین تانسورها شامل موارد زیر هستند:

\textbf{جمع و تفریق تانسورها:} تانسورهای هم‌مرتبه را می‌توان جمع یا تفریق کرد:
\begin{equation}
	(\tenb{A} + \tenb{B})_{ij} = A_{ij} + B_{ij}
\end{equation}

\textbf{ضرب داخلی (انقباض):} این عملیات منجر به کاهش مرتبه تانسور می‌شود:
\begin{equation}
	(\tenb{A} \scp \tena{b})_i = A_{ij} b_j, \quad \tenb{A} : \tenb{B} = A_{ij} B_{ij}
\end{equation}

\textbf{ضرب خارجی (تانسوری):} این عملیات منجر به افزایش مرتبه تانسور می‌شود:
\begin{equation}
	(\tena{a} \otimes \tena{b})_{ij} = a_i b_j
\end{equation}

\textbf{ردِ تانسور:} برای تانسورهای مرتبه دوم، رد به‌صورت زیر تعریف می‌شود:
\begin{equation}
	\text{tr}(\tenb{A}) = A_{ii} = A_{11} + A_{22} + A_{33}
\end{equation}

\textbf{دترمینان تانسورهای مرتبه دوم:} برای تانسور $\tenb{A}$:
\begin{equation}
	\det(\tenb{A}) = \varepsilon_{ijk} A_{1i} A_{2j} A_{3k}
\end{equation}

\subsection{نماد شاخص و قرارداد جمع اینشتین}

نماد شاخص روش کارآمدی برای کار با عبارات تانسوری فراهم می‌کند. بر اساس قرارداد جمع اینشتین، وقتی شاخصی در یک عبارت تکرار شود، جمع روی آن شاخص از $1$ تا $3$ (در فضای سه‌بعدی) در نظر گرفته می‌شود:

\begin{equation}
	a_i b_i = a_1 b_1 + a_2 b_2 + a_3 b_3 = \sum_{i=1}^{3} a_i b_i
\end{equation}

نمادهای مهم تانسوری عبارتند از:

\begin{itemize}
	\item \textbf{دلتای کرونکر}: 
	\begin{equation}
		\delta_{ij} = \begin{cases} 1 & \text{اگر } i = j \\ 0 & \text{اگر } i \neq j \end{cases}
	\end{equation}
	
	\item \textbf{تانسور جایگشت لوی-سیویتا}: 
	\begin{equation}
		\varepsilon_{ijk} = \begin{cases} 
		+1 & \text{اگر } (i,j,k) \text{ جایگشت زوج } (1,2,3) \text{ باشد} \\
		-1 & \text{اگر } (i,j,k) \text{ جایگشت فرد } (1,2,3) \text{ باشد} \\
		0 & \text{اگر دو یا سه شاخص برابر باشند}
		\end{cases}
	\end{equation}
	
	\item \textbf{تانسور واحد مرتبه دوم}: $\unitb = \delta_{ij} \dir{e}_i \otimes \dir{e}_j$
\end{itemize}

\subsection{عمل‌گرهای دیفرانسیل در تحلیل تانسوری}

عمل‌گرهای دیفرانسیل ابزارهای اساسی برای تجزیه و تحلیل میدان‌های پیوسته هستند:

\textbf{گرادیان:} برای میدان اسکالری $\phi$ و میدان برداری $\tena{u}$:
\begin{equation}
	\nabla \phi = \frac{\partial \phi}{\partial x_i} \dir{e}_i, \quad \nabla \tena{u} = \frac{\partial u_j}{\partial x_i} \dir{e}_i \otimes \dir{e}_j
\end{equation}

\textbf{واگرایی:} برای میدان برداری $\tena{u}$ و میدان تانسوری $\tenb{T}$:
\begin{equation}
	\nabla \scp \tena{u} = \frac{\partial u_i}{\partial x_i}, \quad \nabla \scp \tenb{T} = \frac{\partial T_{ij}}{\partial x_i} \dir{e}_j
\end{equation}

\textbf{روتور:} برای میدان برداری $\tena{u}$:
\begin{equation}
	\nabla \times \tena{u} = \varepsilon_{ijk} \frac{\partial u_k}{\partial x_j} \dir{e}_i
\end{equation}

\textbf{لاپلاسین:} برای میدان اسکالری $\phi$:
\begin{equation}
	\nabla^2 \phi = \frac{\partial^2 \phi}{\partial x_i \partial x_i}
\end{equation}

\begin{keypoint}
عمل‌گرهای دیفرانسیل ابزارهای اساسی برای فرمول‌بندی قوانین بقا و تعادل در مکانیک پیوسته هستند و پایه‌ی ریاضی معادلات تعادل، معادلات حرکت، و روابط سازگاری را تشکیل می‌دهند.
\end{keypoint}

\section{سینماتیک حرکت و تغییرشکل}

\subsection{پیکربندی و توصیف حرکت}

توصیف سینماتیکی پایه‌ی مکانیک پیوسته را با فراهم کردن ابزارهای ریاضی برای توصیف حرکت و تغییرشکل بدون ارجاع به نیروهای باعث آن‌ها تشکیل می‌دهد.

\begin{keypoint}
	سینماتیک توصیف کاملاً هندسی از حرکت و تغییرشکل، مستقل از نیروها و خواص مادی، ارائه می‌دهد که پایه‌ی ضروری برای تحلیل تنش و مدل‌سازی ساختاری را تشکیل می‌دهد.
\end{keypoint}

\subsection{جسم مادی و پیکربندی‌ها}

جسم مادی $\mathcal{B}$ را به‌عنوان مجموعه‌ای پیوسته از ذرات یا نقاط مادی $\mathbf{X}$ تعریف می‌کنیم. این ذرات نقاط جرمی گسسته مانند مکانیک نیوتونی نیستند، بلکه بخش‌های بی‌نهایت کوچک یک محیط پیوسته با چگالی جرمی قابل تعریف هستند. برای هر یک از این ذرات، نگاشت یک‌به‌یکی به نقاط فضایی $\mathbf{X}$ در فضای اقلیدسی سه‌بعدی که ذرات در لحظه‌ای معین $t_0$ اشغال می‌کنند، تعریف می‌کنیم.

\textbf{پیکربندی مرجع} $\kappa_0$: پیکربندی انتخاب شده (معمولاً بدون تنش یا پیکربندی اولیه در $t = 0$) که در آن ذرات مادی با بردارهای موقعیت $\tena{X}$ شناسایی می‌شوند. انتخاب پیکربندی مرجع کاملاً اختیاری است.

\textbf{پیکربندی کنونی} $\kappa_t$: پیکربندی در زمان $t$ که ذرات موقعیت‌های $\tena{x}$ را اشغال می‌کنند.

\subsection{نگاشت حرکت}

حرکت پیوستار با نگاشت زیر توصیف می‌شود:
\begin{equation}
	\tena{x} = \boldsymbol{\chi}(\tena{X}, t)
\end{equation}

بنابراین، ذره‌ی $\mathbf{X}$ در موقعیت $\mathbf{X}$ در پیکربندی مرجع به موقعیت جدید $\mathbf{x}$ در پیکربندی کنونی در زمان $t$ منتقل می‌شود. زمانی که $t = t_0$، رابطه فوق $\mathbf{X} = \boldsymbol{\chi}(\mathbf{X}, t_0)$ را می‌دهد.

این تابع باید شرایط زیر را برآورده کند:
\begin{itemize}
	\item پیوستگی: ذرات همسایه، همسایه باقی می‌مانند (عدم نفوذپذیری ماده)
	\item معکوس‌پذیری: تناظر یک‌به‌یک بین پیکربندی‌ها، به‌طوری که حرکت معکوس $\mathbf{X} = \boldsymbol{\chi}^{-1}(\mathbf{x}, t)$ وجود داشته باشد
	\item مشتق‌پذیری: حرکت و معکوس آن توابع پیوسته و مشتق‌پذیر باشند
\end{itemize}

تحت این شرایط، دترمینان ژاکوبین $J = \det(\partial \mathbf{x}/\partial \mathbf{X})$ نمی‌تواند صفر شود و در واقع فرض می‌کنیم:
\begin{equation}
	0 < \det\left(\frac{\partial \mathbf{x}}{\partial \mathbf{X}}\right) < \infty
\end{equation}

\begin{keypoint}
	نگاشت تغییرشکل $\tena{x} = \boldsymbol{\chi}(\tena{X}, t)$ باید پیوسته، معکوس‌پذیر، و مشتق‌پذیر باشد، با $J > 0$ تا اطمینان حاصل شود که ماده به خود نفوذ نمی‌کند.
\end{keypoint}

\section{توصیف‌های لاگرانژی و اویلری}

معادله‌ی حرکت $x_i = \chi_i(X_1, X_2, X_3, t)$ می‌تواند به‌عنوان رابطه‌ای بین مختصات مرجع یا مادی $X_i$ و مختصات کنونی یا فضایی $x_i$ در نظر گرفته شود. بنابراین، اگر مختصات مادی یک ذره‌ی مشخص را بدانیم، این رابطه اجازه می‌دهد موقعیت ذره در پیکربندی کنونی را تعیین کنیم. به همین ترتیب، معادله‌ی معکوس $X_i = \chi_i^{-1}(x_1, x_2, x_3, t)$ رابطه‌ی مخالف را ارائه می‌دهد.

همه‌ی متغیرهای میدان فضایی در مکانیک پیوسته (چگالی، دما، جابه‌جایی، کرنش، تنش و غیره) می‌توانند بر حسب مختصات مادی $X_i$ یا مختصات فضایی $x_i$ توصیف شوند.

\subsection{توصیف لاگرانژی (مادی)}

ذرات مادی منفرد را در طول حرکتشان دنبال می‌کند:
\begin{itemize}
	\item متغیرهای مستقل: $(\tena{X}, t)$
	\item تمرکز: چه بر ذره‌ی مشخصی می‌گذرد
	\item مناسب برای: کاربردهای مکانیک جامدات
\end{itemize}

با دنبال کردن ذرات، می‌توانیم کمیت‌های تانسوری را به‌عنوان توابعی که با مختصات مادی $(X_1, X_2, X_3)$ شناسایی می‌شوند، بیان کنیم. چنین توصیفی به‌عنوان توصیف لاگرانژی، مادی یا مرجع شناخته می‌شود.

\subsection{توصیف اویلری (فضایی)}

نقاط ثابت فضا را مشاهده می‌کند:
\begin{itemize}
	\item متغیرهای مستقل: $(\tena{x}, t)$
	\item تمرکز: چه در موقعیت ثابت می‌گذرد
	\item مناسب برای: کاربردهای مکانیک سیالات
\end{itemize}

با استفاده از طرح دیگر، می‌توانیم تغییرات را در موقعیت‌های ثابت مشاهده کنیم و بنابراین کمیت‌های تانسوری را به‌عنوان توابع مختصات موقعیت $(x_1, x_2, x_3)$ بیان کنیم. چنین توصیفی به‌عنوان توصیف اویلری یا فضایی شناخته می‌شود. توجه کنید که با گذشت زمان، ذرات مختلف موقعیت فضایی یکسانی را اشغال خواهند کرد، و بنابراین توصیف فضایی اطلاعات مشخصی در مورد خواص ذرات در طول حرکت ارائه نمی‌دهد.

\subsection{مشتق مادی}

نرخ تغییر زمانی یک کمیت تانسوری که ذره‌ی مادی را دنبال می‌کند، به‌عنوان مشتق زمانی مادی شناخته شده و معمولاً با $D/Dt$ نشان داده می‌شود. زمانی که توصیف مادی یک میدان تانسوری مشخص $T$ استفاده می‌شود، چنین مشتقی به روش مستقیم محاسبه می‌شود:
\begin{equation}
	\frac{DT}{Dt} = \frac{\partial T(X_1, X_2, X_3, t)}{\partial t}\bigg|_{X_i \text{ ثابت}}
\end{equation}

با این حال، زمانی که توصیف فضایی برای تانسور $T$ استفاده می‌شود، مشتق زمانی کمی پیچیده‌تر است زیرا مختصات فضایی خود اکنون توابعی از زمان هستند. این امر نیازمند استفاده از قانون زنجیره‌ای است:
\begin{equation}
	\frac{D\phi}{Dt} = \frac{\partial \phi}{\partial t} + \tena{v} \scp \nabla \phi
\end{equation}

که در آن $\tena{v}$ بردار سرعت است و $\nabla\phi$ گرادیان فضایی $\phi$ است.

\section{اندازه‌گیری تغییرشکل}

\subsection{گرادیان تغییرشکل}

تانسور گرادیان تغییرشکل $\tenb{F}$ کمیت کلیدی در سینماتیک محدود است که اطلاعات کاملی در مورد تغییرشکل محلی ارائه می‌دهد:
\begin{equation}
	\tenb{F} = \frac{\partial \tena{x}}{\partial \tena{X}} = \nabla_0 \tena{x}
\end{equation}

این تانسور نشان‌دهنده‌ی چگونگی نگاشت عناصر خط بی‌نهایت کوچک از پیکربندی مرجع به پیکربندی کنونی است. اگر $d\tena{X}$ عنصر خط بی‌نهایت کوچک در پیکربندی مرجع باشد، آنگاه عنصر خط متناظر در پیکربندی کنونی $d\tena{x} = \tenb{F} \cdot d\tena{X}$ خواهد بود.

دترمینان گرادیان تغییرشکل $J = \det(\tenb{F})$ نسبت تغییر حجم محلی را نمایندگی می‌کند. برای ماده‌ی غیرقابل تراکم، $J = 1$ و برای مواد قابل تراکم، $J > 0$ باید باشد تا از نفوذ ماده جلوگیری شود.

\subsection{اندازه‌های کرنش}

\subsubsection{تانسور کرنش گرین-لاگرانژ}

تانسور کرنش گرین-لاگرانژ اندازه‌ی لاگرانژی کرنش است که نسبت به پیکربندی مرجع تعریف می‌شود:
\begin{equation}
	\tenb{E} = \frac{1}{2}(\tenb{F}^\tran \scp \tenb{F} - \unitb) = \frac{1}{2}(\tenb{C} - \unitb)
\end{equation}

که $\tenb{C} = \tenb{F}^\tran \scp \tenb{F}$ تانسور تغییرشکل راست کوشی-گرین است.

این تانسور دارای خواص مهم زیر است:
\begin{itemize}
	\item تانسور متقارن است: $E_{ij} = E_{ji}$
	\item در صورت عدم تغییرشکل، $\tenb{E} = \mathbf{0}$
	\item برای چرخش صلب خالص، $\tenb{E} = \mathbf{0}$
	\item مؤلفه‌های قطری $E_{ii}$ نشان‌دهنده‌ی کرنش‌های نرمال هستند
	\item مؤلفه‌های غیرقطری $E_{ij}$ ($i \neq j$) نشان‌دهنده‌ی کرنش‌های برشی هستند
\end{itemize}

برای تغییرشکل‌های کوچک، این تانسور به تانسور کرنش خطی تبدیل می‌شود.

\subsubsection{تانسور کرنش المانسی-اویلری}

تانسور کرنش المانسی-اویلری اندازه‌ی اویلری کرنش است که نسبت به پیکربندی کنونی تعریف می‌شود:
\begin{equation}
	\tenb{e} = \frac{1}{2}(\unitb - \tenb{b}^{-1})
\end{equation}

که $\tenb{b} = \tenb{F} \scp \tenb{F}^\tran$ تانسور تغییرشکل چپ کوشی-گرین است.

تفاوت‌های اساسی با تانسور گرین-لاگرانژ:
\begin{itemize}
	\item در چارچوب فضایی تعریف می‌شود
	\item برای تحلیل‌های اویلری مناسب‌تر است
	\item برای سیالات و مواد با تغییرشکل‌های بزرگ به‌کار می‌رود
	\item رابطه‌ی تبدیل: $\tenb{e} = \tenb{F}^{-\tran} \cdot \tenb{E} \cdot \tenb{F}^{-1}$
\end{itemize}

هر دو تانسور در حد تغییرشکل‌های کوچک به تانسور کرنش خطی همگرا می‌شوند، اما برای تغییرشکل‌های بزرگ تفاوت‌های قابل توجهی دارند.

\subsubsection{تانسور کرنش بی‌نهایت کوچک}
برای تغییرشکل‌های کوچک:
\begin{equation}
	\boldsymbol{\varepsilon} = \frac{1}{2}(\nabla \tena{u} + (\nabla \tena{u})^\tran)
\end{equation}

که $\tena{u} = \tena{x} - \tena{X}$ بردار جابه‌جایی است.

\subsection{تجزیه‌ی قطبی}

قضیه‌ی تجزیه‌ی قطبی یکی از نتایج بنیادی در نظریه‌ی ماتریس است که در مکانیک پیوسته کاربرد مهمی دارد. طبق این قضیه، هر تانسور گرادیان تغییرشکل $\tenb{F}$ قابل تجزیه‌ی یکتا به صورت:
\begin{equation}
	\tenb{F} = \tenb{R} \scp \tenb{U} = \tenb{V} \scp \tenb{R}
\end{equation}

که در آن:
\begin{itemize}
	\item $\tenb{R}$: تانسور چرخش (متعامد با $\det(\tenb{R}) = +1$)
	\item $\tenb{U}$: تانسور کشش راست (متقارن مثبت معین)
	\item $\tenb{V}$: تانسور کشش چپ (متقارن مثبت معین)
\end{itemize}

\textbf{تفسیر فیزیکی}: تجزیه‌ی $\tenb{F} = \tenb{R} \scp \tenb{U}$ نشان می‌دهد که تغییرشکل می‌تواند به‌عنوان کشش خالص توسط $\tenb{U}$ در امتداد محورهای اصلی‌اش، و سپس چرخش صلب توسط $\tenb{R}$ تفسیر شود. تجزیه‌ی $\tenb{F} = \tenb{V} \scp \tenb{R}$ اول چرخش صلب و سپس کشش را انجام می‌دهد.

\textbf{روابط بین تانسورهای کشش}:
\begin{align}
	\tenb{U} & = \sqrt{\tenb{C}} = \sqrt{\tenb{F}^\tran \cdot \tenb{F}} \\
	\tenb{V} & = \sqrt{\tenb{b}} = \sqrt{\tenb{F} \cdot \tenb{F}^\tran} \\
	\tenb{V} & = \tenb{R} \cdot \tenb{U} \cdot \tenb{R}^\tran
\end{align}

مقادیر ویژه‌ی $\tenb{U}$ و $\tenb{V}$ یکسان بوده و کشش‌های اصلی $\lambda_i$ نامیده می‌شوند.

\section{نیرو و اندازه‌های تنش}

\subsection{اصل تنش کوشی}

اصل بنیادی تنش کوشی رابطه‌ی زیر را برقرار می‌کند:
\begin{equation}
	\tena{t}^{(\dir{n})} = \tenb{\sigma} \scp \dir{n}
\end{equation}

که در آن $\tena{t}^{(\dir{n})}$ بردار تنش روی سطح با نرمال $\dir{n}$ و $\tenb{\sigma}$ تانسور تنش کوشی است.

\begin{keypoint}
	اصل تنش کوشی بیان می‌کند که بردار تنش در هر نقطه بر روی سطحی با نرمال واحد $\dir{n}$ به‌صورت خطی به نرمال سطح وابسته است.
\end{keypoint}

\subsection{خواص تانسور تنش}

\subsubsection{تقارن تانسور تنش}
از تعادل تکانه‌ی زاویه‌ای:
\begin{equation}
	\sigma_{ij} = \sigma_{ji}
\end{equation}

\subsubsection{تنش‌های اصلی}
مقادیر ویژه تانسور تنش تنش‌های اصلی نامیده می‌شوند:
\begin{equation}
	\det(\tenb{\sigma} - \sigma \unitb) = 0
\end{equation}

\subsubsection{تجزیه‌ی کروی-انحرافی}
\begin{align}
	\tenb{\sigma} & = \sigma_m \unitb + \tenb{s}                                                         \\
	\sigma_m      & = \frac{1}{3} \text{tr}(\tenb{\sigma}) = \frac{1}{3}(\sigma_1 + \sigma_2 + \sigma_3) \\
	\tenb{s}      & = \tenb{\sigma} - \sigma_m \unitb
\end{align}

\section{اصول بقا و تعادل}

\subsection{قضیه‌ی انتقال رینولدز}

پایه‌ی ریاضی برای تبدیل بیانیه‌های بقای سراسری به محلی:
\begin{equation}
	\frac{D}{Dt} \int_{\mathcal{V}(t)} \phi \, d\mathcal{V} = \int_{\mathcal{V}(t)} \left( \frac{D\phi}{Dt} + \phi \nabla \scp \tena{v} \right) d\mathcal{V}
\end{equation}

\subsection{بقای جرم}

معادله‌ی پیوستگی:
\begin{equation}
	\frac{D\rho}{Dt} + \rho \nabla \scp \tena{v} = 0
\end{equation}

\subsection{تعادل تکانه‌ی خطی}

معادلات حرکت کوشی:
\begin{equation}
	\nabla \scp \tenb{\sigma} + \rho \tena{b} = \rho \frac{D\tena{v}}{Dt}
\end{equation}

که $\tena{b}$ نیروی بدنی در واحد جرم است.

\subsection{تعادل تکانه‌ی زاویه‌ای}

منجر به تقارن تانسور تنش می‌شود:
\begin{equation}
	\tenb{\sigma} = \tenb{\sigma}^\tran
\end{equation}

\subsection{بقای انرژی}

قانون اول ترمودینامیک:
\begin{equation}
	\rho \frac{De}{Dt} = \tenb{\sigma} : \tenb{D} - \nabla \scp \tena{q} + \rho r
\end{equation}

که در آن $e$ انرژی درونی ویژه، $\tena{q}$ بردار شار حرارت، و $r$ منبع حرارت است.

\section{نظریه‌های ساختاری خطی کلاسیک}

\subsection{اصول بنیادی}

\subsubsection{عینیت}
روابط ساختاری باید نسبت به تبدیل مختصات ناوردا باشند.

\subsubsection{تقارن مادی}
خواص مادی باید گروه تقارن مادی را منعکس کنند.

\subsubsection{سازگاری ترمودینامیکی}
روابط ساختاری باید نابرابری کلاؤزیوس-دوهم را رعایت کنند.

\subsection{الاستیسیته‌ی خطی}

\subsubsection{قانون تعمیم یافته‌ی هوک}
برای مواد الاستیک خطی:
\begin{equation}
	\tenb{\sigma} = \mathbb{C} : \boldsymbol{\varepsilon}
\end{equation}

که $\mathbb{C}$ تانسور مدول‌های الاستیک مرتبه‌ی چهارم است.

\subsubsection{مواد همسان}
برای مواد الاستیک همسان:
\begin{equation}
	\tenb{\sigma} = \lambda (\text{tr} \boldsymbol{\varepsilon}) \unitb + 2\mu \boldsymbol{\varepsilon}
\end{equation}

که $\lambda$ و $\mu$ ثابت‌های لامه هستند.

مدول‌های مهندسی:
\begin{align}
	E   & = \frac{\mu(3\lambda + 2\mu)}{\lambda + \mu} & \text{(مدول یانگ)}   \\
	\nu & = \frac{\lambda}{2(\lambda + \mu)}           & \text{(نسبت پواسون)} \\
	G   & = \mu                                        & \text{(مدول برشی)}   \\
	K   & = \lambda + \frac{2\mu}{3}                   & \text{(مدول حجمی)}
\end{align}

\subsection{رفتار ویسکوز خطی}

\subsubsection{سیالات نیوتونی}
\begin{equation}
	\tenb{\sigma} = -p \unitb + 2\mu \tenb{D} + \lambda_v (\nabla \scp \tena{v}) \unitb
\end{equation}

که $\tenb{D} = \frac{1}{2}(\nabla \tena{v} + (\nabla \tena{v})^\tran)$ تانسور نرخ تغییرشکل است.

\subsection{ویسکوالاستیسیته‌ی خطی}

\subsubsection{مدل‌های مکانیکی}
مدل ماکسول:
\begin{equation}
	\frac{d\sigma}{dt} + \frac{E}{\eta} \sigma = E \frac{d\varepsilon}{dt}
\end{equation}

مدل کلوین:
\begin{equation}
	\sigma + \frac{\eta}{E} \frac{d\sigma}{dt} = \eta \frac{d\varepsilon}{dt}
\end{equation}

\section{نظریه‌های میدان جفت شده}

\subsection{چارچوب ترمودینامیکی}

میدان‌های جفت شده نیازمند چارچوب ترمودینامیکی منسجمی هستند که اصول بنیادی حاکم بر اثرات جفت‌شدگی را تعریف کند.

\subsection{ترموالاستیسیته‌ی خطی}

\subsubsection{معادلات جفت شده}
معادله‌ی حرکت:
\begin{equation}
	\nabla \scp \tenb{\sigma} + \rho \tena{b} = \rho \ddot{\tena{u}}
\end{equation}

معادله‌ی انرژی:
\begin{equation}
	\rho c_\varepsilon \dot{T} = k \nabla^2 T + T_0 \alpha_{ij} \dot{\sigma}_{ij} + \rho r
\end{equation}

رابطه‌ی ساختاری:
\begin{equation}
	\sigma_{ij} = C_{ijkl} \varepsilon_{kl} - \alpha_{ij} (T - T_0)
\end{equation}

\subsection{پوروالاستیسیته}

\subsubsection{نظریه‌ی بیوت}
معادله‌ی تعادل:
\begin{equation}
	\nabla \scp \tenb{\sigma}' + \alpha \nabla p + \rho \tena{b} = \rho \ddot{\tena{u}}
\end{equation}

معادله‌ی انتشار:
\begin{equation}
	\nabla \scp \left( \frac{k}{\mu_f} \nabla p \right) = \frac{1}{M} \dot{p} + \alpha \nabla \scp \dot{\tena{u}}
\end{equation}

که در آن $\tenb{\sigma}' = \tenb{\sigma} + \alpha p \unitb$ تنش مؤثر است.

\section{رفتار مادی غیرخطی}

\subsection{چارچوب کلی}

رفتار غیرخطی نیازمند تعمیم اصول عینیت، تقارن مادی، و محدودیت‌های ترمودینامیکی است.

\subsection{فراالاستیسیته}

\subsubsection{تابع انرژی کرنش}
برای مواد فراالاستیک:
\begin{equation}
	\tenb{\sigma} = 2\rho_0 \frac{\partial W}{\partial \tenb{C}} \tenb{F}^\tran
\end{equation}

\subsubsection{مدل‌های متداول}
مدل نئو-هوکین:
\begin{equation}
	W = \frac{\mu}{2}(I_1 - 3) - \mu \ln J + \frac{\lambda}{2}(\ln J)^2
\end{equation}

مدل مونی-ریولین:
\begin{equation}
	W = C_{10}(I_1 - 3) + C_{01}(I_2 - 3) + \frac{\kappa}{2}(J - 1)^2
\end{equation}

\subsection{پلاستیسیته‌ی کلاسیک}

\subsubsection{معیار تسلیم}
معیار فون میزس:
\begin{equation}
	f = \sqrt{\frac{3}{2} \tenb{s} : \tenb{s}} - \sigma_y = 0
\end{equation}

\subsubsection{قانون جریان}
قانون جریان انطباقی:
\begin{equation}
	\dot{\boldsymbol{\varepsilon}}^p = \dot{\lambda} \frac{\partial f}{\partial \tenb{\sigma}}
\end{equation}

\section{ملاحظات ریزساختاری}

\subsection{عنصر حجمی نماینده}

نظریه‌ی همگن‌سازی و جداسازی مقیاس نیازمند تعریف عنصر حجمی نماینده (RVE) است که خواص مؤثر را نمایندگی کند.

\subsection{نظریه‌ی میکروپولار}

\subsubsection{درجات آزادی اضافی}
علاوه بر جابه‌جایی $\tena{u}$، میکروچرخش مستقل $\boldsymbol{\phi}$ در نظر گرفته می‌شود.

\subsubsection{معادلات تعادل}
تعادل نیرو:
\begin{equation}
	\nabla \scp \tenb{\sigma} + \rho \tena{b} = \rho \ddot{\tena{u}}
\end{equation}

تعادل گشتاور:
\begin{equation}
	\nabla \scp \boldsymbol{\mu} + \boldsymbol{\varepsilon} : \tenb{\sigma} + \rho \tena{l} = \rho \tenb{J} \cdot \ddot{\boldsymbol{\phi}}
\end{equation}

\subsection{نظریه‌های گرادیان کرنش}

برای در نظر گیری اثرات مقیاس طول مادی:
\begin{equation}
	\sigma_{ij} = C_{ijkl} \varepsilon_{kl} - l^2 C_{ijkl} \nabla^2 \varepsilon_{kl}
\end{equation}

که $l$ مقیاس طول مشخصه‌ی مادی است.

\section{مدل‌سازی چندمقیاسه}

\subsection{رویکردهای سلسله‌مراتبی}

اطلاعات از مقیاس کوچک‌تر به بزرگ‌تر منتقل می‌شود:
\begin{equation}
	\langle \sigma \rangle = \frac{1}{V} \int_V \sigma(\tena{x}) \, dV
\end{equation}

\subsection{رویکردهای همزمان}

مقیاس‌های مختلف به‌طور همزمان حل می‌شوند.

\section{پیاده‌سازی محاسباتی}

\subsection{روش اجزای محدود}

\subsubsection{فرمول‌بندی ضعیف}
برای مسئله‌ی الاستیسیته:
\begin{equation}
	\int_\Omega \tenb{\sigma} : \nabla^s \tena{v} \, d\Omega = \int_\Omega \rho \tena{b} \scp \tena{v} \, d\Omega + \int_{\Gamma_t} \tena{t} \scp \tena{v} \, d\Gamma
\end{equation}

\subsection{الگوریتم‌های غیرخطی}

\subsubsection{روش نیوتن-رافسون}
برای حل معادلات غیرخطی:
\begin{equation}
	\tenb{K}_t \Delta \tena{u} = \tena{R} - \tena{F}_{int}
\end{equation}

که $\tenb{K}_t$ ماتریس سختی مماسی است.

\section{جمع‌بندی و چشم‌انداز}

مکانیک پیوسته چارچوب جامع و قدرتمندی برای درک و پیش‌بینی رفتار مواد در شرایط مختلف فراهم می‌کند. از مفاهیم بنیادی فرضیه‌ی پیوستار تا نظریه‌های پیشرفته چندمقیاسه، این رشته همچنان در حال تکامل است تا نیازهای مهندسی مدرن را برآورده کند.

\subsection{توسعه‌های آینده}

\paragraph{مواد هوشمند و نانومواد}
\begin{itemize}
	\item توسعه‌ی نظریه‌های ساختاری برای مواد با اثرات اندازه
	\item مدل‌سازی رفتار چندفعالی در مواد هوشمند
	\item درنظرگیری اثرات سطحی در مقیاس‌های کوچک
\end{itemize}

\paragraph{محاسبات عملکرد بالا}
\begin{itemize}
	\item الگوریتم‌های موازی برای مسائل چندمقیاسه
	\item استفاده از هوش مصنوعی در پیش‌بینی رفتار مادی
	\item روش‌های کاهش مدل برای شبیه‌سازی‌های بلادرنگ
\end{itemize}

\paragraph{کاربردهای نوظهور}
\begin{itemize}
	\item بیومکانیک و مهندسی بافت
	\item مواد پایدار و سبز
	\item کاربردهای فضایی و محیط‌های شدید
\end{itemize}

مکانیک پیوسته با ترکیب دقت ریاضی، بینش فیزیکی، و قابلیت‌های محاسباتی، همچنان به‌عنوان پایه‌ی اساسی مهندسی مدرن عمل می‌کند و راه را برای نوآوری‌های آینده در علم و فناوری مواد هموار می‌سازد.