% ══════════════════════════════════════════════════════════════════════════════════════════════════
% Math commands
% ══════════════════════════════════════════════════════════════════════════════════════════════════

% Customized underline


\makeatletter

\newcommand{\thicku}[1]{%
\begingroup
\newdimen\@tempwidth%
\settowidth{\@tempwidth}{$#1$}%
\newdimen\@tempheight%
\settoheight{\@tempheight}{$#1$}%
\advance\@tempheight by -1ex%
\smash{\stackrel[\raisebox{\@tempheight}{\rule{\@tempwidth}{0.15ex}}]{}{\smash{#1}}}%
\endgroup
}

\newcommand{\thickuu}[1]{
\begingroup
\newdimen\@@tempwidth%
\settowidth{\@@tempwidth}{$#1$}%
\newdimen\@@tempheight%
\settoheight{\@@tempheight}{$#1$}%
\advance\@@tempheight by -1.4ex%
\stackrel[\smash{\raisebox{+\@@tempheight}{\rule{\@@tempwidth}{0.15ex}}}]{}{\smash{\thicku{#1}}}%
\endgroup
}
\makeatother



	% Rayleigh product
	\newcommand{\Rayleigh}{\mathbin{\ooalign{$\star$}}}


    % Derivative
    \def\dif{\text{d}}
    \newcommand{\dV}{\;\dif V}
    \newcommand{\dS}{\;\dif S}
    
% Transpose
\newcommand{\tran}{\text{T}}
\newcommand{\ttran}{\text{TT}}
\newcommand{\rtran}{\text{RT}}
\newcommand{\ltran}{\text{LT}}
\newcommand{\mtran}{\text{MT}}
\newcommand{\otran}{\text{OT}}



\def\sym{\text{sym}}
\def\skw{\text{skw}}
\def\dev{\text{dev}}
\def\vol{\text{vol}}
\def\half{\tfrac{1}{2}}
\def\third{\tfrac{1}{3}}
\def\base{\tena{e}}

% Group theory
\newcommand{\Group}{\mathscr{G}}
\newcommand{\Gop}{\star}
\newcommand{\Gorder}[1]{\text{ord}#1}

% Norms
\newcommand{\norm}[1]{\left\lvert{}#1\right\rvert}

% Geometries 
\newcommand{\Body}{\mathfrak{B}}
\newcommand{\Surface}{\mathfrak{S}}
\newcommand{\Volume}{\mathfrak{V}}
\newcommand{\Line}{\mathfrak{L}}
\newcommand{\Point}{\mathfrak{p}}
\newcommand{\Part}{\mathfrak{P}}
\newcommand{\Neigh}{\mathfrak{N}}
\newcommand{\Observer}{\mathfrak{O}}

\newcommand{\funcsc}{\mathcal{F}}
\newcommand{\func}{\mathbfcal{F}}
\newcommand{\Lop}{\mathcal{L}}
\newcommand{\Ltop}{\mathbfcal{L}}



\newcommand{\levi}{\tenc{\epsilon}}
\newcommand{\gibbs}[1]{#1_\cross}
\newcommand{\cross}{\times}

% 3D nabla operator
	\newcommand{\inv}{{-1}}
	%\newcommand{\grad}[1]{\text{grad}(#1)}

	\newcommand{\grad}[1]{\tena{\nabla} #1}
	\newcommand{\gradX}[1]{\tena{\nabla}_{\!X} #1}
	\newcommand{\gradx}[1]{\tena{\nabla}_{\!x} #1}
	
	\newcommand{\gradsym}[1]{\tena{\nabla}^\sym#1}
	\newcommand{\gradskw}[1]{\tena{\nabla}^\skw#1}

	\newcommand{\lap}[1]{\tena{\nabla}^2 #1}
	\newcommand{\diver}[1]{\tena{\nabla}\scp#1}
	\newcommand{\curl}[1]{\tena{\nabla}\cross #1}

% Surface nabla operator
	\newcommand{\sgrad}[1]   {\overline{\tena{\nabla}} #1}
	\newcommand{\sgradX}[1]   {\overline{\tena{\nabla}}_{\!X} #1}
	\newcommand{\sgradx}[1]   {\overline{\tena{\nabla}}_{\!x} #1}
	\newcommand{\sgradsym}[1]{\overline{\tena{\nabla}}^\sym #1}
	\newcommand{\sgradskw}[1]{\overline{\tena{\nabla}}^\skw #1}
	\newcommand{\sdiver}[1]  {\overline{\tena{\nabla}}\scp #1}
	\newcommand{\scurl}[1]   {\overline{\tena{\nabla}}\cross #1}

    % Trace operator
    \newcommand{\tr}[1]{\mathrm{tr}\, #1}

% sign
\newcommand{\sign}[1]{\text{sgn}#1}


    % Sets
    \newcommand{\Real}{\mathbb{R}}
    \newcommand{\Euclid}{\mathbb{E}}
    \newcommand{\Integer}{\mathbb{Z}}
    \newcommand{\Natural}{\mathbb{N}}
    
    % 2nd-O tensor sets (text)
    \newcommand{\linsettext}{\mathrm{Lin}}
    \newcommand{\linpsettext}{\mathrm{Lin+}}
    \newcommand{\symsettext}{\mathrm{Sym}}
    \newcommand{\skwsettext}{\mathrm{Skw}}
    \newcommand{\sympsettext}{\mathrm{Sym+}}
    \newcommand{\orthsettext}{\mathrm{Orth}}
    \newcommand{\orthpsettext}{\mathrm{Orth+}}
    \newcommand{\unisettext}{\mathrm{Uni}}
    
    % 2nd-O tensor sets (math)    
    \newcommand{\linset}{\class{2}{3}}
    \newcommand{\linpset}{\null^+\class{2}{3}}
    \newcommand{\symset}{\null^\text{s}\class{2}{3}}
    \newcommand{\skwset}{\null^\text{a}\class{2}{3}}
    \newcommand{\sympset}{\null^{\text{s}+}\class{2}{3}}
    \newcommand{\orthset}{\null^\text{o}\class{2}{3}}
    \newcommand{\orthpset}{\null^{\text{o}+}\class{2}{3}}
    \newcommand{\uniset}{\null^{\pm1}\class{2}{3}}



% Equality symbols
%\newcommand{\equidef}{\stackrel{\mathrm{def}}{=}}
\newcommand{\equidef}{\coloneq}
\newcommand{\equimust}{\stackrel{\mathrm{!}}{=}}
    
    % Tensor order O(T)
    \newcommand{\order}[1]{\mathscr{O}(#1)}

    % Tensor class O_n^d
    \newcommand{\class}[2]{\mathscr{O}_{#1}^{#2}}


% Roman numbers
\newcommand{\rom}[1]{\mathrm{\expandafter{\romannumeral #1\relax}}}
\newcommand{\Rom}[1]{\mathrm{\uppercase\expandafter{\romannumeral #1\relax}}}

% Tensors
\newcommand{\tena}[1]{\symbf{#1}}                       % tensor with various orders 1
\newcommand{\tenb}[1]{\symbf{#1}}                       % tensor with various orders 2
\newcommand{\tenc}[1]{\symbf{\uline{#1}}}           % tensor with various orders 3
\newcommand{\tend}[1]{\symbf{\wideutilde{#1}}}         % tensor with various orders 4
\newcommand{\ten}[1]{\symbfcal{#1}}                     % n-th order general tensor
\newcommand{\tenn}[2]{\:\!^#2\symbfcal{#1}}               % n-th order general tensor w/t explicit order

\newcommand{\nulla}{\tena{\symbfup{o}}}
\newcommand{\nullb}{\tenb{\symbfup{O}}}
\newcommand{\nullc}{\tenc{\symbfup{O}}}
\newcommand{\nulld}{\tend{\symbfup{O}}}
\newcommand{\nulln}{\ten{\symbfcal{O}}}

\newcommand{\unita}{\base}
\newcommand{\unitb}{\tenb{I}}
%\newcommand{\unitc}{\tenc{I}}
\newcommand{\unitd}{\tend{\;\!I}}
\newcommand{\unitdT}{\tend{\;\!\overbar{I}}}

% Matrix notation
\newcommand{\mat}[1]{\uppercase{\mathbfsf#1}}          % matrix notation matrix
\newcommand{\col}[1]{\{\mat{#1}\}}                     % general column matrix
\newcommand{\row}[1]{\lfloor\mat{#1}\rfloor}           % general row matrix




% maybe 1st-o tensor only bold, 2nd-o double underline, 4th-o utilde

% Marcus' tensor notation
%\newcommand{\tena}[1]{\lowercase{\mathbfit{#1}}}       % 1st order tensor
%\newcommand{\tenb}[1]{{{\symbf{#1}}}}                  % 2nd order tensor
%\newcommand{\tend}[1]{\uppercase{\mathbfcal{#1}}}      % 4th order tensor

\newcommand{\op}{\mdlgwhtsquare}                       % Empty square for operator (unicode)





% old dot product
%\makeatletter%
%\newcommand*\scp{\mathpalette\bigcdot@{.65}}%
%\newcommand*\bigcdot@[2]{\mathbin{\vcenter{\hbox{\scalebox{#2}{$\m@th#1\bullet$}}}}}%
%\newcommand*\dvbigcdot@[2]{%
%\mathbin{\vcenter{\hbox{\raisebox{0.3em}{\scalebox{#2}{$\m@th#1\bullet$}}\kern-0.225em\scalebox{#2}{$\m@th#1\bullet$}} }}}%
%%\newcommand*\dscp{\mathpalette\dvbigcdot@{.65}}        % double vertical scalar product
%\newcommand*\dhscp{\scp\kern-0.5mm\scp\,}               % double horizontal scalar product


% dot product
\makeatletter
\newcommand{\s@cp}{\raisebox{0.5ex}{\scalebox{1.4}[1.4]{$.$}}}

\DeclareMathOperator{\scp}{%
  \mathbin{\ooalign{\s@cp}}}%

\DeclareMathOperator{\dscp}{%
  \mathbin{\ooalign{\raisebox{-0.5ex}{\s@cp}\cr\hidewidth \s@cp\hidewidth}}}%
  
\DeclareMathOperator{\tscp}{%
  \mathbin{\ooalign{\raisebox{-0.5ex}{\s@cp}\cr\hidewidth \raisebox{0.5ex}{\s@cp}\cr\hidewidth \s@cp\hidewidth}}}%
  
\DeclareMathOperator{\hdscp}{%
  \mathbin{\ooalign{\s@cp\kern-0.1ex\s@cp}}}%


\DeclareMathOperator{\thscp}{%
  \mathbin{\ooalign{\s@cp\kern-0.1ex\s@cp\kern-0.1ex\s@cp}}}%




\DeclareMathOperator{\crosscp}{%
  \mathbin{\ooalign{\cross\s@cp}}}%


\DeclareMathOperator{\scpcross}{%
  \mathbin{\ooalign{\s@cp\cross}}}%


    % Conjugation product
    \newcommand{\conj}{\mathbin{\boxtimes}}

	% height of the bar used for overbar and underbar of dyadic operation	
	\newdimen\barh
	\barh=0.7pt\relax
	
	% extension of dyadic product
	\newcommand{\dyadu}{\setbox0\hbox{$\otimes$}\stackrel[\raisebox{5pt}{\vrule height \the\barh width 0.5\wd0\relax}]{}{\dyad}}

	\newcommand{\dyado}{\setbox0\hbox{$\otimes$}\stackrel{\raisebox{-2.5pt}{\vrule height \the\barh width 0.5\wd0\relax}}{\dyad}}
	
	\newcommand{\dyadsym}{\setbox0\hbox{$\otimes$}\stackrel[\raisebox{5pt}{\vrule height \the\barh width 0.5\wd0\relax}]{\raisebox{-2.5pt}{\vrule height \the\barh width 0.5\wd0\relax}}{\dyad}}
    


    %   I could not make the following line work:
    %       \DeclareMathSymbol{\conj}{\mathbin}{symbols}{"22A0}
    %   Possibly mathspec.sty would have the answer 
    %       \XeTeXDeclareMathSymbol{}{\conj}{\eu@GreekUppercase@symfont}{`}[]    
    % It is possible to define a math font group instead of the default symbols, see fontmath.ltx:
    %
    %    \DeclareSymbolFont{Symbols}{OMS}{cmsy}{m}{n}
    %    \SetSymbolFont{Symbols}{bold}{OMS}{cmsy}{b}{n}
    % 

    % Dyadic product
    \DeclareMathOperator{\dyad}{\mathbin{\ooalign{\otimes}}}


    % Through-thickness integration
    \newcommand{\thickint}[1]{\int\limits_{-\frac{h}{2}}^{+\frac{h}{2}}#1\;\dif X_3}
    
    % Put inside left- and right-angles
    \newcommand{\inangle}[1]{\,\langle #1 \rangle\,}

% eigen symbol
\newcommand{\eigen}{\ast}

\makeatother
