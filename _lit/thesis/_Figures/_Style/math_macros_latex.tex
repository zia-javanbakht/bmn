% Math commands




%	Every math character is given an identifying code number between 0 and 4095, obtained by adding 256 times the family number to the position number. This is easily expressed in hexadecimal notation, using one hexadecimal digit for the family and two for the character; for example, \hex{24A} stands for character \hex{4A} in family 2. Each character is also assigned to one of eight classes, numbered 0 to 7, as follows:
%	
%	Class 0: Ordinary (eg., /)
%	Class 1: Large operator (eg., \sum)
%	Class 2: Binary operation (eg., +)
%	Class 3: Relation (eg., =)
%	 4: Opening (eg., ()
%	Class 5: Closing (eg., ))
%	Class 6: Punctuation (eg., ,)
%	Class 7: Variable family (eg., x)
%	
%	Classes 0 to 6 tell what "part of speech" the character belongs to, in math-printing language; class 7 is a special case [...]. The class number is multiplied by 4096 and added to the character number, and this is the same as making it the leading digit of a four-digit hexadecimal number.
%	...
%	TeX associates classes with subformulas as well as with individual characters. Thus, for example, you can treat a complex construction as if it were a binary operation or a relation, etc., if you want to. The commands \mathord, \mathop, \mathbin, \mathrel, \mathopen,  \mathclose, and \mathpunct are used for this purpose; each of them is followed either by a single character or by a subformula in braces. For example, \mathopen\mathchar"1234 is equivalent to \mathchar"4234, because \mathopen forces class 4 (opening). In the formula $G\mathbin:H$, the colon is treated as a binary operation.
%	...
%	There's also an eighth classification, \mathinner, which is not normally used for individual symbols; fractions and \left...\right constructions are treated as "inner" subformulas, which means that they will be surrounded by additional space in certain circumstances. All other subformulas are generally treated as ordinary symbols, whether they are formed by \overline or \hbox or \vcenter or by simply being enclosed in braces. Thus, \mathord isn't really a necessary part of the TeX language; instead of typing  $1\mathord,234$ you can get the same effect from $1{,}234$.


% XeLaTeX version

% Customized underline

	% Average
	\newcommand{\averot}[1]{\langle\!\langle #1\rangle\!\rangle}
	\newcommand{\avevol}[1]{\langle #1\rangle}

	% Rayleigh product
%	\newcommand{\Rayleigh}{\mathbin{\ooalign{$\star$}}}
	\newcommand{\Rayleigh}{\scpn}
	
	% Symmetry operator
	\newcommand{\sym}{\text{sym}}
%	\DeclareMathOperator{\sym}{sym}
	
% Derivative
\def\dif{\text{d}}
\newcommand{\dV}{\;\dif V}
\newcommand{\dS}{\;\dif S}


% Equality symbols
%\newcommand{\equidef}{\stackrel{\mathrm{def}}{=}}
\newcommand{\equidef}{\mathrel{:=}}
\newcommand{\equimust}{\stackrel{\mathrm{!}}{=}}

    
% Transpose
\newcommand{\tran}{\text{T}}
\newcommand{\ttran}{\text{TT}}
\newcommand{\rtran}{\text{RT}}
\newcommand{\ltran}{\text{LT}}
\newcommand{\mtran}{\text{MT}}
\newcommand{\otran}{\text{OT}}


% Tensors
\newcommand{\fett}[1]{\mbox{\boldmath$#1$}}
\newcommand{\tena}[1]{\mathord{\lowercase{\fett{#1}}}}                  % tensor with various orders 1
\newcommand{\tenb}[1]{\mathord{\uppercase{\fett{#1}}}}                     % tensor with various orders 2
\newcommand{\tenc}[1]{\mathord{\fett{\uppercase{\underline{#1}}}}}           % tensor with various orders 3
\newcommand{\tend}[1]{\mathord{\fett{\mathcal{#1}}}}         % tensor with various orders 4
\newcommand{\ten}[1]{\mathord{\fett{\mathcal{#1}}}}                     % n-th order general tensor
\newcommand{\tenn}[2]{\mathord{\null^{#2}\kern-0.01em\fett{\mathcal{#1}}}}             % n-th order general tensor w/t explicit order

\newcommand{\nulla}{\tena{o}}
\newcommand{\nullb}{\tenb{O}}
\newcommand{\nullc}{\tenc{O}}
\newcommand{\nulld}{\tend{O}}
\newcommand{\nulln}{\ten{O}}

\newcommand{\unita}{\base}
\newcommand{\unitb}{\tenb{I}}
%\newcommand{\unitc}{\tenc{I}}
\newcommand{\unitd}{\tend{\;\!I}}
\newcommand{\unitdT}{\tend{\;\!\overline{I}}}

% Dyadic product



\newcommand{\dyad}{\mathbin{\ooalign{$\otimes$}}}

\newcommand{\scpn}{\mathbin{\ooalign{$\odot$}}}
\newcommand{\contract}{\mathbin{\ooalign{$\circ$}}}

	% tensor power
	\newcommand{\tenpow}[1]{{\tiny\dyad#1}}


% dot product
\makeatletter
\newcommand{\s@cp}{\raisebox{0.5ex}{\scalebox{1.4}[1.4]{$.$}}}

\DeclareMathOperator{\scp}{%
  \mathbin{\ooalign{\s@cp}}}%

\DeclareMathOperator{\dscp}{%
  \mathbin{\ooalign{\raisebox{-0.5ex}{\s@cp}\cr\hidewidth \s@cp\hidewidth}}}%
  
\DeclareMathOperator{\tscp}{%
  \mathbin{\ooalign{\raisebox{-0.5ex}{\s@cp}\cr\hidewidth \raisebox{0.5ex}{\s@cp}\cr\hidewidth \s@cp\hidewidth}}}%
  
\DeclareMathOperator{\dhscp}{%
  \mathbin{\ooalign{\s@cp\kern-0.1ex\s@cp}}}%


\DeclareMathOperator{\thscp}{%
  \mathbin{\ooalign{\s@cp\kern-0.1ex\s@cp\kern-0.1ex\s@cp}}}%




\DeclareMathOperator{\crosscp}{%
  \mathbin{\ooalign{\cross\s@cp}}}%


\DeclareMathOperator{\scpcross}{%
  \mathbin{\ooalign{\s@cp\cross}}}%


    % Conjugation product
%    \newcommand{\conj}{\mathbin{\boxtimes}}

	% extension to overbar and underbar dyadic product
    \newcommand{\dyado}{\mathbin{\ooalign{$\overline{\otimes}$}}}
    \newcommand{\dyadu}{\mathbin{\ooalign{$\underline{\otimes}$}}}
    \newcommand{\dyadd}{\mathbin{\ooalign{$\overline{\underline{\otimes}}$}}}
    


    %   I could not make the following line work:
    %       \DeclareMathSymbol{\conj}{\mathbin}{symbols}{"22A0}
    %   Possibly mathspec.sty would have the answer 
    %       \XeTeXDeclareMathSymbol{}{\conj}{\eu@GreekUppercase@symfont}{`}[]    
    % It is possible to define a math font group instead of the default symbols, see fontmath.ltx:
    %
    %    \DeclareSymbolFont{Symbols}{OMS}{cmsy}{m}{n}
    %    \SetSymbolFont{Symbols}{bold}{OMS}{cmsy}{b}{n}
    % 




    % Through-thickness integration
    \newcommand{\thickint}[1]{\int\limits_{-\frac{h}{2}}^{+\frac{h}{2}}#1\;\dif X_3}
    
    % Put inside left- and right-angles
    \newcommand{\inangle}[1]{\,\langle #1 \rangle\,}





% END

\makeatletter

\newcommand{\thicku}[1]{%
\begingroup
\newdimen\@tempwidth%
\settowidth{\@tempwidth}{$#1$}%
\newdimen\@tempheight%
\settoheight{\@tempheight}{$#1$}%
\advance\@tempheight by -1ex%
\smash{\stackrel[\raisebox{\@tempheight}{\rule{\@tempwidth}{0.15ex}}]{}{\smash{#1}}}%
\endgroup
}

\newcommand{\thickuu}[1]{
\begingroup
\newdimen\@@tempwidth%
\settowidth{\@@tempwidth}{$#1$}%
\newdimen\@@tempheight%
\settoheight{\@@tempheight}{$#1$}%
\advance\@@tempheight by -1.4ex%
\stackrel[\smash{\raisebox{+\@@tempheight}{\rule{\@@tempwidth}{0.15ex}}}]{}{\smash{\thicku{#1}}}%
\endgroup
}
\makeatother








\def\skw{\text{skw}}
\def\dev{\text{dev}}
\def\vol{\text{vol}}
\def\half{\tfrac{1}{2}}
\def\third{\tfrac{1}{3}}
\def\base{\tena{e}}

% Group theory
\newcommand{\Group}{\mathscr{G}}
\newcommand{\Gop}{\star}
\newcommand{\Gorder}[1]{\text{ord}#1}

% Norms
\newcommand{\norm}[1]{\left\lvert{}#1\right\rvert}

% Geometries 
\newcommand{\Body}{\mathfrak{B}}
\newcommand{\Surface}{\mathfrak{S}}
\newcommand{\Volume}{\mathfrak{V}}
\newcommand{\Line}{\mathfrak{L}}
\newcommand{\Point}{\mathfrak{p}}
\newcommand{\Part}{\mathfrak{P}}
\newcommand{\Neigh}{\mathfrak{N}}
\newcommand{\Observer}{\mathfrak{O}}

\newcommand{\funcsc}{\mathcal{F}}
\newcommand{\func}{\mathbfcal{F}}
\newcommand{\Lop}{\mathcal{L}}
\newcommand{\Ltop}{\mathbfcal{L}}



\newcommand{\levi}{\tenc{\epsilon}}
\newcommand{\gibbs}[1]{#1_\cross}
\newcommand{\cross}{\times}

% 3D nabla operator
	\newcommand{\inv}{{-1}}
	%\newcommand{\grad}[1]{\text{grad}(#1)}

	\newcommand{\grad}[1]{\tena{\nabla} #1}
	\newcommand{\gradX}[1]{\tena{\nabla}_{\!X} #1}
	\newcommand{\gradx}[1]{\tena{\nabla}_{\!x} #1}
	
	\newcommand{\gradsym}[1]{\tena{\nabla}^\sym#1}
	\newcommand{\gradskw}[1]{\tena{\nabla}^\skw#1}

	\newcommand{\lap}[1]{\tena{\nabla}^2 #1}
	\newcommand{\diver}[1]{\tena{\nabla}\scp#1}
	\newcommand{\curl}[1]{\tena{\nabla}\cross #1}

% Surface nabla operator
	\newcommand{\sgrad}[1]   {\overline{\tena{\nabla}} #1}
	\newcommand{\sgradX}[1]   {\overline{\tena{\nabla}}_{\!X} #1}
	\newcommand{\sgradx}[1]   {\overline{\tena{\nabla}}_{\!x} #1}
	\newcommand{\sgradsym}[1]{\overline{\tena{\nabla}}^\sym #1}
	\newcommand{\sgradskw}[1]{\overline{\tena{\nabla}}^\skw #1}
	\newcommand{\sdiver}[1]  {\overline{\tena{\nabla}}\scp #1}
	\newcommand{\scurl}[1]   {\overline{\tena{\nabla}}\cross #1}

    % Trace operator
    \newcommand{\tr}[1]{\mathrm{tr}\, #1}

% sign
\newcommand{\sign}[1]{\text{sgn}#1}


    % Sets
    \newcommand{\Real}{\mathbb{R}}
    \newcommand{\Euclid}{\mathbb{E}}
    \newcommand{\Integer}{\mathbb{Z}}
    \newcommand{\Natural}{\mathbb{N}}
    
    % 2nd-O tensor sets
    \newcommand{\linset}{\mathrm{Lin}}
    \newcommand{\linpset}{\mathrm{Lin+}}
    \newcommand{\symset}{\mathrm{Sym}}
    \newcommand{\skwset}{\mathrm{Skw}}
    \newcommand{\sympset}{\mathrm{Sym+}}
    \newcommand{\orthset}{\mathrm{Orth}}
    \newcommand{\orthpset}{\mathrm{Orth+}}



    
    % Tensor order O(T)
    \newcommand{\order}[1]{\mathscr{O}(#1)}

    % Tensor class O_n^d
    \newcommand{\class}[2]{\mathscr{O}_{#1}^{#2}}


% Roman numbers
\newcommand{\rom}[1]{\mathrm{\expandafter{\romannumeral #1\relax}}}
\newcommand{\Rom}[1]{\mathrm{\uppercase\expandafter{\romannumeral #1\relax}}}


% Matrix notation
	\newcommand{\mat}[1]{\uppercase{\fett{\sf#1}}}         % matrix notation matrix
\newcommand{\col}[1]{\{\mat{#1}\}}                     % general column matrix
\newcommand{\row}[1]{\lfloor\mat{#1}\rfloor}           % general row matrix

  



% maybe 1st-o tensor only bold, 2nd-o double underline, 4th-o utilde

% Marcus' tensor notation
%\newcommand{\tena}[1]{\lowercase{\mathbfit{#1}}}       % 1st order tensor
%\newcommand{\tenb}[1]{{{\symbf{#1}}}}                  % 2nd order tensor
%\newcommand{\tend}[1]{\uppercase{\mathbfcal{#1}}}      % 4th order tensor

\newcommand{\op}{\square}                       % Empty square for operator (unicode)





% old dot product
%\makeatletter%
%\newcommand*\scp{\mathpalette\bigcdot@{.65}}%
%\newcommand*\bigcdot@[2]{\mathbin{\vcenter{\hbox{\scalebox{#2}{$\m@th#1\bullet$}}}}}%
%\newcommand*\dvbigcdot@[2]{%
%\mathbin{\vcenter{\hbox{\raisebox{0.3em}{\scalebox{#2}{$\m@th#1\bullet$}}\kern-0.225em\scalebox{#2}{$\m@th#1\bullet$}} }}}%
%%\newcommand*\dscp{\mathpalette\dvbigcdot@{.65}}        % double vertical scalar product
%\newcommand*\dhscp{\scp\kern-0.5mm\scp\,}               % double horizontal scalar product


% eigen symbol
\newcommand{\eigen}{\ast}

\makeatother
