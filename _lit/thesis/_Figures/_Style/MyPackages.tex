% ──────────────────────────────────────────────────────────────────────────────────────────────────
% My Packages 
% ──────────────────────────────────────────────────────────────────────────────────────────────────


\newif\ifdouble
\doublefalse




% package for speeding up the compilation (I tried it did not work with the xr-hyper)
%\usepackage{subfiles}
%\usepackage{xr-hyper}

% ══════════════════════════════════════════════════════════════════════════════════════════════════
% Misc. packages
% ══════════════════════════════════════════════════════════════════════════════════════════════════
\usepackage{etex}                                % Solves the problem of too many packages
\usepackage{comment}                             % Defines the comment environment
\usepackage{makeidx}                             % allows index generation
%\usepackage{footnote}
\usepackage[bottom]{footmisc}                    % places footnotes at page bottom
\usepackage{datetime2}                           % date and time
\usepackage{enumitem}                            % flexibile options for enumerate env.


\iffig

\else
	 % geometry of the page
%    \usepackage[xetex, margin=3cm, a4paper]{geometry}                           
\fi


% ══════════════════════════════════════════════════════════════════════════════════════════════════
% Input and encoding
% ══════════════════════════════════════════════════════════════════════════════════════════════════
\usepackage[english]{babel}                      % Typographical and hyphenation
%\usepackage[latin1]{inputenc}                      % Accepts ISO Latin-1 encoding (only PDFTeX)


%\usepackage[T1]{fontenc}                         % The output can be easily copy pasted
\usepackage{lscape}                              % landscape env.
%\usepackage{layout}                              % Allows changing the layout of the page

\usepackage{scrhack}                              % load before setspace for corrections



% This package replaces lstlistings for xelatex engine. 
% minted uses Python pygments library, so you need to:
%    - Install Python, choose any version you like: x64 or x86, it better to choose 2.7.5 version.
%    - Add Python to PATH: C:\Python27;C:\Python27\Scripts
%    - pip install pygments to install Pygments 
\usepackage[cache=false]{minted}


%
%
%\ifdouble
%	\usepackage[doublespacing]{setspace}                             % different line spacings
%	% double spacing
%	\RedeclareSectionCommand[
%	  beforeskip=-1sp,
%	  afterskip=2\baselineskip]{chapter}
%	\RedeclareSectionCommand[
%	  beforeskip=-\baselineskip,
%	  afterskip=0.25\baselineskip]{section}
%	\RedeclareSectionCommand[
%	  beforeskip=-.75\baselineskip,
%	  afterskip=0.25\baselineskip]{subsection}
%	\RedeclareSectionCommand[
%	  beforeskip=-.5\baselineskip,
%	  afterskip=0.25\baselineskip]{subsubsection}
%	\RedeclareSectionCommand[
%	  beforeskip=.5\baselineskip,
%	  afterskip=-1em]{paragraph}
%	\RedeclareSectionCommand[
%	  beforeskip=-.5\baselineskip,
%	  afterskip=-1em]{subparagraph}
%	
%	  \setlist[enumerate,itemize]{
%	                              topsep    = 0.25\parskip,
%	                              itemsep   = 0.25\parskip,
%	                              partopsep = 0.25\parskip,
%	                              parsep    = 0.25\parskip
%	                             }
%\else
%	% Setting the spacing after/before sections
%	% normal spacing values
%	\RedeclareSectionCommand[
%	  beforeskip=-1sp,
%	  afterskip=2\baselineskip]{chapter}
%	\RedeclareSectionCommand[
%	  beforeskip=-\baselineskip,
%	  afterskip=0.5\baselineskip]{section}
%	\RedeclareSectionCommand[
%	  beforeskip=-.75\baselineskip,
%	  afterskip=0.5\baselineskip]{subsection}
%	\RedeclareSectionCommand[
%	  beforeskip=-.5\baselineskip,
%	  afterskip=0.5\baselineskip]{subsubsection}
%	\RedeclareSectionCommand[
%	  beforeskip=.5\baselineskip,
%	  afterskip=-1em]{paragraph}
%	\RedeclareSectionCommand[
%	  beforeskip=-.5\baselineskip,
%	  afterskip=-1em]{subparagraph}
%	
%	  \setlist[enumerate,itemize]{
%	                              topsep    = 0.5\parskip,
%	                              itemsep   = 0.25\parskip,
%	                              partopsep = 0.5\parskip,
%	                              parsep    = 0.5\parskip
%	                             }
%\fi


% Centering the subsection title
%\makeatletter
%\renewcommand{\sectionlinesformat}[4]{%
%  \ifstr{#1}{section}{\centering}{}% center section titles
%  \@hangfrom{\hskip #2#3}{#4.}%
%}
%\makeatother


% Adding a dot 
%\renewcommand{\sectioncatchphraseformat}[4]{%
%  \hskip #2#3#4%
%  \ifstr{#1}{subsubsection}{.}{}% dot after subsection titles
%}
%\renewcommand{\sectioncatchphraseformat}[4]{%
%  \hskip #2#3#4%
%  \ifstr{#1}{paragraph}{.}{}% dot after subsection titles
%}
%  
%
%




% ══════════════════════════════════════════════════════════════════════════════════════════════════
% Fonts
% ══════════════════════════════════════════════════════════════════════════════════════════════════
%\usepackage{helvet}                              % helvetica font loaded
%\usepackage{courier}                             % courier font loaded
%\usepackage{mathptmx}                            % times new roman font loaded for math (for tensors remove this)
%\usepackage{parskip}
\usepackage{type1cm}                              % removes the restriction on font scaling
\usepackage{ragged2e}                             % justifying the text
\usepackage{fontspec}

% Add a font that is not under the system fonts
%\fontspec [ Path            = ./_Fonts/,
%%UprightFont = *-regular,
%% BoldFont = *-bold
% Extension       = .otf,
% ]
%{FontAwesome}

                
    
\usepackage{fontawesome}                          % web fonts

%\setmonofont[                                     % Monospace font
%             Scale=0.8,
%             Path = ./_Fonts/
%            ]{DejaVuSansMono.ttf}                 
%\setmainfont[                                     % Text font
%             Ligatures={Common}, %Numbers={OldStyle}, Variant=01,
%             Path = ./_Fonts/,
%             BoldFont=LinLibertine_RB.otf,
%             ItalicFont=LinLibertine_RI.otf,
%             BoldItalicFont=LinLibertine_RBI.otf
%            ]{LinLibertine_R.otf}

%\setmonofont[Scale=0.9]{SFMono-Bold.otf}
    
    % Monospaced font
    \setmonofont{LibertinusMono-Regular.otf}[
                 Scale=0.8,
                 Path            = ./_Fonts/,
                 Extension       = .otf,
                ]                
    
    % Main text font
    \setmainfont{LibertinusSerif}[%
                 Ligatures       = {Common}, %Numbers={OldStyle}, Variant=01,
                 Path            = ./_Fonts/,
                 Extension       = .otf,
                 UprightFont     = *-Regular,
                 BoldFont        = *-Bold,
                 ItalicFont      = *-Italic,
                 BoldItalicFont  = *-BoldItalic,
                ]
    % Sanserif font
    \setsansfont{LibertinusSans}[
                 Path            = ./_Fonts/,
                 Extension       = .otf,
                 UprightFont     = *-Regular,
                 BoldFont        = *-Bold,
                 ItalicFont      = *-Italic,
                ]
    




%\setmainfont{Times New Roman}

% ══════════════════════════════════════════════════════════════════════════════════════════════════
% Math
% ══════════════════════════════════════════════════════════════════════════════════════════════════

%\usepackage{amsmath}                             % main math environment (always before unicode-math)
\usepackage{mathtools}                            % extension to amsmath which loads it automatically
\usepackage{amsthm}                               % math theorems
\usepackage{mathrsfs}
%\usepackage{amssymb}                             % symbols like \square  (always before unicode-math)
\usepackage{xfrac}                                % slanted fractions 
\usepackage{accents}
%\usepackage[bb=mma,cal=cm,scr=esstix]{mathalfa}
\usepackage{unicode-math}                        % Math for XeLaTeX
\usepackage{stackrel}                             % Enhancements to the \stackrel command


\unimathsetup{math-style=ISO, bold-style=ISO}




    % Math font setting
    \setmathfont{LibertinusMath-Regular}[             % General math font
                 Path            = ./_Fonts/,
                 Extension       = .otf
                 ]
    
    % Requires lm-math package
    \setmathfont{Latin Modern Math}[                  % Calligraphy font for linear operators
                 range           = {cal, bfcal},
                 ]
    
    % Requires tex-gyre-math package
    \setmathfont{TeX Gyre Termes Math}[               % For proper blackboard letters
                 range=bb
                 ]
    
    \setmathfont{XITSMath-Regular}[               % For fraktur
                 Path            = ./_Fonts/,
                 Extension       = .otf,
                 range           = frak,
                 ]
    
    \setmathfont{XITSMath-Bold}[               % For fraktur
                 Path            = ./_Fonts/,
                 Extension       = .otf,
                 range           = bffrak,
                 ]
    
    
    \setmathfont{LibertinusMath-Regular}[             % For upright Greek letters using \mup...
                 Path            = ./_Fonts/,
                 Extension       = .otf,
                 range           = up/{greek,Greek},          
                 math-style      = literal
                 ]
    
    




% Working math fonts
%\setmathfont{Latin Modern Math}
%\setmathfont{XITSMath-bold.otf}[Path = ./_Fonts/,range=bfcal,StylisticSet=1]


%% do not remember what are these two lines for
%\setmathfont{XITSMath-bold.otf}[range=bfgre,StylisticSet=1]
%\setmathfont[range=bb]{Linux Libertine O}







\usepackage{textcase}                            % creeating upper and lower case letters
\usepackage{mathtools}
\usepackage{pifont}

\usepackage{cancel}

%\usepackage{calrsfs} 
%\usepackage{mathrsfs}                            % Styles for MATH mode
%\usepackage{dutchcal}

%\usepackage{epstopdf}





	% Defining new float environments
	\usepackage{newfloat}




% ══════════════════════════════════════════════════════════════════════════════════════════════════
% Setting the chapter style for Komma
% ══════════════════════════════════════════════════════════════════════════════════════════════════
\ifdouble

% Change the chapter style
\renewcommand{\chapterlinesformat}[3]{%
\vspace{0.1\textheight}% space between the chapter number and the line
\raisebox{-0.5\baselineskip}{\parbox[c][15pt]{\linewidth}{#2}}
\rule{\textwidth}{4pt}
\parbox[c][45pt]{\dimexpr\linewidth-2\fboxrule-2\fboxsep}{\mdseries\scshape\fontsize{20}{14}\selectfont#3}\\
\rule{\textwidth}{4pt}\vspace{0.1\textheight}
}

% Change the chapter font
\addtokomafont{chapter}{\mdseries\fontsize{14}{10}\selectfont\scshape}

\else
%
%% Change the chapter style
%\renewcommand{\chapterlinesformat}[3]{%
%\vspace{0.05\textheight}
%\raisebox{-0.5\baselineskip}{\parbox[c][15pt]{\linewidth}{#2}}
%\rule{\textwidth}{4pt}
%\parbox[c][45pt]{\dimexpr\linewidth-2\fboxrule-2\fboxsep}{\mdseries\scshape\fontsize{26}{20}\selectfont#3}\\
%\rule{\textwidth}{4pt}\vspace{0.1\textheight}
%}
%
%% Change the chapter font
%\addtokomafont{chapter}{\mdseries\fontsize{18}{12}\selectfont\scshape}
%\fi
%
%
%
%
%% Change the number formating
%\renewcommand{\chapterformat}{\chaptername\ \thechapter}
%


% Change the appendix style
\makeatletter
\g@addto@macro\appendix{%
  \renewcommand*{\chapterformat}{%
    {\chapapp\nobreakspace\thechapter\enskip}%
  }
  \renewcommand*{\chaptermarkformat}{%
    {\chapapp\nobreakspace\thechapter\enskip}%
  }
  \let\oldaddcontentsline\addcontentsline
  \newcommand\hackedaddcontentsline[3]{\oldaddcontentsline{#1}{#2}{\chapapp\nobreakspace#3}}
%  \let\oldchapter\chapter
%  \renewcommand*\chapter[1]{%
%    \let\addcontentsline\hackedaddcontentsline%
%    \oldchapter{#1}%
%    \let\addcontentsline\oldaddcontentsline%
%  }
}
\makeatother

% ══════════════════════════════════════════════════════════════════════════════════════════════════
% Formatting of the sections
% ══════════════════════════════════════════════════════════════════════════════════════════════════
%
%\addtokomafont{section}{\mdseries\normalfont\Large}
%\addtokomafont{subsection}{\large\mdseries\normalfont}
%\addtokomafont{subsubsection}{\mdseries\itshape\normalsize}
%
%% Change ToC to serif font
%\setkomafont{disposition}{\normalfont\bfseries}     

% ══════════════════════════════════════════════════════════════════════════════════════════════════
% Symbols
% ══════════════════════════════════════════════════════════════════════════════════════════════════
%\usepackage{upgreek}                             % upright greek letter
\usepackage[normalem]{ulem}                      % Various underlinings
\usepackage{marvosym}                            % adds some symbols, e.g., CheckedBox
\usepackage{manfnt}                              % symbols from Knuth's TeX manual
\usepackage{wasysym}                             % adds some symbols, e.g., CheckedBox
\usepackage{tikzsymbols}                         % tree symbol

% ══════════════════════════════════════════════════════════════════════════════════════════════════
% Tables
% ══════════════════════════════════════════════════════════════════════════════════════════════════
\usepackage{array}                               % extended capabilities to tabular env.
\usepackage{booktabs}                            % toprule, bottomrule, midrule etc.
\usepackage{colortbl}                            % color rows and columns for tables
\usepackage{multicol, multirow}                  % used for the two-column index
\usepackage{longtable}
\usepackage{siunitx}							 % SI units + s-type tabular data
\usepackage[skip=0em,
            belowskip=1em,
            aboveskip=1em,
            labelfont=bf                         % use bold caption labels
            ]{caption}                           % placement of caption of the floats is fine-tuned

\setlength{\aboverulesep}{0pt}                   % these two lines are to remove the gap 
\setlength{\belowrulesep}{0pt}                   % in the color tables when toprule and bottomrule is used
\setlength{\extrarowheight}{.75ex}

% ══════════════════════════════════════════════════════════════════════════════════════════════════
% Figures
% ══════════════════════════════════════════════════════════════════════════════════════════════════
%\usepackage{subfigure}                          % using multiple subfigures in figure (obsolete)
%\usepackage{subcaption}                         % using multiple subfigures in figure (obsolete)
\usepackage{subfig}                              % better alternative !

\usepackage{hhline}
%\setlength{\extrarowheight}{.5ex}
% Graphics
\usepackage[export]{adjustbox}                           % frame for pictures
\usepackage{graphicx}                            % standard LaTeX graphics tool
\usepackage{xcolor}                              % used to introduce different colors



\usepackage{afterpage}                           % flush floats \afterpage{\clearpage}
%\usepackage{media9}                              % embeding .h264 format MP4 videos
%\usepackage{bohr}                                % draws elements
%\usepackage{chemformula}                         % for chemical formulas
%\usepackage{chemfig}                             % to draw chemicals
\usepackage[many]{tcolorbox}                          % for colored and framed text boxes
\tcbuselibrary{skins,xparse}
%\usepackage[color=commentcolor,author={Zia}]{pdfcomment}

\usepackage{mdframed}                            % gray box package and the theorem environment



%\lstset{basicstyle=\footnotesize\ttfamily,backgroundcolor=\color{black!10}}


% Programming
\usepackage{xpatch}                              % patching the biblatex package
\usepackage{ifpdf}                               % checks if PDFLaTeX is used


%\addtobeamertemplate{headline}{\hypersetup{linkcolor=blue}}{}
%\addtobeamertemplate{footline}{\hypersetup{linkcolor=.}}{}





	% hyperref must be loaded after all packages
    \usepackage[colorlinks = true,  
                linkcolor  = green!50!black, 
                urlcolor   = blue!75!green!75,
                citecolor  = blue!50!green,
                bookmarks]{hyperref}
	\hypersetup{pdfinfo	= 
					{	Author      	= {Zia Javanbakht},
                 		Subject     	= {Computational Modelling of Natural Fibre Composites, Phd Thesis},
						Institution 	= {Griffith University}
					}
				}
				
	% Customising section commands in a KOMA-script files (\usepackage{sectsty} for others)
    \usepackage[automark, headsepline]{scrlayer-scrpage}       

    \setlist[description]{                           % Setting for symbols and abbreviations
                          leftmargin    = 2.5cm,
                          labelindent   = 0.5cm,
                          labelwidth    = 2cm,
                          itemindent    = 0.5mm,
                          labelsep      = 0mm,
                          topsep        = 0em,
                          itemsep       = 0mm,
                          partopsep     = 1em,
                          parsep        = 0em,
                          listparindent = 0mm,
                         }   






