%%%%%%%%%%%%%%%%%%%
%% Common Macros %%
%%%%%%%%%%%%%%%%%%%

% Common settings for all lectures in this course
\def\mycampus{Gold Coast Campus}
\def\myuniversity{Griffith University}
\def\myschool{Griffith School of Engineering and Built Environment}
\def\mycoursename{Engineering Materials}
\def\mycoursecode{1017ENG}
\def\mycourse{\mycoursecode~\mycoursename}
\def\myinstitute{\myuniversity\par\myschool\par\mycampus}

% These two lines work for both beamer and non-beamer
\date{\mydate}
\title{\mytitle}


% Add blank page
\newcommand\blankpage{%
    \afterpage{\null
    \thispagestyle{empty}%
    \addtocounter{page}{-1}%
    \newpage}\clearpage}

% ══════════════════════════════════════════════════════════════════════════════════════════════════
% Abbreviations in smallcaps
% ══════════════════════════════════════════════════════════════════════════════════════════════════

% Marc in smallcaps
\def\m{\fontshape\scdefault\selectfont Marc\normalfont}
% Mentat in smallcaps
\def\M{\fontshape\scdefault\selectfont Mentat\normalfont}
% Marc/Mentat in smallcaps
\def\mm{\fontshape\scdefault\selectfont Marc/Mentat\normalfont}
% Fortran Code
\def\fc#1{\textsf{\small{#1}}}
% Fortran Code scriptsize
\def\fcscript#1{\textsf{\scriptsize{#1}}}
% KeyWords
\def\kw#1{\fc{\MakeTextUppercase{#1}}}
% KeyWords scriptsize
\def\kwscript#1{\fcscript{\MakeTextUppercase{#1}}}
% New Paragraph without indents
\def\np{\par\noindent}
% My Quotation 
\newcommand{\myquote}[2]{\begin{center}\begin{minipage}{0.6\textwidth}\emph{{#1}}\vspace{-0.3cm}\flushright{{\scriptsize%
\if\relax\detokenize{#2}\relax
Anonymous
\else
#2
\fi
}}\\\end{minipage}\end{center}}

% Fortran Code
  \def\fc#1{\textsf{#1}}
% Fortran Code scriptsize
\def\fcscript#1{\textsf{\scriptsize{#1}}}
% Fortran
\def\f{\fontshape\scdefault\selectfont Fortran\normalfont}
% Marc in smallcaps
\def\m{\fontshape\scdefault\selectfont Marc\normalfont}
% Mentat in smallcaps
\def\M{\fontshape\scdefault\selectfont Mentat\normalfont}
% Marc/Mentat in smallcaps
\def\mm{\fontshape\scdefault\selectfont Marc/Mentat\normalfont}
% Intel
\def\intel{\fontshape\scdefault\selectfont Intel{$^{\scaleobj{0.7}{\textregistered}}$}\normalfont} 
% Syntax


\newcommand{\code}[1]{\texttt{#1}}


% ══════════════════════════════════════════════════════════════════════════════════════════════════
% Symbols
% ══════════════════════════════════════════════════════════════════════════════════════════════════

\newenvironment{syntax}{\par\vspace{0.2cm}\sffamily\small}{\vspace{0.2cm}\normalfont\newline\noindent}%
% x mark
\newcommand{\xmark}{\ding{55}}
% Check mark
\newcommand{\cmark}{\ding{51}}
% My checkedBox
\def\myCheckedBox{\hbox{\raise.2em\hbox{\cmark}\kern-0.9em\hbox{$\square$}}}
% My CrossedBox
\def\myCrossedBox{\hbox{\hbox{\resizebox{1.2ex}{!}{\xmark}}\kern-0.7em\raise-.05em\hbox{$\square$}}}
% The tree
\def\thetree{\BasicTree[3.]{blue!80!red!40!white}{blue!80!red!30!white!30!black}{blue!70!red!60!white}{leaf}}
% The arrow
\def\thearrow{\resizebox{2.5em}{!}{\ding{224}}}
% The hand
\def\thehand{\resizebox{2.5em}{!}{\ding{45}}}

% ══════════════════════════════════════════════════════════════════════════════════════════════════
% Environments
% ══════════════════════════════════════════════════════════════════════════════════════════════════

% Summary box using mdframed
\newenvironment{summary}{%
\begin{mdframed}[middlelinewidth=2pt, backgroundcolor=gray!50, linecolor=white]%
\textbf{Summary}\quad
}{%
\end{mdframed}%
}



\makeatletter
	% For Koma-script, automatically add a full stop after the paragraph title
	



%
%% Defintition box using mdframed
%\newtheoremstyle{mystyle2}{0}{}{}{}{\bfseries}{:}{ }{\thmname{#1} \thmnumber{#2} --- \textbf{\thmnote{#3}}}
%\theoremstyle{mystyle2}
%\newmdtheoremenv[
%hidealllines        = true,
%leftline            = true,
%linewidth           = 4pt,
%linecolor           = gray!40,
%innerrightmargin    = 0pt,
%frametitleaboveskip = 0mm,
%frametitlebelowskip = 0mm,
%skipabove           = \baselineskip,
%skipbelow           = \baselineskip,
%splittopskip        = 0mm,
%innerbottommargin   = 0pt,
%innertopmargin      = 0.75\baselineskip,
%]{definition}{Definition}[chapter]
%
%\renewcommand{\thedefinition}{\arabic{chapter}.\arabic{definition}}
%
%% Theorem box using mdframed
%\newtheoremstyle{mystyle1}{0}{}{}{}{\bfseries}{:}{ }{\thmname{#1} \thmnumber{#2} --- \textbf{\thmnote{#3}}}
%\theoremstyle{mystyle1}
%\newmdtheoremenv[hidealllines=true,
%hidealllines        = true,
%leftline            = true,
%linewidth           = 4pt,
%linecolor           = gray!40,
%innerrightmargin    = 0pt,
%frametitleaboveskip = 0mm,
%frametitlebelowskip = 0mm,
%skipabove           = 2\parskip,
%skipbelow           = -2\parskip,
%splittopskip        = 2\parskip,
%innerbottommargin   = 0pt
%]{theorem}{Theorem}[chapter]

%\renewcommand{\thetheorem}{\arabic{chapter}.\arabic{theorem}}
\makeatother





%%%%%%%%%%%%%%%%%%%%%%%%%%%%%%%%%%%%%%%%%%%%%
% Overlay commands
%%%%%%%%%%%%%%%%%%%%%%%%%%%%%%%%%%%%%%%%%%%%%
% command for coordinates in a graph
\def\sample#1#2#3#4#5[#6]{%
\coordinate[overlay,remember picture] (#3) at (#1,#2);
\draw[thick,fill=gray!30] ($(#3.center)+(#1-#3/2,#2-#4/2)$) rectangle ($(#3.center)+(#1+#3/2,#2+#4/2)$);
  \foreach \x in {0,...,10}
  \path[draw] ($(#1-#3/2+0.2cm+#5*\x/10,#2-0.3*#4)+(#3.center)$) --  ($(#1-#3/2+0.2cm+#5*\x/10,#2+0.3*#4)+(#3.center)$) node[] {};
}%






















% Table color
\colorlet{tableheadercolor}{gray!50}







% Box color definitions
\colorlet{commentcolor}{red!50}

\definecolor{complexboxcolor}{RGB}{244, 217, 83}%{yellow!80!red!75!white}
\colorlet{readboxcolor}{blue!80!red!30!white}
\colorlet{writeboxcolor}{green!65!red!60!white}
\definecolor{doboxcolor}{RGB}{114, 166, 234}%{green!90!yellow!40!red!60!white}
\definecolor{thinkboxcolor}{RGB}{239, 141, 161}

\colorlet{blockboxcolor}{green!40!blue}
\definecolor{exampleboxcolor}{RGB}{38, 115, 77}

\definecolor{enumerateboxcolor}{RGB}{31, 66, 173}

\colorlet{checkedboxcolor}{black!80!white}
\definecolor{itemizeboxcolor}{RGB}{77, 51, 153}
\definecolor{noteboxcolor}{RGB}{173, 31, 31}
\definecolor{quoteboxcolor}{RGB}{230, 195, 0}

\definecolor{instructboxcolor}{RGB}{237, 156, 235}



\def\allboxrule{0.12em}

% Box styles: read/write/do/complex
\tcbset{
  myboxstyle/.style={
  width=\textwidth,
  top=0.7em, left skip=-0.5cm, right skip = -0.25cm,
  bottom=0.7em,
  left=0.7em,
  right=0.7em,
  boxsep=0.3em, 
  sharp corners=west, 
  before=,%\centering,
  breakable,
  enhanced,
  arc=1mm,
  boxrule =\allboxrule,
  leftrule=3.5em
  %
},
  allboxstyle/.style={
  enhanced,
  breakable,
  before=\smallskip,
  left=0.5em,
  right=0.5em,
  top=0.5em, 
  bottom=0.5em,
  boxsep=0mm,
  attach boxed title to top left={xshift=\tcboxedtitleheight/2,yshift=-\tcboxedtitleheight/4},    %xshift=2em,yshift=-0.4em},
  coltitle=white,
  toptitle    = 0mm,
  bottomtitle = 0mm
  titlerule = 0mm,
  arc =0.2em, 
  boxrule =\allboxrule,
  %outer arc = 0mm,
  left skip =0mm,
  right skip= 0mm
  },
  blockboxstyle/.style={
  allboxstyle,
  colframe=blockboxcolor,
  colback=blockboxcolor!10, 
  boxed title style={left=0em,top=0mm,bottom=0mm,right=0em,sharp corners=south, text height=0.8em, boxrule=0mm,colback=blockboxcolor!90}
  },
  exampleboxstyle/.style={
  allboxstyle,
  colframe=exampleboxcolor,
  colback=exampleboxcolor!10,
  boxed title style={left=0em,top=0mm,bottom=0mm,right=0em,sharp corners=south,text height=0.8em, boxrule=0mm,colback=exampleboxcolor!90}
  },
  checkedboxstyle/.style={
  allboxstyle,
  colframe=checkedboxcolor,
  colback=checkedboxcolor!10,
  boxed title style={left=0em,top=0mm,bottom=0mm,right=0em,sharp corners=south,text height=0.8em, boxrule=0mm,colback=checkedboxcolor!90}
  },
    itemizeboxstyle/.style={
    allboxstyle,
    colframe=itemizeboxcolor,
    colback=itemizeboxcolor!10,
    boxed title style={left=0em,top=0mm,bottom=0mm,right=0em,sharp corners=south,text height=0.8em, boxrule=0mm,colback=itemizeboxcolor!90}
    },
    enumerateboxstyle/.style={
    allboxstyle,
    colframe=enumerateboxcolor,
    colback=enumerateboxcolor!10,
    boxed title style={left=0em,top=0mm,bottom=0mm,right=0em,sharp corners=south,text height=0.8em, boxrule=0mm,colback=enumerateboxcolor!90}
    },
    noteboxstyle/.style={
    allboxstyle,
    colframe=noteboxcolor,
    colback=noteboxcolor!10,
    boxed title style={left=0em,top=0mm,bottom=0mm,right=0em,sharp corners=south,text height=0.8em, boxrule=0mm,colback=noteboxcolor!90}
    },
    quoteboxstyle/.style={
    allboxstyle,
    outer arc = 0mm,
    arc = 1mm,
    left=0em,
    right=0em,
    top=1.5em, 
    bottom=0.5em,
    colframe=black,%quoteboxcolor,
    colback = quoteboxcolor!10
%    interior style={
%    top color   = quoteboxcolor!10,
%    bottom color=  quoteboxcolor!60},
%    interior hidden,
%    frame style image=gold.jpg,
   % colback=quoteboxcolor!60,
   % drop fuzzy shadow = black,
   % interior style={fill overzoom image=gold},
    }
}




% blockbox environment
\newenvironment{blockbox}[2][\unskip]{%
\if\relax\detokenize{#1}\relax
\begin{tcolorbox}[blockboxstyle]%
\else%
\begin{tcolorbox}[blockboxstyle,title={#1}
,overlay unbroken and first={
      \path
        let
        \p1=(title.north east),
        \p2=(frame.north east)
        in
        node[anchor=west,color=blockboxcolor,text width=\x2-\x1,yshift=0.1em] 
        at (title.east) {\small\ #2};
  }]%
\fi
}{%
\end{tcolorbox}%
}

% examplebox environment
\newenvironment{examplebox}[2][\unskip]{%
\if\relax\detokenize{#1}\relax
\begin{tcolorbox}[exampleboxstyle]%
\else%
\begin{tcolorbox}[exampleboxstyle,title={#1}
,overlay unbroken and first={
      \path
        let
        \p1=(title.north east),
        \p2=(frame.north east)
        in
        node[anchor=west,color=exampleboxcolor,text width=\x2-\x1,yshift=0.1em] 
        at (title.east) {\small\ #2};
  }]%
\fi
}{%
\end{tcolorbox}%
}


% checkedbox environment
\newenvironment{checkedbox}[2][\unskip]{%
\if\relax\detokenize{#1}\relax
\begin{tcolorbox}[checkedboxstyle]%
\else%
\begin{tcolorbox}[checkedboxstyle, title={#1}
,overlay unbroken and first={
      \path
        let
        \p1=(title.north east),
        \p2=(frame.north east)
        in
        node[anchor=west,color=checkedboxcolor,text width=\x2-\x1,yshift=0.1em] 
        at (title.east) {\small\ #2};
  }]%
\fi
\begin{itemize}[label=\myCheckedBox,leftmargin=*,labelindent=0.5em]%
}{%
\end{itemize}%
\end{tcolorbox}%
}

% crossedox environment
\newenvironment{crossedbox}[2][\unskip]{%
\if\relax\detokenize{#1}\relax
\begin{tcolorbox}[checkedboxstyle]%
\else%
\begin{tcolorbox}[checkedboxstyle, title={#1}
,overlay unbroken and first={
      \path
        let
        \p1=(title.north east),
        \p2=(frame.north east)
        in
        node[anchor=west,color=checkedboxcolor,text width=\x2-\x1,yshift=0.1em] 
        at (title.east) {\small\ #2};
  }]%
\fi
\begin{itemize}[label=\myCrossedBox,leftmargin=*,labelindent=0.5em]%
}{%
\end{itemize}%
\end{tcolorbox}%
}








% itemizedbox environment
\newenvironment{itemizebox}[2][\unskip]{%
\if\relax\detokenize{#1}\relax%
\begin{tcolorbox}[itemizeboxstyle,notitle]%
\else
\begin{tcolorbox}[itemizeboxstyle,title={#1}%
,overlay unbroken and first={%
      \path
        let
        \p1=(title.north east),
        \p2=(frame.north east)
        in
        node (t) [anchor=west,color=itemizeboxcolor,text width=\x2-\x1,yshift=0.1em] 
        at (title.east) {\small\ #2};
  }]%
\fi
\begin{itemize}[leftmargin=*,labelindent=0.5em]%
}{%
\end{itemize}%
\end{tcolorbox}%
}



% enumerate environment
\newenvironment{enumeratebox}[2][\unskip]{%
\if\relax\detokenize{#1}\relax
\begin{tcolorbox}[enumerateboxstyle]%
\else%
\begin{tcolorbox}[enumerateboxstyle,title={#1}%
,overlay unbroken and first={%
      \path
        let
        \p1=(title.north east),
        \p2=(frame.north east)
        in
        node [anchor=west,color=enumerateboxcolor,text width=\x2-\x1,yshift=0.1em] 
        at (title.east) {\small\ #2};
  }]%
\fi
\begin{enumerate}[leftmargin=*,labelindent=0.5em]%
}{%
\end{enumerate}%
\end{tcolorbox}%
}



% notebox environment
\newenvironment{notebox}[2][\unskip]{%
\if\relax\detokenize{#1}\relax%
\begin{tcolorbox}[noteboxstyle]%
\else%
\begin{tcolorbox}[noteboxstyle,title={#1}
,overlay unbroken and first={
      \path
        let
        \p1=(title.north east),
        \p2=(frame.north east)
        in
        node[anchor=west,color=noteboxcolor,text width=\x2-\x1,yshift=0.1em] 
        at (title.east) {\small\ #2};
  }]%
\fi
}{%
\end{tcolorbox}%
}











% The complexbox
\newtcolorbox{tcomplexbox}{myboxstyle,colframe=complexboxcolor,colback=white!90!complexboxcolor,every float=\centering,
overlay={ \node[anchor=west,outer sep=0.4em, xshift=0.4em] at (frame.west) {\textdbend}; }}
\newcommand{\complexbox}[1]{%
\begin{tcomplexbox}
\fontfamily{ptm}\selectfont%
#1
\end{tcomplexbox}
}
% The dobox
\newtcolorbox{tdobox}{myboxstyle,colframe=doboxcolor,colback=white!90!doboxcolor,
overlay={ \node[text centered, anchor=west,outer sep=0em, xshift=0.3em, yshift=-0.15em] at (frame.west) {\thearrow}; }}
\def\dobox#1{
\begin{tdobox}
\fontfamily{ptm}\selectfont%
#1
\end{tdobox}
}
% The writebox
\newtcolorbox{twritebox}{myboxstyle,
colback=white!90!writeboxcolor,colframe=writeboxcolor,
overlay={ \node[anchor=west,outer sep=0em,xshift=0.2em] at (frame.west) {\centering\thehand}; }}
\def\writebox#1{
\begin{twritebox}
\fontfamily{ptm}\selectfont%
#1
\end{twritebox}
}
% The readbox
\newtcolorbox{treadbox}{myboxstyle,
colback=white!90!readboxcolor,colframe=readboxcolor,
overlay={
\path[] ($(frame.west) +(3.5em,0em)$) -- node[midway]  {\thetree} (frame.west) ; }}

% \node[draw,anchor=west,outer sep=0em,text centered,align=flush center, xshift=0.05em] at (frame.west) {\thetree}; }}



%overlay={ \node[draw,anchor=west,outer sep=0em,text centered,align=flush center, xshift=0.05em] at (frame.west) {\thetree}; }}

\def\readbox#1{%
\begin{treadbox}%
\fontfamily{ptm}\selectfont%
#1
\end{treadbox}
}


% The thinkbox
\newtcolorbox{tthinkbox}{myboxstyle,
colback=white!90!thinkboxcolor,colframe=thinkboxcolor,
overlay={ \node[anchor=west,outer sep=0em,xshift=0.3em] at (frame.west) {\huge\faGraduationCap}; }}

\def\thinkbox#1{%
\begin{tthinkbox}%
\fontfamily{ptm}\selectfont%
#1
\end{tthinkbox}
}


% The instructbox
\newtcolorbox{tinstructbox}{myboxstyle,
colback=white!90!instructboxcolor,colframe=instructboxcolor,
overlay={ \node[anchor=west,outer sep=0em,xshift=0.4em] at (frame.west) {\huge\faUsers}; }}

\def\instructbox#1{%
\ifx\instructmode\undefined
%
\else
\begin{tinstructbox}%
\fontfamily{ptm}\selectfont%
#1
\end{tinstructbox}
\fi
}






% The example style
%\tcbset{myexamplestyle/.style={
%  breakable,
%  enhanced,
%  colframe=myexamplecolor,
%  colback=myexamplecolor!10,
%  arc=0mm, outer arc=0mm,
%  attach boxed title to top left={xshift=0em,yshift=-0.13em},
%  boxed title style={
%    colback=myexamplecolor, 
%    outer arc=0mm,
%    arc=0mm,
%    }
%  }
%}
% The example style
%\newtcolorbox[auto counter,number within=section]{examplebox}[1][]{
%  myexamplestyle,
%  title=Example~\thetcbcounter,
%  overlay unbroken and first={
%      \path
%        let
%        \p1=(title.north east),
%        \p2=(frame.north east)
%        in 
%        node[anchor=west,color=exampleboxcolor,text width=\x2-\x1] 
%        at (title.south) {#1};
%  }
%}
% checkedlist environment
%\newenvironment{mycheckedlist}[1][\unskip]{%
%\begin{tcolorbox}[title=#1]%
%\begin{itemize}[label=\myCheckedBox]%
%\justifying%
%}{%
%\end{itemize}%
%\end{tcolorbox}%
%}

% Quotation macro
%\def\placequote#1[#2]{%
%\begin{variableblock}{}{fg=orange!40!yellow, bg=black!70!white}{}%
%\fontfamily{ptm}\selectfont\large%
%\begin{quote}
%``#1''\\%
%\if\relax\detokenize{#2}\relax
%\flushright\footnotesize{Anonymous}\\
%\else
%\flushright\footnotesize{#2}\\
%\fi
%\end{quote}%
%\end{variableblock}}



\def\placequote#1[#2]{
\begin{tcolorbox}[quoteboxstyle]%
\begin{quotation}
\large\fontfamily{ptm}\selectfont%
\noindent``#1''\\%
\if\relax\detokenize{#2}\relax
\flushright\footnotesize{Anonymous}
\else
\flushright\footnotesize{#2}
\fi
\end{quotation}%
\end{tcolorbox}
}

%
\tikzset{%
  highlight/.style={rectangle,rounded corners,fill=red!15,draw,
    fill opacity=0.5,thick,inner sep=0pt}
}
\newcommand{\tikzmark}[2]{\tikz[overlay,remember picture,
  baseline=(#1.base)] \node (#1) {#2};}
%
\newcommand{\Highlight}[1][submatrix]{%
    \tikz[overlay,remember picture]{
    \node[highlight,fit=(left.north west) (right.south east)] (#1) {};}
}

\newcommand{\tikzpin}[1]{\tikz[overlay,remember picture] \node (#1) {};}






% Signature place-holder
\makeatletter
\newcommand{\my@sign}[2]{\begin{tabular}{@{}p{0.35\textwidth}@{}p{0.35\textwidth}@{}p{0.3\textwidth}}
\hrulefill &\hfill Date:\ &\hrulefill#2\hrulefill\\[-0.5cm]
#1&\\
\end{tabular}}

\newcommand{\supersign}[1]{\my@sign{#1}{\relax}}

\newcommand{\Zia}[1]{\my@sign{Zia Javanbakht}{\relax}}
\newcommand{\Wayne}{\supersign{Associate Professor Wayne Hall}}
\makeatother 



	
% ══════════════════════════════════════════════════════════════════════════════════════════════════
% NEW ENVIRONMENTS
% ══════════════════════════════════════════════════════════════════════════════════════════════════

	% Box environment for algorithms
	\DeclareFloatingEnvironment[fileext=frm,placement={!ht},name=Algorithm]{algorithm}
	
	% set up the respective caption placement
	\captionsetup[Algorithm]{
							skip		= 0em,
							belowskip	= 1em,
							aboveskip	= 0em,
							labelfont	= bf 
							}
	% define the style for the respective mdframed
	\mdfdefinestyle{algorithm}{
							innerleftmargin	= 0.01\textwidth,
							linewidth		= 1pt,
							skipabove		= 0mm,
							leftline		= false,
							rightline		= false}

	% define the style of enumerate
	\SetEnumitemKey{algorithm}{leftmargin=0.03\textwidth}

%\newenvironment{algorithm}[1]
%    {\begin{myalgo}\begin{mdframed}[roundcorner=10pt,backgroundcolor=blue!10]#1}
%    {\end{mdframed}\end{myalgo}\ignorespacesafterend}

	% Definition of Concept check environment
%	\newcounter{ex}[chapter]
%	\newenvironment{Example}{%
%	\refstepcounter{ex}%
%	\begin{svgraybox}%
%	\textbf{Example~\theex}%
%	\par\medskip\noindent\rmfamily\ignorespaces}{%
%	\end{svgraybox}\ignorespacesafterend}

