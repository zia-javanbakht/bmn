% !TEX TS-program = xelatex
% !TeX spellcheck = en_GB
% !TeX root = ../My_thesis.tex

\addchap{Abstract}
	In the current study, the major aim was to develop computationally-effective numerical procedures with the capability of modelling the thermal and mechanical response of fibre-reinforced composites. The attempt was made with a focus on the applicability of the procedures to a variety of synthetic and natural fibres with various geometrical characteristics so that a material-independent framework is developed. The work is divided into three main stages: developing methodologies, investigating the performance of the available analytical tools, and extending the material models by numerical techniques. The finite element computational framework was used and a library of general-purpose subroutines was developed that encompassed the methodologies. Then, finite element prototypes were challenged through a variety of thermal and mechanical analyses to set up the interaction between micro- and macro-scales. Representative volume elements (RVEs) were created for a variety of short/long and aligned/randomly-oriented fibres in thermal analyses. Fibre orientation tensors were used to carry out clustering and spectral analyses in order to characterise the morphology of the heterogeneities. According to the results of sensitivity analyses, although spectral analysis seems to be less sensitive to local variation of fibre direction, it correlates better with the effective properties. Furthermore, the element elimination technique was used to indirectly model the progression of damage. This technique was used in both discrete fibre modelling and homogenised elements. It was shown that by following the latter case, mesh-independent results could be obtained. Moreover, natural fibre-reinforced composites were modelled using discrete fibre elements and a new strength-updating scheme was proposed and implemented. The numerical results showed the detrimental effect of considering length-dependent strength properties in computational modelling. Namely, the strength of fibres must be updated for the remaining portions to obtain results that are closer to the experimental one. At the final stage of the study, the localised mesoscopic data was collected through the introduced concept of auxiliary maps. Auxiliary maps of volume fraction and fibre orientation data were obtained and it was shown that their resolution should be linked to the mesh density of the model. A semi-analytical model was created to demonstrate the performance of the purposed method. The proposed models were used in single-scale elastic analyses but are able to be extended to coupled multi-scale analyses. 
	
	
%	
%
%\begin{enumerate}[label=Stage~\Roman*.]
%	\item 
%	\begin{itemize}
%		\item The effective thermal conductivity of discontinuous FRCs is improved by increasing the fibre volume fraction. The deviation of the values is highly depends on the randomness of the fibre generation procedure. Discrete fibre elements can easily capture these aspects by means of the FEA. 
%		\item At higher volume fractions, the sensitivity of the effective properties to mesh density becomes more sensible. Thus, a sensitivity analysis is recommended for non-dilute FRCs. The fibre efficiency depends on the fraction of fibres that are oriented along the temperature gradient. Moreover, one must distinguish between the nominal and true fibre volume fractions when dealing with discrete fibre elements. The deviation becomes highly pronounced at higher volume fractions.
%		\item In modelling the hygroscopic effects of natural fibres, distinguishing between the technical and elemental fibres can be detrimental. The thermal conductivity of saturated natural fibres increases by increasing the porosity of the fibres whereas in the absence moisture, conductivity diminishes.
%	\end{itemize}
%	\item 
%	\begin{itemize}
%		\item In randomly-oriented discontinuous FRCs, which are generated by pseudo-random fibre generation, fibre volume fraction alters the thermal conductivity more than clustering effects. However, non-uniform distribution of the fibres causes a local resistance barrier against conductivity. Spectral analysis of the orientation tensor could be used to detect the principal direction of the fibres but the results of clustering index could not completely relate to conductivity. More localised investigation of the clustering is recommended.
%		\item It was found that orienting the principal direction of fibres along the heat gradient increases the effective conductivity provided that clustering is minimum. The degree of anisotropy of the composite increases as the maximum eigenvalue of the orientation tensor becomes closer to unity. Moreover, spectral analysis seems to be more sensitive to orientation than the clustering index. Thus, spectral analysis of the orientation tensor could aid in determining fibre efficiency in thermal and mechanical analysis.
%	\end{itemize}
%	\item 
%	\begin{itemize}
%		\item Assuming a perfect circular cross-section for fibres, results in underestimation of elastic properties, and thus analytical bounding models should be rectified by the FACF. Moreover, modelling technical fibres as discrete finite elements also necessitates a similar amendment. Implicit modelling of damage can be carried out by removing fibre/matrix finite elements upon failure. The element elimination technique was extended to embedded elements in NFRCs. This incorporates a softening effect on the effective elastic modulus of the FRC, and thus a non-linear behaviour is artificially induced without having any available data on damage. In order to deal with failure strength, one should consider a fracture mechanical point of view: shorter fibres are stronger; this argument is also backed up by statistical data from experiments. It is recommended that the strength of fibres must be updated upon a partial failure in a way that the remaining shortened portion of the original fiber acquires a higher failure stress. Although this updating does not affect the effective elasticity, the failure strength approaches the experimental values.
%		\item The concept of auxiliary maps is introduced that contain localised data. By means of a semi-numerical element-wise scheme it was shown that linking the mesh density to the resolution of auxiliary maps results in quick convergence of effective elasticity. Furthermore, unlike spectral analysis, the derived formulation seems to be sensitive to localised data. This feature could be linked to the use of orientation tensor auxiliary map with an appropriate resolution. The model is capable of incorporating the element removal framework for damage modelling. From a micro-mechanical point of view, this capability differs from other models since elements are not linked to a single RV. Overall, the used of auxiliary maps is recommended against the more common global or layer-wise calculations. It was shown that mesh-independent results were obtained using auxiliary maps and through a very cost effective 2D analysis. This could be a good recommendation that as a result of specific manufacturing techniques, 3D computations could be bypassed. The application of the developed method was carried out for NFRCs but could be used for every type of FRCs. Namely, with adequate local morphological information, the computationally inexpensive simulation of mechanical response could be carried out for every type of FRCs.
%	\end{itemize}
%\end{enumerate}
%	
%	\paragraph{Future work} The final stages of the current work lays a solid understanding of the mechanical response of NFRCs that could potentially be extended to other types of FRCs. More specifically, the following ideas could be pursued:
%	\begin{itemize}
%		\item Auxiliary maps of clustering index and/or damage might be used in conjunction with the introduced models to elaborate the understanding of material response in future works. A comparison between the local and global failure criteria should be carried out to challenge the proposed method in strength calculations---specially when element removal is involved.
%		\item The strength updating procedure could be implemented analytically in the model.
%		\item The introduced method can be used in multi-scale modelling to set up a simple representative volume element with embedded fibres in the meso-scale. 
%		\item An explicit damage parameter could be included in the model by including isotropic/anisotropic damage tensor.
%		\item In addition, element removal scheme could be further improved by considering a non-local continua formulation to make the damage propagation smoother and alleviate the artificial stress concentration due to sudden fibre removal.
%		\item The development of an invariant formulation in terms of higher order orientation tensor could be investigated.
%	\end{itemize}
%	