%!TeX spellcheck = en_GB
%!TEX TS-program = xelatex
%!BIB TS-program = biber

\chapter{Concluding Remarks}\label{chap:conc}

\paragraph{Key findings} The key findings of the conducted research can be summarised as follows:

% contingency measure

\begin{enumerate}[label=Stage~\Roman*.]
	\item 
	\begin{itemize}
		\item The effective thermal conductivity of discontinuous FRCs is improved by increasing the fibre volume fraction. The deviation of the values is highly dependant on the randomness of the fibre generation procedure. Discrete fibre elements can easily capture these aspects by means of the FEA. 
		\item At higher volume fractions, the sensitivity of the effective properties to mesh density becomes more sensible. Thus, a sensitivity analysis is recommended for non-dilute FRCs. The fibre efficiency depends on the fraction of fibres that are oriented along the temperature gradient. Moreover, one must distinguish between the nominal and true fibre volume fractions when dealing with discrete fibre elements since these values do not always correspond. The deviation becomes highly pronounced at higher volume fractions.
		\item In modelling the hygroscopic effects of natural fibres, distinguishing between the technical and elemental fibres can be detrimental. The thermal conductivity of saturated natural fibres increases by increasing the porosity of the fibres whereas in the absence of moisture, conductivity diminishes.
	\end{itemize}
	\item 
	\begin{itemize}
		\item In randomly-oriented discontinuous FRCs, which are generated by pseudo-random fibre generation, the fibre volume fraction alters the thermal conductivity more than the clustering effects. However, the non-uniform distribution of the fibres causes a local resistance barrier against conductivity. Spectral analysis of the orientation tensor could be used to detect the principal direction of the fibres but the results of clustering index could not completely relate to conductivity. More localised investigation of the clustering is recommended.
		\item It was found that orienting the principal direction of fibres along the heat gradient increases the effective conductivity provided that clustering is minimum. The degree of anisotropy of the composite increases as the maximum eigenvalue of the orientation tensor becomes closer to unity. Moreover, spectral analysis seems to be more sensitive to orientation than the clustering index. Thus, spectral analysis of the orientation tensor could aid in determining the fibre efficiency in the thermal and mechanical analyses.
	\end{itemize}
	\item 
	\begin{itemize}
		\item Assuming a perfect circular cross-section for fibres, results in underestimation of elastic properties, and thus analytical bounding models should be rectified by the FACF. Moreover, modelling technical fibres as discrete finite elements also necessitates a similar amendment. Implicit modelling of damage can be carried out by removing the fibre/matrix finite elements upon failure. The element elimination technique was extended to simulating NFRCs with embedded elements. This incorporates a softening effect into the effective elastic modulus of the FRC, and thus a non-linear behaviour is artificially induced without having any available data on damage. In order to deal with the failure strength, one should acquire a fracture mechanical point of view, i.e., to consider that shorter fibres are stronger; this argument is also backed up by statistical data from experiments. It is recommended that the strength of fibres must be updated upon a partial failure in a way that the remaining shortened portion of the original fiber attains a higher failure stress. Although this updating does not affect the effective elasticity, the failure strength of the NFRCs approaches the experimental values.
		\item The concept of auxiliary maps is introduced, which is map containing localised data. By means of a semi-numerical element-wise scheme it was shown that linking the mesh density to the resolution of auxiliary maps results in the quick convergence of effective elasticity. Furthermore, unlike spectral analysis, the derived formulation seems to be sensitive to localised data. This feature could be linked to the use of the orientation tensor auxiliary map with an appropriate resolution. Moreover, the model is capable of incorporating the element removal framework for damage modelling. From a micro-mechanical point of view, this capability differs from other models since elements are not linked to a single RVE but instead, each element has its own separate localised model. Overall, the use of auxiliary maps is recommended against the more common global or layer-wise calculations. It was shown that mesh-independent results were obtained using auxiliary maps and through a very cost effective equivalent 2D analysis. This could be a good recommendation that as a result of specific manufacturing techniques, 3D computations could be bypassed. The application of the developed method was carried out for NFRCs but could be applied to any type of FRCs. Namely, provided that adequate local morphological information exists, the computationally inexpensive simulation of mechanical response could be carried out for any type of FRCs. To this end, an in-house portable program is developed that carries out prototyping, pre- and post-analyses, automatically.
	\end{itemize}
\end{enumerate}
	
	\paragraph{Future work} The final stages of the current work lays a solid understanding of the mechanical response of NFRCs that could potentially be extended to other types of FRCs. More specifically, the following ideas could be pursued:
	\begin{itemize}
		\item Auxiliary maps of clustering index and/or damage might be used in conjunction with the introduced models to elaborate the understanding of the material response in future works. A comparison between the local and global failure criteria should be carried out to challenge the proposed method in strength calculations---specially when element removal is involved.
		\item The strength-updating procedure could be implemented analytically in the model.
		\item The introduced method can be used in multi-scale modelling to set up a simple RVE with embedded fibres in the meso-scale. 
		\item An explicit damage parameter could be included in the model by including the isotropic/anisotropic damage tensor.
		\item In addition, the element elimination scheme could be further improved by considering a non-local continua formulation, such as the non-local strain field, to make the damage propagation smoother and alleviate the artificial stress concentration due to the sudden fibre removal.
		\item The development of an invariant formulation in terms of higher order orientation tensor could be investigated.
	\end{itemize}

			