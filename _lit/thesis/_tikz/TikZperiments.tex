% !TeX TS-program = xelatex

\documentclass[tikz,fontsize=12pt,class=scrbook, paper=a4, headinclude, margin=3cm]{standalone}





%\usepackage[margin=3cm, a4paper]{geometry}
\usepackage{pgfplots}                            % loads all the pgf packages
\usepackage[]{contour}                           % for outlining text
\usepackage{tikz-3dplot,comment}


\usetikzlibrary{spy,fit,matrix,shapes.callouts,calc,trees,positioning,arrows,chains,shapes.geometric,shapes.multipart,arrows.meta,  decorations.pathreplacing,decorations.text,decorations.pathmorphing,decorations.markings,shapes,matrix,shapes.symbols,patterns,datavisualization,datavisualization.formats.functions}

\pgfplotsset{compat=newest}

\storeareas\standalonelayout% save the standalone layout
\recalctypearea% recalculate the typearea layout
\edef\savedtextheight{\the\textheight}% save the text height of the typearea layout
\standalonelayout% restore the standalone layout



% ──────────────────────────────────────────────────────────────────────────────────────────────────
% draw a red rectangle aligned with the upper left corner of the TikZ picture
% to show the size of the text area in the KOMA document:
\tikzset{
  every picture/.append style={
    execute at end picture={%
      \draw[red](current bounding box.north west)
        rectangle
        ++(\the\textwidth,-\savedtextheight);%
}}}


% ──────────────────────────────────────────────────────────────────────────────────────────────────
% Definition of pic

% width and length of pic arrows
\newlength{\picw}
\newlength{\picl}
\setlength{\picw}{0.13pt}
\setlength{\picl}{0.25pt}

\tikzset{
 xaxis/.pic     ={\path[draw, fill,  pic actions] (0,0,0) -- ++(0,0,\picw/2) -- ++(\picl,0,-\picw/2)-- ++ (-\picl,0,-\picw/2) -- cycle;},
 xaxisflat/.pic ={\path[draw, fill,  pic actions] (0,0,0) -- ++(0,\picw/2,0) -- ++(\picl,-\picw/2,0)-- ++ (-\picl,-\picw/2,0) -- cycle;},
 yaxis/.pic     ={\path[draw, fill,  pic actions] (0,0,0) -- ++(0,0,\picw/2) -- ++(0,\picl,-\picw/2)-- ++ (0,-\picl,-\picw/2) -- cycle;},
 yaxisflat/.pic ={\path[draw, fill,  pic actions] (0,0,0) -- ++(\picw/2,0,0) -- ++(-\picw/2,\picl,0)-- ++ (-\picw/2,-\picl,0) -- cycle;}, 
 zaxis/.pic     ={\path[draw, fill,  pic actions] (0,0,0) -- ++(0,\picw/2,0) -- ++(0,-\picw/2,\picl)-- ++ (0,-\picw/2,-\picl) -- cycle;},
 zaxisrot/.pic  ={\path[draw, fill,  pic actions] (0,0,0) -- ++(-\picw/2,0,0) -- ++(\picw/2,0,\picl)-- ++ (+\picw/2,0,-\picl) -- cycle;}
}


% Setting a right-handed 
\setlength{\unitlength}{1mm}
\tdplotsetmaincoords{235}{-40}

% ──────────────────────────────────────────────────────────────────────────────────────────────────
% using the definition from the thesis

\usepackage{mdframed, amsmath, amsthm, tcolorbox, unicode-math, tkz-euclide}
\input{../_style/commonmacros.tex}




% Dimensions of the plate
\newlength{\plx}
\newlength{\ply}
\newlength{\plt}

\setlength{\plx}{6pt}
\setlength{\ply}{6pt}
\setlength{\plt}{2pt}

% Arrow dimensions
\newlength{\annotdim}
\setlength{\annotdim}{5.5pt}

\newlength{\annottext}
\setlength{\annottext}{10pt}

% Force dimension
\newlength{\forcelen}
\setlength{\forcelen}{1.5pt}

\newlength{\baselen}
\setlength{\baselen}{1pt}


% //////////////////////////////////////////////////////////////////////////////////////////////////
\begin{document}
\centering
\begin{tikzpicture}
    \begin{scope}[opacity=0.2]

        \path[draw, fill=gray] (0,0) rectangle (10,5) node [above,midway,opacity=1] {Generlized continua};


    \end{scope}



\end{tikzpicture}
\end{document}
% //////////////////////////////////////////////////////////////////////////////////////////////////












\begin{tikzpicture}[tdplot_main_coords,framed]
\begin{scope}


%% side surfaces
%\path[draw, fill=gray, opacity=0.20] (\the\plx,0,\the\plt/2) -- ++(0,\the\ply,0) -- ++(0,0,-\the\plt) -- ++(0,-\the\ply,0) -- cycle;
%\path[draw, fill=gray, opacity=0.20] (0,\the\ply,\the\plt/2) -- ++(\the\plx,0,0) -- ++(0,0,-\the\plt) -- ++(-\the\ply,0,0) -- cycle;

% 3 axis & bottom layer
\draw[thick] (0,0,0) -- (0,0,1) node[anchor=west] {\contour{white}{$\tena{n}$}} pic[scale=0.65] {zaxis};
%\path[draw, fill=gray, opacity=0.20] (0,0,+\the\plt/2) -- ++(\the\plx,0,0) -- ++(0,\the\ply,0) -- ++(-\the\plx,0,0) -- cycle;

% line at the back of the COORDSYS
%\path[draw, fill=gray, opacity=0.20] (0,0,-\the\plt/2) -- ++(0,0,\the\plt);

% mid layer
\path[draw, thick, fill=gray, opacity=0.5, dashed]  (0,0,0) -- ++(\the\plx,0,0) -- ++(0,\the\ply,0) -- ++(-\the\plx,0,0) -- cycle;

% 1 and 2 directions
\draw[thick, line cap = round] (0,0,0) -- (1,0,0) node[anchor=south west] {\contour{white}{$\base_1$}} pic[scale=0.65] {xaxisflat};
\draw[thick, line cap = round] (0,0,0) -- (0,1,0) node[anchor=south east] {\contour{white}{$\base_2$}} pic [scale=0.65] {yaxisflat};

% top layer
%\path[draw, fill=gray, opacity=0.20] (0,0,-\the\plt/2) -- ++(\the\plx,0,0) -- ++(0,\the\ply,0) -- ++(-\the\plx,0,0) -- cycle;


% Dimensions
\begin{scope}[{Classical TikZ Rightarrow[length=\the\annotdim, width=\the\annotdim/2]}-{Classical TikZ Rightarrow[length=\the\annotdim, width=\the\annotdim/2]}  ]

% height
%\path[draw] (-0.2,\the\ply+.2,-\the\plt/2) -- ++ (0,0,+\the\plt/2);
%\path[draw] (-0.2,\the\ply+.2,+\the\plt/2) -- ++ (0,0,-\the\plt/2) node [at end, left] {$\textstyle 2\times \frac{h}{2}$};

% Length
%\path[draw] (0,\the\ply,\the\plt-0.1) -- ++ (\the\plx,0,0) node [midway, below left] {$\ell_1$};
%\path[draw] (\the\plx,0,\the\plt-0.1) -- ++ (0,\the\ply,0) node [midway, below right] {$\ell_2$};

\end{scope}

% Annotation
\begin{scope}[-{Kite[fill=white, length=\the\annottext]}]
 \path[draw] (\the\plx*2/3,0,0) -- +(1.5,-1.5,1) node [pos=0.1, anchor=south west] {\contour{white}{$\partial\mathfrak{S}$}};
 \node at (\the\plx-0.5,0.5,0) [] {$\mathfrak{S}$};
\end{scope}

% Surface forces
\begin{scope}
  \draw[thick, line cap = round] (\the\plx*2/3,\the\ply*2/3,0) -- +(-1,0,0) node[anchor=north east] {\contour{white}{$\tena{s}_1$}} pic[rotate=180,scale=0.65] {xaxisflat};
  \draw[thick, line cap = round] (\the\plx*2/3,\the\ply*2/3,0) -- +(0,-1,0) node[anchor=north west] {\contour{white}{$\tena{s}_2$}} pic[rotate=180,scale=0.65] {yaxisflat};
  \draw[thick, line cap = round] (\the\plx*2/3,\the\ply*2/3,-1) node[anchor=south west, xshift=-0.5] {\contour{white}{$\tena{p}$}} -- +(0,0,+1-\the\picl) pic[scale=0.65] {zaxisrot};
\end{scope}

% normal vectors to the contour
\begin{scope}
  \draw[thick, line cap = round] (\the\plx,\the\ply/2,0) -- +(+1,0,0) node[anchor=south west] {\contour{white}{$\tena{\upsilon}$}} pic[scale=0.65] {xaxisflat};
  \draw[thick, line cap = round] (\the\plx/2,\the\ply,0) -- +(0,1,0) node[anchor=north west] {\contour{white}{$\tena{\upsilon}$}} pic[scale=0.65] {yaxisflat};
\end{scope}

% arclength
\begin{scope}
  \draw[-,thick, line cap = round] (\the\plx,\the\ply,0) -- +(-1,0,0) node[anchor=north east] {s} pic[rotate=180,scale=0.65] {xaxisflat};
\end{scope}

\end{scope}


