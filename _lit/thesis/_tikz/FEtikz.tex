% ══════════════════════════════════════════════════════════════════════════════════════════════════
%
%
%
% Last update 19/12/2017
% ══════════════════════════════════════════════════════════════════════════════════════════════════



%\newcommand\CL{\sffamily C\kern -4pt\raise -2pt\hbox{L}}
\newcommand\CL{\sffamily CL}
% Roman element numbers - lowercase
\def\rnum#1{\scriptsize\romannumeral #1}
% Roman element numbers - uppercase
\def\Rnum#1{\scriptsize\uppercase\expandafter{\romannumeral #1\relax}}
% Roman number for the text
\def\Rnumt#1{\uppercase\expandafter{\romannumeral #1\relax}}


% tikzbeam
\pgfkeys{/tikz/beam/.cd,
x/.initial=0cm,
y/.initial=0cm,
length/.initial=4cm,
width/.initial=0.3cm,
fill/.initial=black!15,
border/.initial=1pt,
axis thickness/.initial=0.5pt,
rotation/.initial=0,
axis/.code={%
\draw[line width=\pgfkeysvalueof{/tikz/beam/axis thickness},
      dash pattern=on 2pt off 2pt on 8pt off 2pt,
      color=black,
      rotate around={\pgfkeysvalueof{/tikz/beam/rotation}:(\pgfkeysvalueof{/tikz/beam/x},\pgfkeysvalueof{/tikz/beam/y})}] (\pgfkeysvalueof{/tikz/beam/x},\pgfkeysvalueof{/tikz/beam/y}) -- ++(\pgfkeysvalueof{/tikz/beam/length},0);},
cap start end/.code={%
\draw[line width=\pgfkeysvalueof{/tikz/beam/width},
      color=black,
      rotate around={\pgfkeysvalueof{/tikz/beam/rotation}:(\pgfkeysvalueof{/tikz/beam/x},\pgfkeysvalueof{/tikz/beam/y})}] 
      (\pgfkeysvalueof{/tikz/beam/x}-\pgfkeysvalueof{/tikz/beam/border}/2,\pgfkeysvalueof{/tikz/beam/y}) -- 
    ++(\pgfkeysvalueof{/tikz/beam/length}+\pgfkeysvalueof{/tikz/beam/border},0);
\draw[line width=\pgfkeysvalueof{/tikz/beam/width}-\pgfkeysvalueof{/tikz/beam/border}/0.5,
      color=\pgfkeysvalueof{/tikz/beam/fill},
      rotate around={\pgfkeysvalueof{/tikz/beam/rotation}:(\pgfkeysvalueof{/tikz/beam/x},\pgfkeysvalueof{/tikz/beam/y})}] (\pgfkeysvalueof{/tikz/beam/x}+\pgfkeysvalueof{/tikz/beam/border}/2,\pgfkeysvalueof{/tikz/beam/y}) -- ++(\pgfkeysvalueof{/tikz/beam/length}-\pgfkeysvalueof{/tikz/beam/border},0cm);},
uncap start/.code={
\draw[line width=\pgfkeysvalueof{/tikz/beam/width},
      color=black,
      rotate around={\pgfkeysvalueof{/tikz/beam/rotation}:(\pgfkeysvalueof{/tikz/beam/x},\pgfkeysvalueof{/tikz/beam/y})}] 
      (\pgfkeysvalueof{/tikz/beam/x},\pgfkeysvalueof{/tikz/beam/y}) -- ++(\pgfkeysvalueof{/tikz/beam/length}+\pgfkeysvalueof{/tikz/beam/border}/2,0);
\draw[line width=\pgfkeysvalueof{/tikz/beam/width}-\pgfkeysvalueof{/tikz/beam/border}/0.5,
      color=\pgfkeysvalueof{/tikz/beam/fill},
      rotate around={\pgfkeysvalueof{/tikz/beam/rotation}:(\pgfkeysvalueof{/tikz/beam/x},\pgfkeysvalueof{/tikz/beam/y})}] 
      (\pgfkeysvalueof{/tikz/beam/x},\pgfkeysvalueof{/tikz/beam/y}) -- ++(\pgfkeysvalueof{/tikz/beam/length}-\pgfkeysvalueof{/tikz/beam/border}/2,0cm);
},
uncap end/.code={%
\draw[line width=\pgfkeysvalueof{/tikz/beam/width},
      color=black,
      rotate around={\pgfkeysvalueof{/tikz/beam/rotation}:(\pgfkeysvalueof{/tikz/beam/x},\pgfkeysvalueof{/tikz/beam/y})}]%
      (\pgfkeysvalueof{/tikz/beam/x}-\pgfkeysvalueof{/tikz/beam/border}/2,\pgfkeysvalueof{/tikz/beam/y}) -- ++(\pgfkeysvalueof{/tikz/beam/length}+\pgfkeysvalueof{/tikz/beam/border}/2,0);
%
\draw[line width=\pgfkeysvalueof{/tikz/beam/width}-\pgfkeysvalueof{/tikz/beam/border}/0.5,
      color=\pgfkeysvalueof{/tikz/beam/fill},
      rotate around={\pgfkeysvalueof{/tikz/beam/rotation}:(\pgfkeysvalueof{/tikz/beam/x},\pgfkeysvalueof{/tikz/beam/y})}] (\pgfkeysvalueof{/tikz/beam/x}+\pgfkeysvalueof{/tikz/beam/border}/2,\pgfkeysvalueof{/tikz/beam/y}) -- ++(\pgfkeysvalueof{/tikz/beam/length}-\pgfkeysvalueof{/tikz/beam/border}/2,0cm);
},
uncap start end/.code={
\draw[line width=\pgfkeysvalueof{/tikz/beam/width},
      color=black,
      rotate around={\pgfkeysvalueof{/tikz/beam/rotation}:(\pgfkeysvalueof{/tikz/beam/x},\pgfkeysvalueof{/tikz/beam/y})}] (\pgfkeysvalueof{/tikz/beam/x},\pgfkeysvalueof{/tikz/beam/y}) -- 
    ++(\pgfkeysvalueof{/tikz/beam/length},0);
    
\draw[line width=\pgfkeysvalueof{/tikz/beam/width}-\pgfkeysvalueof{/tikz/beam/border}/0.5,
      color=\pgfkeysvalueof{/tikz/beam/fill},
      rotate around={\pgfkeysvalueof{/tikz/beam/rotation}:(\pgfkeysvalueof{/tikz/beam/x},\pgfkeysvalueof{/tikz/beam/y})}] (\pgfkeysvalueof{/tikz/beam/x},\pgfkeysvalueof{/tikz/beam/y}) -- ++(\pgfkeysvalueof{/tikz/beam/length},0cm);

}
}
\newcommand{\tikzbeam}[1][]{\tikzset{/tikz/beam/.cd,#1}}
% ══════════════════════════════════════════════════════════════════════════════════════════════════
% tikzlabel
\pgfkeys{/tikz/label/.cd,
x/.initial=0cm,
y/.initial=0cm,
x distance/.initial=0.2cm,
y distance/.initial=-0.4cm,
node option/.style={below right,pos=0.6},
path option/.style={draw},
line thickness/.initial=0.5pt,
text/.initial=$EA$,
draw/.code={\path[line width=\pgfkeysvalueof{/tikz/label/line thickness},/tikz/label/path option] (\pgfkeysvalueof{/tikz/label/x},\pgfkeysvalueof{/tikz/label/y}) -- +(\pgfkeysvalueof{/tikz/label/x distance},\pgfkeysvalueof{/tikz/label/y distance}) node [/tikz/label/node option] {\pgfkeysvalueof{/tikz/label/text}};
},
top left/.style={y distance=0.4cm, x distance=-0.2cm,node option/.style={above left,pos=0.7}},
top right/.style={y distance=0.4cm, x distance=0.2cm,node option/.style={above right,pos=0.8,xshift=-0.1cm}},
top/.style={y distance=0.4cm, x distance=0cm,node option/.style={above,pos=0.2}},
no line/.code={\pgfkeys{/tikz/label/path option/.style={}}},
use arrow/.style={path option/.style={draw,arrows={Stealth[length=2.5mm,fill=white]-}}},
reverse arrow/.style={path option/.style={draw,arrows={-Stealth[length=2.5mm,fill=white]}}}
}
\newcommand{\tikzlabel}[1][]{\tikzset{/tikz/label/.cd,#1}}
% ══════════════════════════════════════════════════════════════════════════════════════════════════
% tikzsupport
\pgfkeys{/tikz/support/.cd,
x/.initial=0cm,
y/.initial=0cm,
height/.initial=1.25cm,
width/.initial=0.2cm,
fill/.initial=black!15,
fill/.initial=black!15,
border/.initial=1pt,
rotation/.initial=0,
sliding distance/.initial=0.1cm,
pinned/height/.initial=0.35cm,
pinned/depth/.initial=0.15cm,
pinned/width/.initial=0.6cm,
pinned/gap/.initial=0.1cm,
pinned/thickness/.initial=0.5pt,
pinned/border thickness/.initial=1pt,
fixed/.code={
    \draw [draw=none,pattern={north east lines},
    rotate around={\pgfkeysvalueof{/tikz/support/rotation}:(\pgfkeysvalueof{/tikz/support/x},\pgfkeysvalueof{/tikz/support/y})}]
   (\pgfkeysvalueof{/tikz/support/x}-\pgfkeysvalueof{/tikz/support/width},\pgfkeysvalueof{/tikz/support/y}-\pgfkeysvalueof{/tikz/support/height}/2) rectangle
   (\pgfkeysvalueof{/tikz/support/x},\pgfkeysvalueof{/tikz/support/y}+\pgfkeysvalueof{/tikz/support/height}/2);
   \draw [line width=\pgfkeysvalueof{/tikz/support/border}, rotate around={\pgfkeysvalueof{/tikz/support/rotation}:(\pgfkeysvalueof{/tikz/support/x},\pgfkeysvalueof{/tikz/support/y})}] (\pgfkeysvalueof{/tikz/support/x},\pgfkeysvalueof{/tikz/support/y}-\pgfkeysvalueof{/tikz/support/height}/2) -- +(0cm,\pgfkeysvalueof{/tikz/support/height});
},
fixed sliding/.code={
\draw [draw=none,pattern={north east lines},
    rotate around={\pgfkeysvalueof{/tikz/support/rotation}:(\pgfkeysvalueof{/tikz/support/x},\pgfkeysvalueof{/tikz/support/y})}]
   (\pgfkeysvalueof{/tikz/support/x}-\pgfkeysvalueof{/tikz/support/width}-\pgfkeysvalueof{/tikz/support/sliding distance},\pgfkeysvalueof{/tikz/support/y}-\pgfkeysvalueof{/tikz/support/height}/2) rectangle
   (\pgfkeysvalueof{/tikz/support/x}-\pgfkeysvalueof{/tikz/support/sliding distance},\pgfkeysvalueof{/tikz/support/y}+\pgfkeysvalueof{/tikz/support/height}/2);
   \draw [line width=\pgfkeysvalueof{/tikz/support/border}, rotate around={\pgfkeysvalueof{/tikz/support/rotation}:(\pgfkeysvalueof{/tikz/support/x},\pgfkeysvalueof{/tikz/support/y})}] (\pgfkeysvalueof{/tikz/support/x},\pgfkeysvalueof{/tikz/support/y}-\pgfkeysvalueof{/tikz/support/height}/2) -- +(0cm,\pgfkeysvalueof{/tikz/support/height});

   \draw [line width=\pgfkeysvalueof{/tikz/support/border},rotate around={\pgfkeysvalueof{/tikz/support/rotation}:(\pgfkeysvalueof{/tikz/support/x},\pgfkeysvalueof{/tikz/support/y})}] (\pgfkeysvalueof{/tikz/support/x}-\pgfkeysvalueof{/tikz/support/sliding distance},\pgfkeysvalueof{/tikz/support/y}-\pgfkeysvalueof{/tikz/support/height}/2) -- +(0cm,\pgfkeysvalueof{/tikz/support/height});
},
pinned/.code={
\pgfkeysgetvalue{/tikz/support/y}{\temp}
\pgfkeyssetvalue{/tikz/support/y}{\temp-0.04cm}
\path[draw,line width=\pgfkeysvalueof{/tikz/support/pinned/border thickness}, rotate around={\pgfkeysvalueof{/tikz/support/rotation}:(\pgfkeysvalueof{/tikz/support/x},\pgfkeysvalueof{/tikz/support/y})}] (\pgfkeysvalueof{/tikz/support/x}+\pgfkeysvalueof{/tikz/support/pinned/height}/2,
\pgfkeysvalueof{/tikz/support/y}-\pgfkeysvalueof{/tikz/support/pinned/height}) -- ++(-\pgfkeysvalueof{/tikz/support/pinned/height}/2,
+\pgfkeysvalueof{/tikz/support/pinned/height}) -- ++(-\pgfkeysvalueof{/tikz/support/pinned/height}/2,-\pgfkeysvalueof{/tikz/support/pinned/height});
\draw[line width=\pgfkeysvalueof{/tikz/support/pinned/border thickness}, rotate around={\pgfkeysvalueof{/tikz/support/rotation}:(\pgfkeysvalueof{/tikz/support/x},\pgfkeysvalueof{/tikz/support/y})}] (\pgfkeysvalueof{/tikz/support/x}-\pgfkeysvalueof{/tikz/support/pinned/width}/2,\pgfkeysvalueof{/tikz/support/y}-\pgfkeysvalueof{/tikz/support/pinned/height}) -- 
++(\pgfkeysvalueof{/tikz/support/pinned/width},0);
\draw [draw=none,pattern={north east lines},
    rotate around={\pgfkeysvalueof{/tikz/support/rotation}:(\pgfkeysvalueof{/tikz/support/x},\pgfkeysvalueof{/tikz/support/y})}]
  (\pgfkeysvalueof{/tikz/support/x}-\pgfkeysvalueof{/tikz/support/pinned/width}/2,\pgfkeysvalueof{/tikz/support/y}-\pgfkeysvalueof{/tikz/support/pinned/height}-\pgfkeysvalueof{/tikz/support/pinned/depth}) rectangle
   +(\pgfkeysvalueof{/tikz/support/pinned/width},\pgfkeysvalueof{/tikz/support/pinned/depth});
},
roller/.code={
\pgfkeysgetvalue{/tikz/support/y}{\temp}
\pgfkeyssetvalue{/tikz/support/y}{\temp-0.04cm}
\path[draw,line width=\pgfkeysvalueof{/tikz/support/pinned/border thickness}, rotate around={\pgfkeysvalueof{/tikz/support/rotation}:(\pgfkeysvalueof{/tikz/support/x},\pgfkeysvalueof{/tikz/support/y})}] (\pgfkeysvalueof{/tikz/support/x}+\pgfkeysvalueof{/tikz/support/pinned/height}/2,
\pgfkeysvalueof{/tikz/support/y}-\pgfkeysvalueof{/tikz/support/pinned/height}) -- ++(-\pgfkeysvalueof{/tikz/support/pinned/height}/2,
+\pgfkeysvalueof{/tikz/support/pinned/height}) -- ++(-\pgfkeysvalueof{/tikz/support/pinned/height}/2,-\pgfkeysvalueof{/tikz/support/pinned/height}) -- cycle;

\draw[line width=\pgfkeysvalueof{/tikz/support/pinned/border thickness}, rotate around={\pgfkeysvalueof{/tikz/support/rotation}:(\pgfkeysvalueof{/tikz/support/x},\pgfkeysvalueof{/tikz/support/y})}] (\pgfkeysvalueof{/tikz/support/x}-\pgfkeysvalueof{/tikz/support/pinned/width}/2,\pgfkeysvalueof{/tikz/support/y}-\pgfkeysvalueof{/tikz/support/pinned/height}-\pgfkeysvalueof{/tikz/support/pinned/gap}) -- ++(\pgfkeysvalueof{/tikz/support/pinned/width},0);

\draw [draw=none,pattern={north east lines},
    rotate around={\pgfkeysvalueof{/tikz/support/rotation}:(\pgfkeysvalueof{/tikz/support/x},\pgfkeysvalueof{/tikz/support/y})}]
  (\pgfkeysvalueof{/tikz/support/x}-\pgfkeysvalueof{/tikz/support/pinned/width}/2,\pgfkeysvalueof{/tikz/support/y}-\pgfkeysvalueof{/tikz/support/pinned/height}-\pgfkeysvalueof{/tikz/support/pinned/depth}-\pgfkeysvalueof{/tikz/support/pinned/gap}) rectangle
   +(\pgfkeysvalueof{/tikz/support/pinned/width},\pgfkeysvalueof{/tikz/support/pinned/depth});
}
}
\newcommand{\tikzsupport}[1][]{\tikzset{/tikz/support/.cd,#1}}

% tikz
\pgfkeys{/tikz/point/.cd,
x/.initial=0cm,
y/.initial=0cm,
radius/.initial=0.05cm,
draw/.code={\draw [fill] (\pgfkeysvalueof{/tikz/point/x},\pgfkeysvalueof{/tikz/point/y}) circle [radius=\pgfkeysvalueof{/tikz/point/radius}];}
}
\newcommand{\tikzpoint}[1][]{\tikzset{/tikz/point/.cd,#1}}

% tikzforce
\pgfkeys{/tikz/force/.cd,
x/.initial=0cm,
y/.initial=0cm,
%x distance/.initial=0.2cm,
%y distance/.initial=-0.4cm,
radius/.initial=0.03cm,
length/.initial=0.5cm,
thickness/.initial=1pt,
rotation/.initial=0,
text/.initial=$F_0$,
node option/.style={below,pos=0.9,yshift=-0.05cm},
downward/.style={rotation=-90,node option/.style={above left,pos=0.4, yshift=-0.1cm}},
upward/.style={rotation=+90,node option/.style={above left,pos=0.6, yshift=-0.1cm}},
point load/.code={
\draw [fill] (\pgfkeysvalueof{/tikz/force/x},\pgfkeysvalueof{/tikz/force/y}) circle [radius=\pgfkeysvalueof{/tikz/force/radius}];
\draw [line width=\pgfkeysvalueof{/tikz/force/thickness},
       -latex,
       rotate around={\pgfkeysvalueof{/tikz/force/rotation}:(\pgfkeysvalueof{/tikz/force/x},\pgfkeysvalueof{/tikz/force/y})}] (\pgfkeysvalueof{/tikz/force/x},\pgfkeysvalueof{/tikz/force/y})  -- +(\pgfkeysvalueof{/tikz/force/length},0) node [/tikz/force/node option] {\pgfkeysvalueof{/tikz/force/text}};
},
load/.code={
\draw [line width=\pgfkeysvalueof{/tikz/force/thickness},
       -latex,
       rotate around={\pgfkeysvalueof{/tikz/force/rotation}:(\pgfkeysvalueof{/tikz/force/x},\pgfkeysvalueof{/tikz/force/y})}] (\pgfkeysvalueof{/tikz/force/x},\pgfkeysvalueof{/tikz/force/y})  -- +(\pgfkeysvalueof{/tikz/force/length},0) node [/tikz/force/node option] {\pgfkeysvalueof{/tikz/force/text}};
},
reaction load/.code={
\draw [line width=\pgfkeysvalueof{/tikz/force/thickness},
       -latex,
       rotate around={\pgfkeysvalueof{/tikz/force/rotation}:(\pgfkeysvalueof{/tikz/force/x},\pgfkeysvalueof{/tikz/force/y})},postaction= {decorate,decoration={markings,mark= at position 0.4 with {\draw[-] ++(-0.05,0.05) -- (+0.05,-0.05);}
       } }] (\pgfkeysvalueof{/tikz/force/x}-\pgfkeysvalueof{/tikz/force/length},\pgfkeysvalueof{/tikz/force/y})  -- +(+\pgfkeysvalueof{/tikz/force/length},0) node [/tikz/force/node option] {\pgfkeysvalueof{/tikz/force/text}};

}
}
\newcommand{\tikzforce}[1][]{\tikzset{/tikz/force/.cd,#1}}

% tikzmoment
\pgfkeys{/tikz/moment/.cd,
x/.initial=0cm,
y/.initial=0cm,
%x distance/.initial=0.2cm,
%y distance/.initial=-0.4cm,
point radius/.initial=0.03cm,
radius/.initial=0.3cm,
thickness/.initial=1pt,
rotation/.initial=0,
text/.initial=$M$,
node option/.style={left,pos=0.5,xshift=0.1cm},
moment/.code={
\draw [line width=\pgfkeysvalueof{/tikz/moment/thickness}, -{Latex[sep=-4,scale=0.7]},
       rotate around={\pgfkeysvalueof{/tikz/moment/rotation}:(\pgfkeysvalueof{/tikz/moment/x},\pgfkeysvalueof{/tikz/moment/y})}]  
       (\pgfkeysvalueof{/tikz/moment/x},\pgfkeysvalueof{/tikz/moment/y}-\pgfkeysvalueof{/tikz/moment/radius}) 
       arc [start angle = 270, end angle=90,radius=\pgfkeysvalueof{/tikz/moment/radius}] node [/tikz/moment/node option] {\pgfkeysvalueof{/tikz/moment/text}};
},
point moment/.code={
\draw [fill] (\pgfkeysvalueof{/tikz/moment/x},\pgfkeysvalueof{/tikz/moment/y}) circle [radius=\pgfkeysvalueof{/tikz/moment/point radius}];
\draw [line width=\pgfkeysvalueof{/tikz/moment/thickness}, -{Latex[sep=-4,scale=0.7]},
       rotate around={\pgfkeysvalueof{/tikz/moment/rotation}:(\pgfkeysvalueof{/tikz/moment/x},\pgfkeysvalueof{/tikz/moment/y})}]  
       (\pgfkeysvalueof{/tikz/moment/x},\pgfkeysvalueof{/tikz/moment/y}-\pgfkeysvalueof{/tikz/moment/radius}) 
       arc [start angle = 270, end angle=90,radius=\pgfkeysvalueof{/tikz/moment/radius}] node [/tikz/moment/node option] {\pgfkeysvalueof{/tikz/moment/text}};
},
reaction moment/.code={
\draw [line width=\pgfkeysvalueof{/tikz/moment/thickness}, -{Latex[sep=-4,scale=0.7]},
       rotate around={\pgfkeysvalueof{/tikz/moment/rotation}:(\pgfkeysvalueof{/tikz/moment/x},\pgfkeysvalueof{/tikz/moment/y})},postaction= {decorate,decoration={markings,mark= at position 0.5 with {\draw[-] ++(-0.05,0.05) -- (+0.05,-0.05);}
       } }] 
       (\pgfkeysvalueof{/tikz/moment/x},\pgfkeysvalueof{/tikz/moment/y}-\pgfkeysvalueof{/tikz/moment/radius}) 
       arc [start angle = 270, end angle=90,radius=\pgfkeysvalueof{/tikz/moment/radius}] node [/tikz/moment/node option] {\pgfkeysvalueof{/tikz/moment/text}};
},
point reaction moment/.code={
\draw [fill] (\pgfkeysvalueof{/tikz/moment/x},\pgfkeysvalueof{/tikz/moment/y}) circle [radius=\pgfkeysvalueof{/tikz/moment/point radius}];
\draw [line width=\pgfkeysvalueof{/tikz/moment/thickness}, -{Latex[sep=-4,scale=0.7]},
       rotate around={\pgfkeysvalueof{/tikz/moment/rotation}:(\pgfkeysvalueof{/tikz/moment/x},\pgfkeysvalueof{/tikz/moment/y})},postaction= {decorate,decoration={markings,mark= at position 0.5 with {\draw[-] ++(-0.05,0.05) -- (+0.05,-0.05);}
       } }] 
       (\pgfkeysvalueof{/tikz/moment/x},\pgfkeysvalueof{/tikz/moment/y}-\pgfkeysvalueof{/tikz/moment/radius}) 
       arc [start angle = 270, end angle=90,radius=\pgfkeysvalueof{/tikz/moment/radius}] node [/tikz/moment/node option] {\pgfkeysvalueof{/tikz/moment/text}};
}
}
\newcommand{\tikzmoment}[1][]{\tikzset{/tikz/moment/.cd,#1}}

% tikzaxis
\pgfkeys{/tikz/axis/.cd,
x1/.initial=0cm,
y1/.initial=0cm,
x2/.initial=1cm,
y2/.initial=1cm,
thickness/.initial=0.5pt,
line type/.initial=densely dash dot,
rotation/.initial=0,
rotation x/.initial=0,
rotation y/.initial=0,
node option/.style={above},
text/.initial=\CL,
draw/.code={
\draw [line width=\pgfkeysvalueof{/tikz/axis/thickness},\pgfkeysvalueof{/tikz/axis/line type}] (\pgfkeysvalueof{/tikz/axis/x1},\pgfkeysvalueof{/tikz/axis/y1}) -- (\pgfkeysvalueof{/tikz/axis/x2},\pgfkeysvalueof{/tikz/axis/y2}) node [/tikz/axis/node option] {\pgfkeysvalueof{/tikz/axis/text}};
}
}
\newcommand{\tikzaxis}[1][]{\tikzset{/tikz/axis/.cd,#1}}



\makeatletter
\newcount\my@repeat@count
\newcommand{\myrepeat}[2]{%
  \begingroup
  \my@repeat@count=\z@
  \@whilenum\my@repeat@count<#1\do{#2\advance\my@repeat@count\@ne}%
  \endgroup
}
\makeatother


% tikzdistributed
\pgfkeys{/tikz/distributed/.cd,
x1/.initial=0cm,
y1/.initial=0cm,
x2/.initial=1cm,
y2/.initial=1cm,
line width/.initial=1pt,
line type/.initial=densely dash dot,
rotation/.initial=0,
text/.initial=$q$,
uniform/length/.initial      = 0.5cm,
uniform/step/.initial        = 0.2cm,
uniform/n/.initial           = 5,
uniform/gap/.initial         = 0.1cm,
uniform/line width/.initial  = 0.5pt,
uniform/node option/.style   = {above},
uni/.code={%
%\pgfkeys{/tikz/distributed/uniform/step=(\pgfkeysvalueof{/tikz/distributed/uniform/x2}-\pgfkeysvalueof{/tikz/distributed/uniform/x1})/ \pgfkeysvalueof{/tikz/distributed/uniform/n}}
\draw [cap=rect, rotate around={\pgfkeysvalueof{/tikz/distributed/rotation}:(\pgfkeysvalueof{/tikz/distributed/x1},\pgfkeysvalueof{/tikz/distributed/y1})},
line width=\pgfkeysvalueof{/tikz/distributed/line width}, 
postaction= {decorate,
             decoration={markings,
                         mark= between positions 0 and 1 step \pgfkeysvalueof{/tikz/distributed/uniform/step} with {\draw[-latex,line width=\pgfkeysvalueof{/tikz/distributed/line width}] ++(0,0) -- (0,-\pgfkeysvalueof{/tikz/distributed/uniform/length});}
                        }
            }] (\pgfkeysvalueof{/tikz/distributed/x1},\pgfkeysvalueof{/tikz/distributed/y1}) -- (\pgfkeysvalueof{/tikz/distributed/x2},\pgfkeysvalueof{/tikz/distributed/y2});
},
uni axial/.code={
% For the axial uniform load, the step of each axial load (uniform/step), i.e., their length, is calculated by specifying the number of arrows (uniform/n). 
\begingroup
\pgfmathsetlengthmacro\step{ (\pgfkeysvalueof{/tikz/distributed/x2} - \pgfkeysvalueof{/tikz/distributed/x1} - 2*\pgfkeysvalueof{/tikz/distributed/uniform/gap}) / \pgfkeysvalueof{/tikz/distributed/uniform/n}}
\pgfmathsetmacro\num{\pgfkeysvalueof{/tikz/distributed/uniform/n}-1}

\draw[line width=\pgfkeysvalueof{/tikz/distributed/uniform/line width}] (\pgfkeysvalueof{/tikz/distributed/x1},\pgfkeysvalueof{/tikz/distributed/y1}) rectangle (\pgfkeysvalueof{/tikz/distributed/x2},\pgfkeysvalueof{/tikz/distributed/y2}) node [distributed/uniform/node option] {\pgfkeysvalueof{/tikz/distributed/text}};

\foreach \counter in {0,...,\num} 
{
\draw[-latex, line width=\pgfkeysvalueof{/tikz/distributed/line width}] 
(\pgfkeysvalueof{/tikz/distributed/x1}+\pgfkeysvalueof{/tikz/distributed/uniform/gap}+\counter*\step,\pgfkeysvalueof{/tikz/distributed/y1}/2+\pgfkeysvalueof{/tikz/distributed/y2}/2) -- +(\step,0cm);
}
\endgroup
}
}

\newcommand{\tikzdistributed}[1][]{\tikzset{/tikz/distributed/.cd,#1}}
%
% Original uniform code which did not work
%
%\pgfmathparse{(\pgfkeysvalueof{/tikz/distributed/y2}-\pgfkeysvalueof{/tikz/distributed/y1})/(\pgfkeysvalueof{/tikz/distributed/x2}-\pgfkeysvalueof{/tikz/distributed/x1})}
%\let\myslope\pgfmathresult%
%\pgfmathparse{(\pgfkeysvalueof{/tikz/distributed/x2}-\pgfkeysvalueof{/tikz/distributed/x1})/(\pgfkeysvalueof{/tikz/distributed/uniform/n})}
%\let\myalpha\pgfmathresult%
%\pgfmathparse{\pgfkeysvalueof{/tikz/distributed/y1}-(\myslope*\pgfkeysvalueof{/tikz/distributed/x1})}
%\let\myabscissa\pgfmathresult%
%\newcounter{mytempcounter}%
%\setcounter{mytempcounter}{0}
%\myrepeat{\pgfkeysvalueof{/tikz/distributed/uniform/n}}{%
%\pgfmathparse{\pgfkeysvalueof{/tikz/distributed/x1}+(\value{mytempcounter}*\myalpha)}%
%\let\myxi\pgfmathresult%
%\pgfmathparse{(\myslope*\myxi)+\myabscissa}%
%\let\myyi\pgfmathresult%
%\draw (\myxi,\myyi) -- +(0,\pgfkeysvalueof{/tikz/distributed/uniform/length});
%\stepcounter{mytempcounter}
%%%%%%%%%%%%%%%%%%%%%%%%%%%%%%%%%%%%%%%%%%%%%


%%%%%%%%%%%%%%%%%%%%%%%%%%%%%%%%%%%%%%%%%%%%%
% tikztruss
%%%%%%%%%%%%%%%%%%%%%%%%%%%%%%%%%%%%%%%%%%%%%
\pgfkeys{/tikz/truss/.cd,
x/.initial=0cm,
y/.initial=0cm,
length/.initial=4cm,
width/.initial=0.2cm,
fill/.initial=black!15,
border/.initial=1pt,
axis thickness/.initial=0.5pt,
rotation/.initial=0,
axis/.code={%
\draw[line width=\pgfkeysvalueof{/tikz/truss/axis thickness},
      dash pattern=on 2pt off 2pt on 8pt off 2pt,
      color=black,
      rotate around={\pgfkeysvalueof{/tikz/truss/rotation}:(\pgfkeysvalueof{/tikz/truss/x},\pgfkeysvalueof{/tikz/truss/y})}] (\pgfkeysvalueof{/tikz/truss/x},\pgfkeysvalueof{/tikz/truss/y}) -- ++(\pgfkeysvalueof{/tikz/truss/length},0cm);},
draw/.code={%
\draw[line cap=round,line width=\pgfkeysvalueof{/tikz/truss/width},
      color=black,
      rotate around={\pgfkeysvalueof{/tikz/truss/rotation}:(\pgfkeysvalueof{/tikz/truss/x},\pgfkeysvalueof{/tikz/truss/y})}] 
      (\pgfkeysvalueof{/tikz/truss/x},\pgfkeysvalueof{/tikz/truss/y}) -- 
    +(\pgfkeysvalueof{/tikz/truss/length},0cm);

\draw[line cap=round,line width=\pgfkeysvalueof{/tikz/truss/width}-\pgfkeysvalueof{/tikz/truss/border}/0.5,      color=\pgfkeysvalueof{/tikz/truss/fill},
      rotate around={\pgfkeysvalueof{/tikz/truss/rotation}:(\pgfkeysvalueof{/tikz/truss/x},\pgfkeysvalueof{/tikz/truss/y})}] (\pgfkeysvalueof{/tikz/truss/x},\pgfkeysvalueof{/tikz/truss/y}) -- +(\pgfkeysvalueof{/tikz/truss/length},0cm);


      \draw [rotate around={\pgfkeysvalueof{/tikz/truss/rotation}:(\pgfkeysvalueof{/tikz/truss/x},\pgfkeysvalueof{/tikz/truss/y})}] (\pgfkeysvalueof{/tikz/truss/x},\pgfkeysvalueof{/tikz/truss/y}) circle (1pt);


      \draw[rotate around={\pgfkeysvalueof{/tikz/truss/rotation}:(\pgfkeysvalueof{/tikz/truss/x},\pgfkeysvalueof{/tikz/truss/y})}] ( \pgfkeysvalueof{/tikz/truss/x}+\pgfkeysvalueof{/tikz/truss/length},\pgfkeysvalueof{/tikz/truss/y}) circle (1pt);
      }
}
\newcommand{\tikztruss}[1][]{\tikzset{/tikz/truss/.cd,#1}}

%%%%%%%%%%%%%%%%%%%%%%%%%%%%%%%%%%%%%%%%%%%%%
% tikzframe
%%%%%%%%%%%%%%%%%%%%%%%%%%%%%%%%%%%%%%%%%%%%%
\pgfkeys{/tikz/frame/.cd,
x/.initial=0cm,
y/.initial=0cm,
width/.initial=0.3cm,
thickness/.initial=0.5pt,
axis thickness/.initial=0.5pt,
fill/.initial=black!15,
border/.initial=1pt,
line type/.initial=densely dash dot,
rotation/.initial=0,
axis/.code={%
\draw[line width=\pgfkeysvalueof{/tikz/frame/axis thickness},
      dash pattern=on 2pt off 2pt on 8pt off 2pt,
      color=black,
      rotate around={\pgfkeysvalueof{/tikz/frame/rotation}:(\pgfkeysvalueof{/tikz/frame/x},\pgfkeysvalueof{/tikz/frame/y})}] (\pgfkeysvalueof{/tikz/frame/x},\pgfkeysvalueof{/tikz/frame/y}) \foreach \x/\y in {#1} {-- (\x,\y)};
      },
%%%%%%%%%%%%%%%%%%%%%%%%%%%%%%%%%%%%%%%%%%%%%
% This code assumes that the frame is drawn from left to write and the last part of it is a horizontal beam.
cap end x/.code={
% let us first count how many elements are in the array:
\foreach \x/\y [count=\xi] in {#1} {}
% Only the last item must be handled differently, and thus that is our target:
\edef\mytarget{\xi}
\draw[line width=\pgfkeysvalueof{/tikz/frame/width},
      color=black,
      rotate around={\pgfkeysvalueof{/tikz/frame/rotation}:(\pgfkeysvalueof{/tikz/frame/x},\pgfkeysvalueof{/tikz/frame/y})}] (\pgfkeysvalueof{/tikz/frame/x},\pgfkeysvalueof{/tikz/frame/y})
\foreach \x/\y [count=\xi] in {#1}
{\ifnum\xi=\mytarget
-- (\x+\pgfkeysvalueof{/tikz/frame/border},\y)
\else
-- (\x,\y)
\fi};
\draw[line width=\pgfkeysvalueof{/tikz/frame/width}-\pgfkeysvalueof{/tikz/frame/border}/0.5,
      color=\pgfkeysvalueof{/tikz/frame/fill},
      rotate around={\pgfkeysvalueof{/tikz/frame/rotation}:(\pgfkeysvalueof{/tikz/frame/x},\pgfkeysvalueof{/tikz/frame/y})}] (\pgfkeysvalueof{/tikz/frame/x},\pgfkeysvalueof{/tikz/frame/y}) \foreach \x/\y in {#1} {-- (\x,\y)};
},
%%%%%%%%%%%%%%%%%%%%%%%%%%%%%%%%%%%%%%%%%%%%%
draw/.code={%
\draw[line width=\pgfkeysvalueof{/tikz/frame/width},
      color=black,
      rotate around={\pgfkeysvalueof{/tikz/frame/rotation}:(\pgfkeysvalueof{/tikz/frame/x},\pgfkeysvalueof{/tikz/frame/y})}] (\pgfkeysvalueof{/tikz/frame/x},\pgfkeysvalueof{/tikz/frame/y}) \foreach \x/\y in {#1} {-- (\x,\y)};
\draw[line width=\pgfkeysvalueof{/tikz/frame/width}-\pgfkeysvalueof{/tikz/frame/border}/0.5,
      color=\pgfkeysvalueof{/tikz/frame/fill},
      rotate around={\pgfkeysvalueof{/tikz/frame/rotation}:(\pgfkeysvalueof{/tikz/frame/x},\pgfkeysvalueof{/tikz/frame/y})}] (\pgfkeysvalueof{/tikz/frame/x},\pgfkeysvalueof{/tikz/frame/y}) \foreach \x/\y in {#1} {-- (\x,\y)};
}
}
\newcommand{\tikzframe}[1][]{\tikzset{/tikz/frame/.cd,#1}}
%%%%%%%%%%%%%%%%%%%%%%%%%%%%%%%%%%%%%%%%%%%%%
% tikzNODE
%%%%%%%%%%%%%%%%%%%%%%%%%%%%%%%%%%%%%%%%%%%%%
\pgfkeys{/tikz/node/.cd,
x/.initial          = 0cm,
y/.initial          = 0cm,
radius/.initial     = 0.15cm,
fill/.initial       = white,
color/.initial      = black,
line width/.initial = 0.5pt,
text size/.initial  = \tiny,
text/.initial       = 1,
draw/.code={
\filldraw [fill = \pgfkeysvalueof{/tikz/node/fill}, line width=\pgfkeysvalueof{/tikz/node/line width}] (\pgfkeysvalueof{/tikz/node/x},\pgfkeysvalueof{/tikz/node/y}) circle [radius=\pgfkeysvalueof{/tikz/node/radius}] node {\pgfkeysvalueof{/tikz/node/text size}\pgfkeysvalueof{/tikz/node/text}};
}
}
%%%%%%%%%%%%%%%%%%%%%%%%%%%%%%%%%%%%%%%%%%%%%
\newcommand{\tikznode}[1][]{\tikzset{/tikz/node/.cd,#1}}
%%%%%%%%%%%%%%%%%%%%%%%%%%%%%%%%%%%%%%%%%%%%%

%%%%%%%%%%%%%%%%%%%%%%%%%%%%%%%%%%%%%%%%%%%%%
% tikzELEMENT
%%%%%%%%%%%%%%%%%%%%%%%%%%%%%%%%%%%%%%%%%%%%%
\pgfkeys{/tikz/element/.cd,
x1/.initial         = 0cm,
y1/.initial         = 0cm,
x2/.initial         = 0cm,
y2/.initial         = 0cm,
color/.initial      = black,
line width/.initial = 2pt,
text size/.initial  = \tiny,
text/.initial       = \Rnum{1},
node option/.style  = {midway, below},
draw/.code={
\draw[line width = \pgfkeysvalueof{/tikz/element/line width}] (\pgfkeysvalueof{/tikz/element/x1},\pgfkeysvalueof{/tikz/element/y1}) -- (\pgfkeysvalueof{/tikz/element/x2},\pgfkeysvalueof{/tikz/element/y2}) node [/tikz/element/node option] {\pgfkeysvalueof{/tikz/element/text}};
}
}
%%%%%%%%%%%%%%%%%%%%%%%%%%%%%%%%%%%%%%%%%%%%%
\newcommand{\tikzelement}[1][]{\tikzset{/tikz/element/.cd,#1}}
%%%%%%%%%%%%%%%%%%%%%%%%%%%%%%%%%%%%%%%%%%%%%


\newcommand{\lcomposite}[2]{%
\begin{scope}
\def\wid{1.25}
\begin{scope}[fill=gray!30, semithick,shading=axis,shading angle=0,line cap= round,line join=round]
\filldraw[] (#1,#2) -- ++(50:2) -- ++(\wid,0) -- ++(-130:2) -- cycle;
\end{scope}%
\begin{scope}[fill=gray!10, semithick,shading=axis,shading angle=0,line cap= round,line join=round]
\filldraw[] (#1,#2-0.2) -- ++(0,.2) -- ++(\wid,0) -- +(0,-.2) -- cycle;
\end{scope}
\begin{scope}[fill=gray!10, semithick,shading=axis,shading angle=0,line cap= round,line join=round]
\filldraw[] (#1+\wid,#2-0.2) -- ++(0,.2) --  ++(50:2) -- +(0,-.2) -- cycle;
\end{scope}
\foreach \x in {0.1,0.2,...,1.1} 
{\draw[very thin] (#1+.1+\x,#2+.1) -- ++(50:1.75);}
\end{scope}
}

\newcommand{\pcomposite}[2]{%
\begin{scope}[fill=gray!30, semithick,shading=axis,shading angle=0,line cap= round,line join=round]
\filldraw[] (#1,#2) -- ++(50:2) -- ++(1.25,0) -- ++(-130:2) -- cycle;
\end{scope}%
\begin{scope}[fill=gray!10, semithick,shading=axis,shading angle=0,line cap= round,line join=round]
\filldraw[] (#1,#2-0.2) -- ++(0,.2) -- ++(1.25,0) -- +(0,-.2) -- cycle;
\end{scope}
\begin{scope}[fill=gray!10, semithick,shading=axis,shading angle=0,line cap= round,line join=round]
\filldraw[] (#1+1.25,#2-0.2) -- ++(0,.2) --  ++(50:2) -- +(0,-.2) -- cycle;
\end{scope}
\foreach \x in {0.1,0.2,...,1.9} 
{\draw[very thin] (#1+.1+\x*0.65,#2+0.76*\x) -- ++(0:1.05);}
}


% ══════════════════════════════════════════════════════════════════════════════════════════════════
% Composite
% ══════════════════════════════════════════════════════════════════════════════════════════════════
\pgfkeys{/tikz/composite/.cd,
x1/.initial              = 0cm,
y1/.initial              = 0cm,
x2/.initial              = 3.5cm,
y2/.initial              = 1.8cm,
angle/.initial           = 50,
thickness/.initial       = 0.2cm,
top fill option/.style   = {fill=gray!30, line cap= round, line join=round},
front fill option/.style = {fill=gray!10, line cap= round, line join=round},
side fill option/.style  = {fill=gray!10, line cap= round, line join=round},
line width/.initial      = 1pt,
fiber step/.initial      = 0.15cm,
fiber gap/.initial       = 0.15cm,
fiber width/.initial     = 0.8pt,
% ──────────────────────────────────────────────────────────────────────────────────────────────────
draw longitudinal/.code={%
\begingroup
% Calculations
\pgfmathsetlengthmacro\edgelen{(\pgfkeysvalueof{/tikz/composite/y2}-\pgfkeysvalueof{/tikz/composite/y1})/sin(\pgfkeysvalueof{/tikz/composite/angle})}
\pgfmathsetlengthmacro\otheredgelen{(\pgfkeysvalueof{/tikz/composite/x2}-\pgfkeysvalueof{/tikz/composite/x1})-1*(\pgfkeysvalueof{/tikz/composite/y2}-\pgfkeysvalueof{/tikz/composite/y1})/tan(\pgfkeysvalueof{/tikz/composite/angle})}
% Composite
\filldraw[line width=\pgfkeysvalueof{/tikz/distributed/uniform/line width}, /tikz/composite/top fill option] (\pgfkeysvalueof{/tikz/composite/x1},\pgfkeysvalueof{/tikz/composite/y1}) -- ++(\pgfkeysvalueof{/tikz/composite/angle}:\edgelen) -- (\pgfkeysvalueof{/tikz/composite/x2},\pgfkeysvalueof{/tikz/composite/y2}) -- ++(-180+\pgfkeysvalueof{/tikz/composite/angle}:\edgelen) -- cycle;
\filldraw[line width=\pgfkeysvalueof{/tikz/distributed/uniform/line width}, /tikz/composite/front fill option] (\pgfkeysvalueof{/tikz/composite/x1},\pgfkeysvalueof{/tikz/composite/y1}) rectangle  ++(\otheredgelen,-\pgfkeysvalueof{/tikz/composite/thickness}); 
\filldraw[line width=\pgfkeysvalueof{/tikz/distributed/uniform/line width}, /tikz/composite/side fill option] (\pgfkeysvalueof{/tikz/composite/x1}+\otheredgelen,\pgfkeysvalueof{/tikz/composite/y1}) -- ++(\pgfkeysvalueof{/tikz/composite/angle}:\edgelen) -- ++(0,-\pgfkeysvalueof{/tikz/composite/thickness}) -- ++(-180+\pgfkeysvalueof{/tikz/composite/angle}:\edgelen)  -- cycle; 
% Fibres
\coordinate (o) at ($(\pgfkeysvalueof{/tikz/composite/x1},\pgfkeysvalueof{/tikz/composite/y1})+(\pgfkeysvalueof{/tikz/composite/angle}:\pgfkeysvalueof{/tikz/composite/fiber gap})$);
\pgfmathsetlengthmacro\fiberlen{\edgelen-2*\pgfkeysvalueof{/tikz/composite/fiber gap}}
\path [
postaction= {decorate,
             decoration={markings,
                         mark= between positions 0 and 1 step \pgfkeysvalueof{/tikz/composite/fiber step} with {\draw[cap=round,line width=\pgfkeysvalueof{/tikz/composite/fiber width}] ++(0,0) -- (\pgfkeysvalueof{/tikz/composite/angle}:\fiberlen);}
                        }
            }] ($(o)+(\pgfkeysvalueof{/tikz/composite/fiber gap},0cm)$) -- +(\otheredgelen-2*\pgfkeysvalueof{/tikz/composite/fiber gap},0cm);
\endgroup
},
% ──────────────────────────────────────────────────────────────────────────────────────────────────
draw transverse/.code={%
\begingroup
% Calculations
\pgfmathsetlengthmacro\edgelen{(\pgfkeysvalueof{/tikz/composite/y2}-\pgfkeysvalueof{/tikz/composite/y1})/sin(\pgfkeysvalueof{/tikz/composite/angle})}
\pgfmathsetlengthmacro\otheredgelen{(\pgfkeysvalueof{/tikz/composite/x2}-\pgfkeysvalueof{/tikz/composite/x1})-1*(\pgfkeysvalueof{/tikz/composite/y2}-\pgfkeysvalueof{/tikz/composite/y1})/tan(\pgfkeysvalueof{/tikz/composite/angle})}
% Composite
\filldraw[line width=\pgfkeysvalueof{/tikz/distributed/uniform/line width}, /tikz/composite/top fill option] (\pgfkeysvalueof{/tikz/composite/x1},\pgfkeysvalueof{/tikz/composite/y1}) -- ++(\pgfkeysvalueof{/tikz/composite/angle}:\edgelen) -- (\pgfkeysvalueof{/tikz/composite/x2},\pgfkeysvalueof{/tikz/composite/y2}) -- ++(-180+\pgfkeysvalueof{/tikz/composite/angle}:\edgelen) -- cycle;
\filldraw[line width=\pgfkeysvalueof{/tikz/distributed/uniform/line width}, /tikz/composite/front fill option] (\pgfkeysvalueof{/tikz/composite/x1},\pgfkeysvalueof{/tikz/composite/y1}) rectangle  ++(\otheredgelen,-\pgfkeysvalueof{/tikz/composite/thickness}); 
\filldraw[line width=\pgfkeysvalueof{/tikz/distributed/uniform/line width}, /tikz/composite/side fill option] (\pgfkeysvalueof{/tikz/composite/x1}+\otheredgelen,\pgfkeysvalueof{/tikz/composite/y1}) -- ++(\pgfkeysvalueof{/tikz/composite/angle}:\edgelen) -- ++(0,-\pgfkeysvalueof{/tikz/composite/thickness}) -- ++(-180+\pgfkeysvalueof{/tikz/composite/angle}:\edgelen)  -- cycle; 
% Fibres
\coordinate (o) at ($(\pgfkeysvalueof{/tikz/composite/x1},\pgfkeysvalueof{/tikz/composite/y1})+(\pgfkeysvalueof{/tikz/composite/angle}:\pgfkeysvalueof{/tikz/composite/fiber gap})$);
\pgfmathsetlengthmacro\otherfiberlen{\otheredgelen-2*\pgfkeysvalueof{/tikz/composite/fiber gap}}
\path [
postaction= {decorate,
             decoration={markings,
                         mark= between positions 0 and 1 step \pgfkeysvalueof{/tikz/composite/fiber step} with {\draw[cap=round,line width=\pgfkeysvalueof{/tikz/composite/fiber width}] (0,0) -- (-\pgfkeysvalueof{/tikz/composite/angle}:\otherfiberlen);}
                        }
            }] ($(o)+(\pgfkeysvalueof{/tikz/composite/fiber gap},0cm)$) -- +(\pgfkeysvalueof{/tikz/composite/angle}:\edgelen-2*\pgfkeysvalueof{/tikz/composite/fiber gap});
\endgroup
},
% ──────────────────────────────────────────────────────────────────────────────────────────────────
draw/.code={% This is for 45 degree fibers
\begingroup
% Calculations
\pgfmathsetlengthmacro\edgelen{(\pgfkeysvalueof{/tikz/composite/y2}-\pgfkeysvalueof{/tikz/composite/y1})/sin(\pgfkeysvalueof{/tikz/composite/angle})}
\pgfmathsetlengthmacro\otheredgelen{(\pgfkeysvalueof{/tikz/composite/x2}-\pgfkeysvalueof{/tikz/composite/x1})-1*(\pgfkeysvalueof{/tikz/composite/y2}-\pgfkeysvalueof{/tikz/composite/y1})/tan(\pgfkeysvalueof{/tikz/composite/angle})}
% Composite
\filldraw[line width=\pgfkeysvalueof{/tikz/distributed/uniform/line width}, /tikz/composite/top fill option] (\pgfkeysvalueof{/tikz/composite/x1},\pgfkeysvalueof{/tikz/composite/y1}) -- ++(\pgfkeysvalueof{/tikz/composite/angle}:\edgelen) -- (\pgfkeysvalueof{/tikz/composite/x2},\pgfkeysvalueof{/tikz/composite/y2}) -- ++(-180+\pgfkeysvalueof{/tikz/composite/angle}:\edgelen) -- cycle;
\filldraw[line width=\pgfkeysvalueof{/tikz/distributed/uniform/line width}, /tikz/composite/front fill option] (\pgfkeysvalueof{/tikz/composite/x1},\pgfkeysvalueof{/tikz/composite/y1}) rectangle  ++(\otheredgelen,-\pgfkeysvalueof{/tikz/composite/thickness}); 
\filldraw[line width=\pgfkeysvalueof{/tikz/distributed/uniform/line width}, /tikz/composite/side fill option] (\pgfkeysvalueof{/tikz/composite/x1}+\otheredgelen,\pgfkeysvalueof{/tikz/composite/y1}) -- ++(\pgfkeysvalueof{/tikz/composite/angle}:\edgelen) -- ++(0,-\pgfkeysvalueof{/tikz/composite/thickness}) -- ++(-180+\pgfkeysvalueof{/tikz/composite/angle}:\edgelen)  -- cycle; 
% Fibres
\coordinate (o) at ($(\pgfkeysvalueof{/tikz/composite/x1},\pgfkeysvalueof{/tikz/composite/y1})+(\pgfkeysvalueof{/tikz/composite/angle}:\pgfkeysvalueof{/tikz/composite/fiber gap})$);
\pgfmathsetlengthmacro\otherfiberlen{\otheredgelen-2*\pgfkeysvalueof{/tikz/composite/fiber gap}}
\path [
postaction= {decorate,
             decoration={markings,
                         mark= between positions 0 and 1 step \pgfkeysvalueof{/tikz/composite/fiber step} with {\draw[cap=round,line width=\pgfkeysvalueof{/tikz/composite/fiber width}] (0,0) -- (-\pgfkeysvalueof{/tikz/composite/angle}:\otherfiberlen);}
                        }
            }] ($(o)+(\pgfkeysvalueof{/tikz/composite/fiber gap},0cm)$) -- +(\pgfkeysvalueof{/tikz/composite/angle}:\edgelen-2*\pgfkeysvalueof{/tikz/composite/fiber gap});
\endgroup
}
}
% ══════════════════════════════════════════════════════════════════════════════════════════════════
\newcommand{\tikzcomposite}[1][]{\tikzset{/tikz/composite/.cd,#1}}
% ══════════════════════════════════════════════════════════════════════════════════════════════════

