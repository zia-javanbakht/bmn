\chapter{Force and Stress Measures}

\section{Introduction to Force and Stress}

The concept of stress represents one of the most fundamental notions in continuum mechanics, providing the mathematical framework for describing internal forces within deformable bodies~\autocite{Sadd.2019}. Unlike discrete particle systems where forces act at specific points, continuous media require distributed force descriptions that account for the infinite number of material points and their interactions.

\begin{keypoint}
Stress provides the crucial link between external loading and internal material response, enabling the analysis of failure, design optimization, and material characterization.
\end{keypoint}

In continuum mechanics, forces are classified into two primary categories: body forces and surface forces. Body forces, such as gravitational, electromagnetic, or inertial forces, act throughout the material volume and are expressed per unit mass or per unit volume. Surface forces represent contact interactions between different parts of the continuum or with external boundaries, acting across real or imaginary surfaces within the material~\autocite{Sadd.2019}.

The mathematical description of internal forces leads naturally to the stress tensor concept, which characterizes the force per unit area transmitted across internal surfaces. This tensor field completely determines the local force state at every point in the continuum, making it essential for formulating equilibrium conditions and constitutive relations.

\section{Cauchy Stress Principle and Stress Tensor}

The fundamental principle governing stress in continuum mechanics is Cauchy's stress principle, which establishes the relationship between surface tractions and the underlying stress state. Consider an arbitrary internal surface within a continuum body, characterized by its unit normal vector $\dir{n}$. The traction vector $\tena{t}^{(\dir{n})}$ represents the force per unit area acting across this surface, transmitted from material on the positive side of the normal to material on the negative side~\autocite{Sadd.2019}.

Cauchy's stress principle states that this traction vector depends linearly on the surface orientation:
\begin{equation}
\tena{t}^{(\dir{n})} = \tenb{\sigma} \scp \dir{n}
\end{equation}

where $\tenb{\sigma}$ is the Cauchy stress tensor. This fundamental relationship, known as Cauchy's formula, demonstrates that the stress tensor completely characterizes the internal force state at any material point.

\begin{keypoint}
Cauchy's formula $\tena{t}^{(\dir{n})} = \tenb{\sigma} \scp \dir{n}$ is the cornerstone of stress analysis, showing that the stress tensor contains complete information about forces on all possible surface orientations.
\end{keypoint}

The stress tensor $\tenb{\sigma}$ is a second-order tensor with nine components $\sigma_{ij}$ in Cartesian coordinates. The physical interpretation of these components follows from Cauchy's formula: $\sigma_{ij}$ represents the $i$-th component of the traction vector acting on a surface with unit normal in the positive $j$-direction. The first index indicates the force direction, while the second index specifies the surface orientation~\autocite{Sadd.2019}.

The stress tensor components have the following physical interpretation:
\begin{itemize}
\item Diagonal terms ($\sigma_{11}$, $\sigma_{22}$, $\sigma_{33}$): Normal stresses acting perpendicular to coordinate surfaces
\item Off-diagonal terms ($\sigma_{ij}$, $i \neq j$): Shear stresses acting parallel to coordinate surfaces
\end{itemize}

\section{Stress Tensor Properties and Symmetry}

A crucial property of the Cauchy stress tensor is its symmetry, which follows from the requirement of moment equilibrium for infinitesimal material elements. Consider a differential volume element in equilibrium under applied tractions and body forces. The vanishing of angular momentum about any axis requires that the moments of all forces sum to zero~\autocite{Sadd.2019}.

Application of moment equilibrium conditions yields:
\begin{equation}
\sigma_{ij} = \sigma_{ji}
\end{equation}

This symmetry property reduces the number of independent stress components from nine to six, significantly simplifying stress analysis and constitutive modeling.

The stress tensor transformation under coordinate rotations follows the standard tensor transformation law:
\begin{equation}
\sigma'_{ij} = Q_{ik}Q_{jl}\sigma_{kl}
\end{equation}

where $\tenb{Q}$ is the rotation tensor and primed quantities represent components in the rotated coordinate system. This objectivity property ensures that stress descriptions remain consistent across different coordinate systems.

\section{Principal Stresses and Stress Invariants}

At any point in a continuum, there exist three mutually perpendicular directions along which the stress state is characterized by pure normal stresses with no shear components. These special directions are called principal stress directions, and the corresponding normal stresses are principal stresses~\autocite{Sadd.2019}.

Mathematically, principal stresses $\sigma_I$, $\sigma_{II}$, $\sigma_{III}$ (often denoted $\sigma_1$, $\sigma_2$, $\sigma_3$) are eigenvalues of the stress tensor, obtained by solving the characteristic equation:
\begin{equation}
\det(\tenb{\sigma} - \sigma \tenb{I}) = 0
\end{equation}

Expanding this determinant yields the characteristic polynomial:
\begin{equation}
\sigma^3 - I_\sigma \sigma^2 + II_\sigma \sigma - III_\sigma = 0
\end{equation}

The coefficients of this polynomial are the stress invariants:
\begin{align}
I_\sigma &= \sigma_{11} + \sigma_{22} + \sigma_{33} = \text{tr}(\tenb{\sigma})\\
II_\sigma &= \sigma_{11}\sigma_{22} + \sigma_{22}\sigma_{33} + \sigma_{33}\sigma_{11} - \sigma_{12}^2 - \sigma_{23}^2 - \sigma_{13}^2\\
III_\sigma &= \det(\tenb{\sigma})
\end{align}

These invariants provide objective measures of the stress state, remaining unchanged under coordinate transformations. They form the foundation for developing yield criteria and constitutive relations in materials with isotropic properties~\autocite{Sadd.2019}.

\begin{keypoint}
Principal stress invariants provide coordinate-independent measures of stress state, forming the foundation for objective failure criteria and constitutive relations.
\end{keypoint}

The principal stress directions $\dir{n}^{(i)}$ are the corresponding eigenvectors, satisfying:
\begin{equation}
(\tenb{\sigma} - \sigma_i \tenb{I}) \scp \dir{n}^{(i)} = \tena{0}
\end{equation}

At any material point, the principal stress directions form an orthonormal basis that diagonalizes the stress tensor.

\section{Stress Deviator and Spherical Decomposition}

The stress tensor can be decomposed into volumetric (spherical) and deviatoric (distortional) components, providing insight into different mechanisms of deformation and failure~\autocite{Sadd.2019}:
\begin{equation}
\tenb{\sigma} = \sigma_m \tenb{I} + \tenb{s}
\end{equation}

where:
\begin{align}
\sigma_m &= \frac{1}{3}\text{tr}(\tenb{\sigma}) = \frac{1}{3}(\sigma_{11} + \sigma_{22} + \sigma_{33}) \quad \text{(mean normal stress)}\\
\tenb{s} &= \tenb{\sigma} - \sigma_m \tenb{I} \quad \text{(stress deviator)}
\end{align}

The mean normal stress $\sigma_m$ represents the hydrostatic component responsible for volumetric changes, while the stress deviator $\tenb{s}$ characterizes distortional deformation and shape changes. This decomposition proves fundamental for understanding yielding behavior, as many materials yield primarily due to distortional effects rather than hydrostatic pressure.

\begin{keypoint}
Stress deviator decomposition separates volumetric and distortional effects, which is essential for understanding material yielding and failure mechanisms.
\end{keypoint}

The stress deviator has zero trace by construction:
\begin{equation}
\text{tr}(\tenb{s}) = 0
\end{equation}

Important invariants of the stress deviator include:
\begin{align}
J_1 &= \text{tr}(\tenb{s}) = 0\\
J_2 &= \frac{1}{2}s_{ij}s_{ij} = \frac{1}{2}\text{tr}(\tenb{s}^2)\\
J_3 &= \det(\tenb{s}) = \frac{1}{3}s_{ij}s_{jk}s_{ki}
\end{align}

These deviatoric invariants are particularly useful for formulating yield criteria and plastic flow rules in metal plasticity and other constitutive theories~\autocite{Sadd.2019}.

\section{Special Stress States and Applications}

Several idealized stress states appear frequently in engineering applications and theoretical developments:

\subsection{Uniaxial Stress State}
This state occurs when only one principal stress is non-zero ($\sigma_1 \neq 0$, $\sigma_2 = \sigma_3 = 0$). Common examples include tension and compression tests on cylindrical specimens. The stress tensor in principal coordinates becomes:
\begin{equation}
[\tenb{\sigma}] = \begin{bmatrix} \sigma_1 & 0 & 0 \\ 0 & 0 & 0 \\ 0 & 0 & 0 \end{bmatrix}
\end{equation}

\subsection{Biaxial Stress State}
This state features two non-zero principal stresses ($\sigma_1, \sigma_2 \neq 0$, $\sigma_3 = 0$), typical in thin sheets under in-plane loading or pressure vessels with thin walls. Applications include forming operations and structural membranes~\autocite{Sadd.2019}.

\subsection{Hydrostatic Stress State}
All principal stresses are equal ($\sigma_1 = \sigma_2 = \sigma_3 = \sigma_m$), resulting in pure volumetric loading without shape change. This state occurs in deep ocean conditions or high-pressure material processing.

\subsection{Pure Shear State}
The mean normal stress vanishes ($\sigma_m = 0$), resulting in pure deviatoric stress. This state produces maximum distortional deformation and is relevant for torsion problems and certain failure modes.

\section{Equivalent Stress Measures}

Several scalar measures have been developed to characterize the intensity of complex stress states, facilitating comparison with simple test results and implementation in failure criteria.

\subsection{Von Mises Equivalent Stress}
The von Mises equivalent stress provides a scalar measure of deviatoric stress intensity:
\begin{equation}
\sigma_{\text{vm}} = \sqrt{3J_2} = \sqrt{\frac{3}{2}s_{ij}s_{ij}}
\end{equation}

In terms of principal stresses:
\begin{equation}
\sigma_{\text{vm}} = \sqrt{\frac{1}{2}[(\sigma_1-\sigma_2)^2 + (\sigma_2-\sigma_3)^2 + (\sigma_3-\sigma_1)^2]}
\end{equation}

This measure forms the basis of the von Mises yield criterion, widely used for ductile materials. It represents the magnitude of deviatoric stress and correlates well with yielding behavior in metals~\autocite{Sadd.2019}.

\subsection{Octahedral Shear Stress}
The octahedral shear stress acts on planes equally inclined to all three principal stress directions:
\begin{equation}
\tau_{\text{oct}} = \frac{1}{3}\sqrt{(\sigma_1-\sigma_2)^2 + (\sigma_2-\sigma_3)^2 + (\sigma_3-\sigma_1)^2} = \frac{\sigma_{\text{vm}}}{\sqrt{3}}
\end{equation}

The octahedral normal stress equals the mean normal stress:
\begin{equation}
\sigma_{\text{oct}} = \frac{1}{3}(\sigma_1 + \sigma_2 + \sigma_3) = \sigma_m
\end{equation}

These octahedral stresses provide alternative characterizations of the stress state that are particularly useful for understanding three-dimensional failure behavior.

\subsection{Maximum Shear Stress}
The maximum shear stress occurs on planes oriented 45° to the principal stress directions:
\begin{equation}
\tau_{\text{max}} = \frac{1}{2}(\sigma_{\text{max}} - \sigma_{\text{min}})
\end{equation}

This measure forms the basis of the Tresca yield criterion and is relevant for understanding brittle fracture and certain modes of plastic failure~\autocite{Sadd.2019}.

\section{Alternative Stress Measures for Finite Deformation}

For problems involving large deformations, the Cauchy stress tensor may not provide the most convenient description. Alternative stress measures that refer to different configurations are often more suitable for constitutive modeling and computational implementation.

\subsection{First Piola-Kirchhoff Stress Tensor}
The first Piola-Kirchhoff stress tensor $\tenb{P}$ relates current forces to areas in the reference configuration:
\begin{equation}
\tenb{P} = J\tenb{\sigma}\tenb{F}^{-T}
\end{equation}

where $J = \det(\tenb{F})$ is the Jacobian of deformation and $\tenb{F}$ is the deformation gradient. This two-point tensor has the advantage of relating to the undeformed configuration but is generally not symmetric~\autocite{Sadd.2019}.

The nominal traction $\tena{T}_0$ on a surface with reference unit normal $\dir{N}$ is given by:
\begin{equation}
\tena{T}_0 = \tenb{P} \scp \dir{N}
\end{equation}

\subsection{Second Piola-Kirchhoff Stress Tensor}
The second Piola-Kirchhoff stress tensor $\tenb{S}$ provides a purely material description:
\begin{equation}
\tenb{S} = J\tenb{F}^{-1}\tenb{\sigma}\tenb{F}^{-T} = \tenb{F}^{-1}\tenb{P}
\end{equation}

This tensor is symmetric and work-conjugate to the Green-Lagrange strain tensor, making it particularly suitable for hyperelastic constitutive modeling. The work density per unit reference volume is:
\begin{equation}
\mathcal{W} = \tenb{S} \dscp \dot{\tenb{E}}
\end{equation}

where $\tenb{E}$ is the Green-Lagrange strain tensor~\autocite{Sadd.2019}.

\subsection{Kirchhoff Stress Tensor}
The Kirchhoff stress tensor $\tenb{\tau}$ represents a weighted Cauchy stress:
\begin{equation}
\tenb{\tau} = J\tenb{\sigma}
\end{equation}

This measure is particularly useful for isochoric (volume-preserving) deformations and appears naturally in certain constitutive frameworks.

\section{Stress Transformation and Mohr's Circle}

The transformation of stress components under coordinate rotation provides essential tools for determining critical stress states and failure conditions. For plane stress conditions, Mohr's circle offers a graphical method for visualizing stress transformations and determining principal stresses and maximum shear stresses~\autocite{Sadd.2019}.

Consider a two-dimensional stress state with components $\sigma_{xx}$, $\sigma_{yy}$, and $\sigma_{xy}$ in the $x$-$y$ coordinate system. The stress components on a plane inclined at angle $\theta$ to the $x$-axis are:
\begin{align}
\sigma_{\theta} &= \frac{\sigma_{xx} + \sigma_{yy}}{2} + \frac{\sigma_{xx} - \sigma_{yy}}{2}\cos(2\theta) + \sigma_{xy}\sin(2\theta)\\
\tau_{\theta} &= -\frac{\sigma_{xx} - \sigma_{yy}}{2}\sin(2\theta) + \sigma_{xy}\cos(2\theta)
\end{align}

Mohr's circle provides a geometric representation of these transformations, with the center at $(\sigma_m, 0)$ and radius $R$ given by:
\begin{align}
\sigma_m &= \frac{\sigma_{xx} + \sigma_{yy}}{2}\\
R &= \sqrt{\left(\frac{\sigma_{xx} - \sigma_{yy}}{2}\right)^2 + \sigma_{xy}^2}
\end{align}

The principal stresses are located at the intersections of the circle with the normal stress axis:
\begin{equation}
\sigma_{1,2} = \sigma_m \pm R
\end{equation}

\section{Stress Boundary Conditions and Equilibrium}

Boundary conditions specify stress or traction values on the body's surface, providing essential constraints for solving boundary value problems in continuum mechanics. For a surface $\Gamma$ with outward unit normal $\dir{n}$, the stress boundary condition is:
\begin{equation}
\tenb{\sigma} \scp \dir{n} = \tena{t}_0 \quad \text{on } \Gamma_t
\end{equation}

where $\tena{t}_0$ is the prescribed surface traction and $\Gamma_t$ denotes portions of the boundary where tractions are specified~\autocite{Sadd.2019}.

Mixed boundary conditions can prescribe different components of traction or displacement on the same surface, requiring careful mathematical treatment to ensure well-posed problems. The compatibility of boundary conditions with equilibrium requirements is essential for solution existence and uniqueness.

For static equilibrium problems, the stress field must satisfy the equilibrium equations throughout the domain:
\begin{equation}
\nabla \scp \tenb{\sigma} + \tena{b} = \tena{0}
\end{equation}

where $\tena{b}$ represents body forces per unit volume. These equations, combined with appropriate boundary conditions and constitutive relations, form the foundation for stress analysis in engineering structures.

\section{Applications and Engineering Significance}

The stress tensor concept and its associated measures provide the fundamental framework for understanding and predicting material behavior under various loading conditions. Applications span diverse fields including:

\begin{itemize}
\item \textbf{Structural Analysis}: Design of buildings, bridges, and mechanical components requires accurate stress prediction to prevent failure and optimize performance.
\item \textbf{Material Characterization}: Standard test procedures rely on stress measures to quantify material properties and establish failure criteria.
\item \textbf{Manufacturing Processes}: Forming operations, machining, and other manufacturing processes involve complex stress states that determine product quality and process feasibility.
\item \textbf{Geomechanics}: Understanding stress distributions in soil and rock masses is crucial for foundation design, tunneling, and resource extraction.
\item \textbf{Biomechanics}: Stress analysis in biological tissues helps understand injury mechanisms and design medical devices.
\end{itemize}

The concepts presented in this chapter form the essential foundation for constitutive modeling, failure analysis, and computational mechanics. Modern engineering design increasingly relies on sophisticated stress analysis techniques that build upon these fundamental principles, enabling the development of safer, more efficient, and more innovative engineering solutions~\autocite{Sadd.2019}.

\section{Summary}

\begin{subox}[Summary]
This chapter developed the fundamental concepts of stress and force measures in continuum mechanics:

\textbf{Force and Stress Concepts:}
\begin{itemize}
\item Stress provides mathematical framework for describing internal forces in continuous media
\item Body forces act throughout volume; surface forces act across boundaries and internal surfaces
\item Cauchy stress principle: $\tena{t}^{(\dir{n})} = \tenb{\sigma} \scp \dir{n}$ completely characterizes force state
\item Stress tensor $\tenb{\sigma}$ has 9 components: diagonal (normal stresses), off-diagonal (shear stresses)
\end{itemize}

\textbf{Stress Tensor Properties:}
\begin{itemize}
\item Symmetry: $\tenb{\sigma} = \tenb{\sigma}^T$ from angular momentum balance
\item Reduces independent components from 9 to 6
\item Objective transformation under coordinate rotations: $\sigma'_{ij} = Q_{ik}Q_{jl}\sigma_{kl}$
\item Second-order tensor field completely determining local force state
\end{itemize}

\textbf{Principal Stresses and Invariants:}
\begin{itemize}
\item Principal stresses $\sigma_1, \sigma_2, \sigma_3$ are eigenvalues: $\det(\tenb{\sigma} - \sigma \tenb{I}) = 0$
\item Principal invariants: $I_\sigma = \text{tr}(\tenb{\sigma})$, $II_\sigma$, $III_\sigma = \det(\tenb{\sigma})$
\item Coordinate-independent measures forming basis for constitutive relations and failure criteria
\item Principal directions form orthonormal basis diagonalizing stress tensor
\end{itemize}

\textbf{Stress Decomposition:}
\begin{itemize}
\item Spherical-deviatoric: $\tenb{\sigma} = \sigma_m \tenb{I} + \tenb{s}$ separating volumetric and distortional effects
\item Mean stress $\sigma_m = \frac{1}{3}\text{tr}(\tenb{\sigma})$ controls volume change
\item Stress deviator $\tenb{s}$ with zero trace characterizes shape change and yielding
\item Deviatoric invariants $J_2$, $J_3$ used in plasticity and failure theories
\end{itemize}

\textbf{Equivalent Stress Measures:}
\begin{itemize}
\item Von Mises stress: $\sigma_{\text{vm}} = \sqrt{3J_2}$ for ductile material yielding
\item Octahedral stresses on planes equally inclined to principal directions
\item Maximum shear stress: $\tau_{\text{max}} = \frac{1}{2}(\sigma_{\text{max}} - \sigma_{\text{min}})$ for Tresca criterion
\item Scalar measures enabling comparison with simple test results
\end{itemize}

\textbf{Alternative Stress Measures:}
\begin{itemize}
\item First Piola-Kirchhoff: $\tenb{P} = J\tenb{\sigma}\tenb{F}^{-T}$ for finite deformation (non-symmetric)
\item Second Piola-Kirchhoff: $\tenb{S} = J\tenb{F}^{-1}\tenb{\sigma}\tenb{F}^{-T}$ (symmetric, work-conjugate to Green-Lagrange strain)
\item Kirchhoff stress: $\tenb{\tau} = J\tenb{\sigma}$ for isochoric deformations
\item Enable constitutive modeling for large deformations
\end{itemize}

\textbf{Stress Analysis Tools:}
\begin{itemize}
\item Stress transformation under coordinate rotation for determining critical states
\item Mohr's circle for graphical representation of plane stress states
\item Principal stress determination and maximum shear stress identification
\item Stress boundary conditions: $\tenb{\sigma} \scp \dir{n} = \tena{t}_0$ on surface $\Gamma_t$
\end{itemize}

\textbf{Engineering Applications:}
\begin{itemize}
\item Structural analysis and design optimization requiring stress prediction
\item Material characterization through standard test procedures
\item Manufacturing process analysis involving complex stress states
\item Failure analysis using appropriate stress measures and criteria
\item Computational mechanics implementation in finite element methods
\end{itemize}

\textbf{Physical Significance:}
\begin{itemize}
\item Stress tensor provides complete local force description independent of material properties
\item Essential for formulating equilibrium equations and failure criteria
\item Foundation for constitutive modeling relating stress to deformation
\item Enables engineering design ensuring structural safety and performance
\end{itemize}

The stress concepts developed here form the cornerstone for understanding material behavior under load, providing the foundation for constitutive relations, failure analysis, and engineering design.
\end{subox}