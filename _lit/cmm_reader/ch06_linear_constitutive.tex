\chapter{Classical Linear Constitutive Theories}

\section{Introduction to Constitutive Relations}

Constitutive equations provide the essential link between stress and deformation measures that characterizes material behavior, completing the mathematical framework of continuum mechanics by specifying material-specific relationships~\autocite{Sadd.2019}. These relationships must be developed to satisfy fundamental principles including objectivity, material symmetry, thermodynamic consistency, and dimensional homogeneity.

The formulation of constitutive equations represents one of the most challenging aspects of continuum mechanics, as it requires capturing the complex physical and chemical processes that occur at the microstructural level through macroscopic mathematical relationships. The constitutive response determines how materials respond to external loading and environmental conditions, making it essential for engineering design and analysis.

\begin{keypoint}
Constitutive equations bridge the gap between universal balance laws and material-specific behavior, enabling prediction of material response under various loading conditions.
\end{keypoint}

Linear constitutive theories represent the simplest class of material models, valid for small deformations and moderate stress levels. These theories assume proportional relationships between generalized forces (stress) and generalized displacements (strain), enabling analytical solutions for many practical engineering problems. The linearity assumption greatly simplifies mathematical analysis while maintaining reasonable accuracy for a wide range of engineering applications~\autocite{Sadd.2019}.

The development of constitutive relations requires careful consideration of material symmetry, which reflects the underlying microstructural organization. Material symmetry groups dictate the form of constitutive equations and determine the number of independent material constants. Additionally, thermodynamic restrictions ensure that constitutive models satisfy energy conservation and entropy production requirements.

\section{Fundamental Principles of Constitutive Modeling}

\subsection{Principle of Objectivity}

Constitutive equations must be objective (frame-indifferent), meaning they remain invariant under rigid body motions~\autocite{Sadd.2019}. This principle ensures that material response depends only on deformation, not on the motion of the observer or the coordinate system choice.

For finite deformation elasticity, objectivity requires that stress depends on deformation measures that are unaffected by rigid body rotations. The right Cauchy-Green tensor $\tenb{C} = \tenb{F}^T \tenb{F}$ satisfies this requirement, leading to:
\begin{equation}
\tenb{S} = \tenb{S}(\tenb{C})
\end{equation}

where $\tenb{S}$ is the second Piola-Kirchhoff stress tensor.

\subsection{Material Symmetry}

Material symmetry reflects the equivalence of certain material directions due to microstructural organization. A material has a symmetry if its constitutive response remains unchanged under specific transformations of the reference configuration~\autocite{Sadd.2019}.

The symmetry group determines the form of constitutive equations:
\begin{itemize}
\item \textbf{Triclinic}: No symmetry (21 independent constants for elasticity)
\item \textbf{Monoclinic}: One plane of symmetry (13 constants)
\item \textbf{Orthotropic}: Three orthogonal planes of symmetry (9 constants)
\item \textbf{Transversely isotropic}: Rotational symmetry about one axis (5 constants)
\item \textbf{Isotropic}: All directions equivalent (2 constants)
\end{itemize}

\subsection{Thermodynamic Restrictions}

The Clausius-Duhem inequality constrains admissible constitutive relations by requiring non-negative entropy production. For elastic materials, this leads to the requirement that the stress derives from a stored energy function:
\begin{equation}
\tenb{S} = \rho_0 \frac{\partial \psi}{\partial \tenb{E}}
\end{equation}

where $\psi$ is the Helmholtz free energy per unit mass and $\tenb{E}$ is the Green-Lagrange strain tensor.

\section{Linear Elasticity Theory}

Linear elastic materials exhibit instantaneous and reversible deformation under applied loads with no permanent set upon unloading. The fundamental assumption is that stress depends linearly on strain with no rate effects or history dependence~\autocite{Sadd.2019}.

\begin{keypoint}
Linear elasticity assumes small deformations, proportional stress-strain relationships, and instantaneous, reversible response without hysteresis or time dependence.
\end{keypoint}

\subsection{Generalized Hooke's Law}

For a general anisotropic linear elastic material, the stress-strain relationship is:
\begin{equation}
\sigma_{ij} = C_{ijkl} \varepsilon_{kl}
\end{equation}

The fourth-order elasticity tensor $\tenb{C}$ possesses several inherent symmetries:
\begin{align}
C_{ijkl} &= C_{jikl} \quad \text{(from stress symmetry)}\\
C_{ijkl} &= C_{ijlk} \quad \text{(from strain symmetry)}\\
C_{ijkl} &= C_{klij} \quad \text{(from thermodynamic restrictions)}
\end{align}

These symmetries reduce the independent elastic constants from 81 to 21 for the most general anisotropic case~\autocite{Sadd.2019}.

\subsection{Isotropic Linear Elasticity}

For isotropic materials, rotational invariance requires that the constitutive response be independent of coordinate orientation. This symmetry reduces the elastic constants to only two independent parameters. The general form becomes:
\begin{equation}
\sigma_{ij} = \lambda \varepsilon_{kk} \delta_{ij} + 2\mu \varepsilon_{ij}
\end{equation}

where $\lambda$ and $\mu$ are the Lamé constants. The first term represents the volumetric response, while the second term characterizes distortional behavior.

\begin{keypoint}
Isotropic elasticity requires only two independent material constants, reflecting the directional equivalence of material properties.
\end{keypoint}

The inverse relationship, expressing strain in terms of stress, is:
\begin{equation}
\varepsilon_{ij} = \frac{1+\nu}{E}\sigma_{ij} - \frac{\nu}{E}\sigma_{kk}\delta_{ij}
\end{equation}

where $E$ is Young's modulus and $\nu$ is Poisson's ratio. These engineering constants relate to the Lamé constants through:
\begin{align}
E &= \frac{\mu(3\lambda + 2\mu)}{\lambda + \mu}\\
\nu &= \frac{\lambda}{2(\lambda + \mu)}
\end{align}

Other useful elastic constants include:
\begin{align}
G &= \mu \quad \text{(shear modulus)}\\
K &= \lambda + \frac{2\mu}{3} \quad \text{(bulk modulus)}
\end{align}

\subsection{Strain Energy and Elastic Potentials}

For elastic materials, the stress derives from a scalar strain energy density function $W(\tenb{E})$:
\begin{equation}
\tenb{S} = \frac{\partial W}{\partial \tenb{E}}
\end{equation}

For linear isotropic elasticity, the strain energy density is:
\begin{equation}
W = \frac{1}{2}\lambda(\text{tr}(\tenb{E}))^2 + \mu \text{tr}(\tenb{E}^2)
\end{equation}

This quadratic form ensures linear stress-strain relationships and provides the foundation for energy methods in elasticity~\autocite{Sadd.2019}.

\section{Orthotropic and Anisotropic Elasticity}

\subsection{Orthotropic Materials}

Orthotropic materials possess three mutually perpendicular planes of material symmetry, typical of many composite materials and wood. The constitutive matrix in principal material coordinates has nine independent constants:

\begin{equation}
\begin{bmatrix}
\sigma_{11} \\
\sigma_{22} \\
\sigma_{33} \\
\sigma_{23} \\
\sigma_{13} \\
\sigma_{12}
\end{bmatrix}
=
\begin{bmatrix}
C_{11} & C_{12} & C_{13} & 0 & 0 & 0 \\
C_{12} & C_{22} & C_{23} & 0 & 0 & 0 \\
C_{13} & C_{23} & C_{33} & 0 & 0 & 0 \\
0 & 0 & 0 & C_{44} & 0 & 0 \\
0 & 0 & 0 & 0 & C_{55} & 0 \\
0 & 0 & 0 & 0 & 0 & C_{66}
\end{bmatrix}
\begin{bmatrix}
\varepsilon_{11} \\
\varepsilon_{22} \\
\varepsilon_{33} \\
2\varepsilon_{23} \\
2\varepsilon_{13} \\
2\varepsilon_{12}
\end{bmatrix}
\end{equation}

In engineering notation using compliance matrix:
\begin{equation}
\begin{bmatrix}
\varepsilon_{11} \\
\varepsilon_{22} \\
\varepsilon_{33} \\
\gamma_{23} \\
\gamma_{13} \\
\gamma_{12}
\end{bmatrix}
=
\begin{bmatrix}
1/E_1 & -\nu_{21}/E_2 & -\nu_{31}/E_3 & 0 & 0 & 0 \\
-\nu_{12}/E_1 & 1/E_2 & -\nu_{32}/E_3 & 0 & 0 & 0 \\
-\nu_{13}/E_1 & -\nu_{23}/E_2 & 1/E_3 & 0 & 0 & 0 \\
0 & 0 & 0 & 1/G_{23} & 0 & 0 \\
0 & 0 & 0 & 0 & 1/G_{13} & 0 \\
0 & 0 & 0 & 0 & 0 & 1/G_{12}
\end{bmatrix}
\begin{bmatrix}
\sigma_{11} \\
\sigma_{22} \\
\sigma_{33} \\
\sigma_{23} \\
\sigma_{13} \\
\sigma_{12}
\end{bmatrix}
\end{equation}

The compliance matrix must satisfy symmetry conditions: $\nu_{ij}/E_i = \nu_{ji}/E_j$.

\subsection{Transversely Isotropic Materials}

Materials with one axis of rotational symmetry (such as unidirectional fiber composites) require five independent constants. The axis of symmetry is typically denoted as the 3-direction~\autocite{Sadd.2019}.

\section{Elastic Boundary Value Problems}

Linear elasticity problems require simultaneous satisfaction of three sets of equations:

\subsection{Equilibrium Equations}
\begin{equation}
\nabla \scp \tenb{\sigma} + \rho \tena{b} = \tena{0}
\end{equation}

\subsection{Kinematic Relations}
\begin{equation}
\varepsilon_{ij} = \frac{1}{2}\left(\frac{\partial u_i}{\partial x_j} + \frac{\partial u_j}{\partial x_i}\right)
\end{equation}

\subsection{Constitutive Equations}
For isotropic materials:
\begin{equation}
\sigma_{ij} = \lambda \varepsilon_{kk} \delta_{ij} + 2\mu \varepsilon_{ij}
\end{equation}

Combining these yields the displacement-based field equations (Navier equations):
\begin{equation}
\mu \nabla^2 \tena{u} + (\lambda + \mu) \nabla(\nabla \scp \tena{u}) + \rho \tena{b} = \tena{0}
\end{equation}

\subsection{Boundary Conditions}

Proper boundary conditions must be specified on the domain boundary $\partial\Omega$:

\textbf{Displacement boundary conditions} on $\Gamma_u$:
\begin{equation}
\tena{u} = \tena{u}_0
\end{equation}

\textbf{Traction boundary conditions} on $\Gamma_t$:
\begin{equation}
\tenb{\sigma} \scp \dir{n} = \tena{t}_0
\end{equation}

where $\Gamma_u \cup \Gamma_t = \partial\Omega$ and $\Gamma_u \cap \Gamma_t = \emptyset$~\autocite{Sadd.2019}.

\section{Linear Viscous Flow}

\subsection{Newtonian Fluid Behavior}

Newtonian fluids exhibit stress proportional to strain rate, with the constitutive relationship:
\begin{equation}
\tenb{\tau} = \lambda_v (\nabla \scp \tena{v}) \tenb{I} + 2\mu_v \tenb{D}
\end{equation}

where $\tenb{\tau}$ is the viscous stress tensor, $\tenb{D}$ is the deformation rate tensor, and $\mu_v$ is the dynamic viscosity. The bulk viscosity $\lambda_v$ often satisfies Stokes' hypothesis: $\lambda_v = -\frac{2}{3}\mu_v$.

\begin{keypoint}
Newtonian fluids exhibit linear relationships between viscous stress and strain rate, analogous to Hookean elasticity but with rate dependence.
\end{keypoint}

For incompressible flow ($\nabla \scp \tena{v} = 0$):
\begin{equation}
\tenb{\tau} = 2\mu_v \tenb{D}
\end{equation}

The total stress includes both pressure and viscous contributions:
\begin{equation}
\tenb{\sigma} = -p\tenb{I} + \tenb{\tau}
\end{equation}

\subsection{Navier-Stokes Equations}

Combining momentum balance with Newtonian constitutive relations yields the Navier-Stokes equations for incompressible flow:
\begin{align}
\rho \frac{D\tena{v}}{Dt} &= -\nabla p + \mu_v \nabla^2 \tena{v} + \rho \tena{b}\\
\nabla \scp \tena{v} &= 0
\end{align}

These fundamental equations govern fluid motion in countless engineering applications~\autocite{Sadd.2019}.

\section{Linear Viscoelasticity}

Viscoelastic materials exhibit time-dependent behavior that combines elastic and viscous characteristics. The material response depends on the entire loading history, not just the current state.

\begin{keypoint}
Viscoelastic materials exhibit memory effects, with current stress depending on the complete deformation history through integral relationships.
\end{keypoint}

\subsection{Mechanical Models}

Simple mechanical models using springs and dashpots provide insight into viscoelastic behavior:

\textbf{Maxwell Model} (spring and dashpot in series):
\begin{equation}
\frac{d\sigma}{dt} + \frac{\sigma}{\tau_\sigma} = E \frac{d\varepsilon}{dt}
\end{equation}

where $\tau_\sigma = \eta/E$ is the relaxation time. This model exhibits stress relaxation under constant strain but no equilibrium modulus.

\textbf{Kelvin-Voigt Model} (spring and dashpot in parallel):
\begin{equation}
\sigma = E\varepsilon + \eta \frac{d\varepsilon}{dt}
\end{equation}

This model exhibits creep under constant stress but instantaneous elastic response.

\textbf{Standard Linear Solid} combines Maxwell and Kelvin-Voigt elements:
\begin{equation}
\sigma + \tau_\sigma \frac{d\sigma}{dt} = E_\infty \varepsilon + \tau_\varepsilon E_0 \frac{d\varepsilon}{dt}
\end{equation}

where $E_0$ and $E_\infty$ are initial and long-term moduli~\autocite{Sadd.2019}.

\subsection{Integral Constitutive Relations}

General linear viscoelasticity employs hereditary integrals:
\begin{equation}
\sigma(t) = \int_{-\infty}^t G(t-s) \frac{d\varepsilon(s)}{ds} ds
\end{equation}

where $G(t)$ is the relaxation modulus. For creep response:
\begin{equation}
\varepsilon(t) = \int_{-\infty}^t J(t-s) \frac{d\sigma(s)}{ds} ds
\end{equation}

where $J(t)$ is the creep compliance.

\subsection{Correspondence Principle}

The correspondence principle enables solution of viscoelastic problems by analogy with elasticity. Laplace transforms convert hereditary integrals to algebraic relationships, allowing adaptation of elastic solutions to viscoelastic cases~\autocite{Sadd.2019}.

\section{Classical Plasticity Theory}

Plastic materials undergo permanent deformation when stress exceeds a critical threshold, exhibiting irreversible behavior with energy dissipation.

\subsection{Fundamental Concepts}

Plasticity theory requires specification of:
\begin{itemize}
\item \textbf{Yield criterion}: Defines stress states that initiate plastic flow
\item \textbf{Flow rule}: Determines direction of plastic strain rate
\item \textbf{Hardening rule}: Describes evolution of yield surface
\end{itemize}

\begin{keypoint}
Plasticity involves threshold behavior, irreversible deformation, and dissipative processes that require specialized mathematical frameworks beyond linear elasticity.
\end{keypoint}

\subsection{Yield Criteria}

\textbf{Tresca Criterion} (maximum shear stress):
\begin{equation}
\max_i \abs{\sigma_i - \sigma_j} = \sigma_Y
\end{equation}

This criterion assumes yielding occurs when maximum shear stress reaches a critical value.

\textbf{Von Mises Criterion} (distortional energy):
\begin{equation}
\sqrt{\frac{3}{2}s_{ij}s_{ij}} = \sigma_Y
\end{equation}

where $\tenb{s}$ is the stress deviator. This criterion correlates well with yielding in ductile metals~\autocite{Sadd.2019}.

\subsection{Flow Rules}

The plastic strain rate direction follows the normality principle:
\begin{equation}
\dot{\tenb{\varepsilon}}^p = \dot{\lambda} \frac{\partial f}{\partial \tenb{\sigma}}
\end{equation}

where $f(\tenb{\sigma})$ is the yield function and $\dot{\lambda} \geq 0$ is the plastic multiplier determined by consistency conditions.

For associated plasticity (normality rule), the plastic flow direction is normal to the yield surface. Non-associated plasticity allows different flow directions to account for dilatant behavior in materials like soils and rocks.

\subsection{Hardening Behavior}

\textbf{Isotropic Hardening}: Uniform expansion of yield surface
\begin{equation}
f = f(\tenb{\sigma}, \kappa)
\end{equation}

where $\kappa$ is a hardening parameter.

\textbf{Kinematic Hardening}: Translation of yield surface
\begin{equation}
f = f(\tenb{\sigma} - \tenb{\alpha})
\end{equation}

where $\tenb{\alpha}$ is the backstress tensor representing the center of the yield surface.

\section{Thermoelasticity}

Linear thermoelasticity couples mechanical and thermal effects through thermal expansion and thermomechanical coupling~\autocite{Sadd.2019}.

\subsection{Constitutive Relations}

For isotropic thermoelasticity:
\begin{align}
\sigma_{ij} &= \lambda \varepsilon_{kk} \delta_{ij} + 2\mu \varepsilon_{ij} - \beta (T - T_0) \delta_{ij}\\
\eta &= \eta_0 + c_v \ln\left(\frac{T}{T_0}\right) + \frac{\beta}{\rho_0} \varepsilon_{kk}
\end{align}

where $\beta = (3\lambda + 2\mu)\alpha$ is the thermal stress modulus and $\alpha$ is the thermal expansion coefficient.

\subsection{Coupled Field Equations}

The complete system includes:

\textbf{Momentum balance}:
\begin{equation}
\nabla \scp \tenb{\sigma} + \rho \tena{b} = \rho \frac{D\tena{v}}{Dt}
\end{equation}

\textbf{Energy balance}:
\begin{equation}
\rho c \frac{DT}{Dt} = -\nabla \scp \tena{q} + \rho r + \beta T_0 \frac{D\varepsilon_{kk}}{Dt}
\end{equation}

\textbf{Heat conduction}:
\begin{equation}
\tena{q} = -k \nabla T
\end{equation}

The coupling term $\beta T_0 D\varepsilon_{kk}/Dt$ represents thermoelastic heating/cooling~\autocite{Sadd.2019}.

\section{Applications and Engineering Significance}

Linear constitutive theories provide the foundation for engineering analysis across diverse applications:

\begin{itemize}
\item \textbf{Structural Analysis}: Buildings, bridges, and mechanical components rely on elastic analysis for design and safety assessment.
\item \textbf{Manufacturing}: Metal forming, machining, and joining processes require plasticity theory for process optimization.
\item \textbf{Composite Materials}: Anisotropic elasticity theory enables design of fiber-reinforced composites and layered structures.
\item \textbf{Fluid Mechanics}: Newtonian viscous flow governs many industrial processes including lubrication and pipeline flow.
\item \textbf{Geomechanics}: Soil and rock behavior under moderate loads often follows linear elastic or elastic-plastic models.
\item \textbf{Biomechanics}: Biological tissues exhibit viscoelastic behavior that requires time-dependent constitutive models.
\end{itemize}

Modern computational mechanics extensively employs these linear theories as building blocks for more complex nonlinear models. The mathematical tractability of linear relationships enables analytical solutions that provide fundamental insight into material behavior and serve as benchmarks for numerical methods~\autocite{Sadd.2019}.

\section{Summary}

\begin{subox}[Summary]
This chapter developed the fundamental linear constitutive theories that relate stress and deformation in continuum mechanics:

\textbf{Constitutive Modeling Principles:}
\begin{itemize}
\item Objectivity (frame indifference): constitutive relations invariant under rigid body motions
\item Material symmetry: microstructural organization determines form of constitutive equations
\item Thermodynamic restrictions: Clausius-Duhem inequality constrains admissible material behavior
\item Dimensional consistency and physical realizability requirements
\end{itemize}

\textbf{Linear Elasticity Theory:}
\begin{itemize}
\item Generalized Hooke's law: $\sigma_{ij} = C_{ijkl} \varepsilon_{kl}$ for anisotropic materials
\item Isotropic elasticity: $\sigma_{ij} = \lambda \varepsilon_{kk} \delta_{ij} + 2\mu \varepsilon_{ij}$ (two constants)
\item Strain energy function: $W = \frac{1}{2}\lambda(\text{tr}(\tenb{E}))^2 + \mu \text{tr}(\tenb{E}^2)$
\item Engineering constants: Young's modulus $E$, Poisson's ratio $\nu$, shear modulus $G$, bulk modulus $K$
\end{itemize}

\textbf{Material Symmetries:}
\begin{itemize}
\item Isotropic: all directions equivalent (2 elastic constants)
\item Transversely isotropic: one axis of rotational symmetry (5 constants)
\item Orthotropic: three orthogonal planes of symmetry (9 constants)
\item General anisotropic: no material symmetry (21 constants)
\end{itemize}

\textbf{Elastic Boundary Value Problems:}
\begin{itemize}
\item Equilibrium: $\nabla \scp \tenb{\sigma} + \rho \tena{b} = \tena{0}$
\item Kinematics: $\varepsilon_{ij} = \frac{1}{2}(\partial u_i/\partial x_j + \partial u_j/\partial x_i)$
\item Constitutive relations: stress-strain relationships
\item Navier equations: $\mu \nabla^2 \tena{u} + (\lambda + \mu) \nabla(\nabla \scp \tena{u}) + \rho \tena{b} = \tena{0}$
\end{itemize}

\textbf{Linear Viscous Flow:}
\begin{itemize}
\item Newtonian constitutive relation: $\tenb{\tau} = 2\mu_v \tenb{D}$ for incompressible flow
\item Navier-Stokes equations: $\rho D\tena{v}/Dt = -\nabla p + \mu_v \nabla^2 \tena{v} + \rho \tena{b}$
\item Viscous stress proportional to strain rate (analogous to elastic stress-strain)
\item Applications in fluid mechanics and lubrication theory
\end{itemize}

\textbf{Linear Viscoelasticity:}
\begin{itemize}
\item Memory effects: current stress depends on deformation history
\item Mechanical models: Maxwell, Kelvin-Voigt, standard linear solid combinations
\item Integral relations: $\sigma(t) = \int G(t-s) d\varepsilon(s)/ds ds$ with relaxation modulus $G(t)$
\item Correspondence principle: Laplace transforms enable adaptation of elastic solutions
\end{itemize}

\textbf{Classical Plasticity:}
\begin{itemize}
\item Yield criteria: Tresca (maximum shear), von Mises (distortional energy)
\item Flow rule: $\dot{\tenb{\varepsilon}}^p = \dot{\lambda} \partial f/\partial \tenb{\sigma}$ (normality principle)
\item Hardening: isotropic (yield surface expansion) and kinematic (yield surface translation)
\item Irreversible deformation and energy dissipation beyond elastic limit
\end{itemize}

\textbf{Linear Thermoelasticity:}
\begin{itemize}
\item Coupled mechanical-thermal effects through thermal expansion
\item Constitutive relations: $\sigma_{ij} = \lambda \varepsilon_{kk} \delta_{ij} + 2\mu \varepsilon_{ij} - \beta (T - T_0) \delta_{ij}$
\item Energy balance with thermoelastic coupling: $\rho c DT/Dt = -\nabla \scp \tena{q} + \rho r + \beta T_0 D\varepsilon_{kk}/Dt$
\item Applications in thermal stress analysis and thermomechanical design
\end{itemize}

\textbf{Engineering Applications:}
\begin{itemize}
\item Structural analysis and design using elastic theory
\item Manufacturing process modeling with plasticity theory
\item Composite material analysis using anisotropic elasticity
\item Fluid flow analysis with Newtonian viscous models
\item Time-dependent material behavior with viscoelastic models
\end{itemize}

\textbf{Mathematical Framework:}
\begin{itemize}
\item Linear constitutive theories enable analytical solutions and closed-form expressions
\item Superposition principle applicable for linear elastic problems
\item Foundation for computational mechanics and finite element implementations
\item Building blocks for more complex nonlinear constitutive models
\end{itemize}

\textbf{Physical Significance:}
\begin{itemize}
\item Linear theories valid for small deformations and moderate stress levels
\item Provide fundamental insight into material behavior mechanisms
\item Enable parameter identification from standard material tests
\item Form basis for material characterization and quality control
\item Essential for understanding more complex nonlinear behavior
\end{itemize}

These linear constitutive theories provide the essential foundation for engineering analysis of material behavior under moderate loading conditions, enabling analytical solutions and forming the building blocks for advanced nonlinear theories.
\end{subox}