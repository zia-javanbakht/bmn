\chapter{Kinematics of Motion and Deformation}

\section{Configuration and Motion Description}

The kinematic description forms the foundation of continuum mechanics by providing mathematical tools to describe motion and deformation without reference to the forces causing them~\autocite{Sadd.2019}.

\begin{keypoint}
Kinematics provides a purely geometric description of motion and deformation, independent of forces and material properties, forming the essential foundation for stress analysis and constitutive modeling.
\end{keypoint}

\subsection{Material Body and Configurations}

A material body $\mathcal{B}$ consists of material particles that can be mapped between different configurations:

\textbf{Reference Configuration} $\kappa_0$: A chosen configuration (usually stress-free) where material particles are identified by position vectors $\tena{X}$.

\textbf{Current Configuration} $\kappa_t$: The configuration at time $t$ where particles occupy positions $\tena{x}$.

\subsection{Motion Mapping}

The motion of the continuum is described by the mapping:
\begin{equation}
\tena{x} = \boldsymbol{\chi}(\tena{X}, t)
\end{equation}

This function must satisfy:
\begin{itemize}
\item Continuity: Neighboring particles remain neighbors
\item Invertibility: One-to-one correspondence between configurations
\item Differentiability: Smooth deformation
\end{itemize}

The Jacobian determinant $J = \det(\partial \tena{x}/\partial \tena{X}) > 0$ ensures physical realizability.

\begin{keypoint}
The deformation mapping $\tena{x} = \boldsymbol{\chi}(\tena{X}, t)$ must be continuous, invertible, and differentiable, with $J > 0$ to ensure material doesn't penetrate itself.
\end{keypoint}

\section{Lagrangian and Eulerian Descriptions}

\subsection{Lagrangian (Material) Description}

Follows individual material particles through their motion:
\begin{itemize}
\item Independent variables: $(\tena{X}, t)$
\item Focus: What happens to a specific particle
\item Suitable for: Solid mechanics applications
\end{itemize}

Physical quantities are expressed as functions of material coordinates:
\begin{equation}
\phi = \phi(\tena{X}, t)
\end{equation}

\subsection{Eulerian (Spatial) Description}

Observes what occurs at fixed spatial points:
\begin{itemize}
\item Independent variables: $(\tena{x}, t)$
\item Focus: What happens at a specific location
\item Suitable for: Fluid mechanics applications
\end{itemize}

Physical quantities are expressed as functions of spatial coordinates:
\begin{equation}
\phi = \phi(\tena{x}, t)
\end{equation}

\section{Velocity and Acceleration}

\subsection{Velocity Field}

\textbf{Material Description}:
\begin{equation}
\tena{v}(\tena{X}, t) = \frac{\partial \boldsymbol{\chi}(\tena{X}, t)}{\partial t}
\end{equation}

\textbf{Spatial Description}:
\begin{equation}
\tena{v}(\tena{x}, t) = \tena{v}(\boldsymbol{\chi}(\tena{X}, t), t)
\end{equation}

\subsection{Acceleration Field}

\textbf{Material Description}:
\begin{equation}
\tena{a}(\tena{X}, t) = \frac{\partial \tena{v}(\tena{X}, t)}{\partial t}
\end{equation}

\textbf{Spatial Description}:
\begin{equation}
\tena{a}(\tena{x}, t) = \frac{\partial \tena{v}}{\partial t} + (\tena{v} \scp \nabla)\tena{v}
\end{equation}

The second term represents convective acceleration due to spatial variation of velocity.

\section{Material Time Derivative}

The material time derivative provides the rate of change of any quantity following a material particle:

\begin{equation}
\frac{D\phi}{Dt} = \frac{\partial \phi}{\partial t} + \tena{v} \scp \nabla \phi
\end{equation}

\begin{keypoint}
The material time derivative tracks changes following material particles, distinguishing between local time changes and convective changes due to particle motion.
\end{keypoint}

For vector and tensor fields:
\begin{align}
\frac{D\tena{a}}{Dt} &= \frac{\partial \tena{a}}{\partial t} + (\tena{v} \scp \nabla)\tena{a} \\
\frac{D\tenb{A}}{Dt} &= \frac{\partial \tenb{A}}{\partial t} + (\tena{v} \scp \nabla)\tenb{A}
\end{align}

\section{Deformation Gradient Tensor}

\subsection{Definition and Properties}

The deformation gradient tensor characterizes local deformation:
\begin{equation}
\tenb{F} = \frac{\partial \tena{x}}{\partial \tena{X}} = \frac{\partial \chi_i}{\partial X_j} \base_i \otimes \Base_j
\end{equation}

Properties:
\begin{itemize}
\item Two-point tensor: components in different coordinate systems
\item Generally non-symmetric
\item $J = \det(\tenb{F}) > 0$ for physical deformation
\end{itemize}

\begin{keypoint}
The deformation gradient $\tenb{F}$ is the fundamental kinematic quantity that completely characterizes local deformation, containing all information about stretching, rotation, and volume change.
\end{keypoint}

\subsection{Geometric Interpretation}

$\tenb{F}$ maps line elements from reference to current configuration:
\begin{equation}
d\tena{x} = \tenb{F} \scp d\tena{X}
\end{equation}

\subsection{Volume and Area Changes}

\textbf{Volume Change}:
\begin{equation}
dv = J \, dV
\end{equation}

\textbf{Area Change} (Nanson's formula):
\begin{equation}
\dir{n} \, da = J \tenb{F}^{-T} \scp \Normal \, dA
\end{equation}

where $\dir{n}$ and $\Normal$ are unit normals in current and reference configurations.

\section{Strain Measures}

\subsection{Right and Left Cauchy-Green Tensors}

\textbf{Right Cauchy-Green Tensor}:
\begin{equation}
\tenb{C} = \tenb{F}^T \tenb{F}
\end{equation}

\textbf{Left Cauchy-Green Tensor}:
\begin{equation}
\tenb{b} = \tenb{F} \tenb{F}^T
\end{equation}

Both are symmetric, positive definite tensors.

\subsection{Green-Lagrange Strain Tensor}

The Green-Lagrange strain tensor provides a measure of finite deformation:
\begin{equation}
\tenb{E} = \frac{1}{2}(\tenb{C} - \tenb{I}) = \frac{1}{2}(\tenb{F}^T \tenb{F} - \tenb{I})
\end{equation}

In terms of displacement $\tena{u} = \tena{x} - \tena{X}$:
\begin{equation}
\tenb{E} = \frac{1}{2}(\nabla_{\tena{X}} \tena{u} + (\nabla_{\tena{X}} \tena{u})^T + (\nabla_{\tena{X}} \tena{u})^T \nabla_{\tena{X}} \tena{u})
\end{equation}

\subsection{Almansi-Eulerian Strain Tensor}

The Almansi strain tensor is defined in the current configuration:
\begin{equation}
\tenb{e} = \frac{1}{2}(\tenb{I} - \tenb{b}^{-1}) = \frac{1}{2}(\tenb{I} - (\tenb{F}\tenb{F}^T)^{-1})
\end{equation}

\subsection{Physical Interpretation of Strain}

\textbf{Length Change}:
For a line element $d\tena{X}$ with unit direction $\Normal$:
\begin{equation}
\lambda^2 = \Normal \scp \tenb{C} \scp \Normal = 1 + 2\Normal \scp \tenb{E} \scp \Normal
\end{equation}

where $\lambda = \lVert d\tena{x}\rVert/\lVert d\tena{X}\rVert$ is the stretch ratio.

\textbf{Angle Change}:
The change in angle between two initially perpendicular directions involves off-diagonal strain components.

\section{Polar Decomposition}

\subsection{Decomposition Theorem}

Any invertible tensor can be uniquely decomposed as:
\begin{align}
\tenb{F} &= \tenb{R} \tenb{U} \quad \text{(right polar decomposition)} \\
\tenb{F} &= \tenb{V} \tenb{R} \quad \text{(left polar decomposition)}
\end{align}

where:
\begin{itemize}
\item $\tenb{R}$: orthogonal tensor ($\tenb{R}\tenb{R}^T = \tenb{I}$, $\det(\tenb{R}) = +1$)
\item $\tenb{U}, \tenb{V}$: symmetric, positive definite stretch tensors
\end{itemize}

\begin{keypoint}
Polar decomposition separates pure stretching from rigid rotation, providing clear physical insight into the nature of deformation.
\end{keypoint}

\subsection{Relationship to Cauchy-Green Tensors}

\begin{align}
\tenb{U}^2 &= \tenb{C} = \tenb{F}^T \tenb{F} \\
\tenb{V}^2 &= \tenb{b} = \tenb{F} \tenb{F}^T
\end{align}

\subsection{Principal Stretches}

The principal values of $\tenb{U}$ and $\tenb{V}$ are the principal stretches $\lambda_1, \lambda_2, \lambda_3$:
\begin{align}
\tenb{U} &= \lambda_1 \Normal_1 \otimes \Normal_1 + \lambda_2 \Normal_2 \otimes \Normal_2 + \lambda_3 \Normal_3 \otimes \Normal_3 \\
\tenb{V} &= \lambda_1 \dir{n}_1 \otimes \dir{n}_1 + \lambda_2 \dir{n}_2 \otimes \dir{n}_2 + \lambda_3 \dir{n}_3 \otimes \dir{n}_3
\end{align}

where $\dir{n}_i = \tenb{R} \scp \Normal_i$.

\section{Velocity Gradient and Rate Tensors}

\subsection{Velocity Gradient Tensor}

The spatial velocity gradient characterizes the rate of deformation:
\begin{equation}
\tenb{L} = \nabla \tena{v} = \frac{\partial v_i}{\partial x_j} \base_i \otimes \base_j
\end{equation}

\subsection{Decomposition of Velocity Gradient}

\begin{equation}
\tenb{L} = \tenb{D} + \tenb{W}
\end{equation}

where:
\begin{align}
\tenb{D} &= \frac{1}{2}(\tenb{L} + \tenb{L}^T) \quad \text{(rate of deformation tensor)} \\
\tenb{W} &= \frac{1}{2}(\tenb{L} - \tenb{L}^T) \quad \text{(vorticity tensor)}
\end{align}

\subsection{Physical Interpretation}

\textbf{Rate of Deformation Tensor} $\tenb{D}$:
\begin{itemize}
\item Symmetric tensor describing local shape changes
\item Diagonal components: normal strain rates
\item Off-diagonal components: shear strain rates
\end{itemize}

\textbf{Vorticity Tensor} $\tenb{W}$:
\begin{itemize}
\item Skew-symmetric tensor describing local rotation
\item Associated with vorticity vector: $\boldsymbol{\omega} = \frac{1}{2}\nabla \times \tena{v}$
\end{itemize}

\section{Infinitesimal Strain Theory}

\subsection{Small Displacement Assumption}

When displacement gradients are small ($\|\nabla \tena{u}\| \ll 1$), the nonlinear terms in finite strain expressions can be neglected.

\begin{keypoint}
Infinitesimal strain theory provides significant mathematical simplification for small deformations but becomes invalid for finite rotations and large strains.
\end{keypoint}

\subsection{Infinitesimal Strain Tensor}

The linearized strain tensor becomes:
\begin{equation}
\boldsymbol{\varepsilon} = \frac{1}{2}(\nabla \tena{u} + (\nabla \tena{u})^T)
\end{equation}

Components:
\begin{align}
\varepsilon_{ij} &= \frac{1}{2}\left(\frac{\partial u_i}{\partial x_j} + \frac{\partial u_j}{\partial x_i}\right) \quad (i \neq j) \\
\varepsilon_{ii} &= \frac{\partial u_i}{\partial x_i} \quad \text{(no sum on } i \text{)}
\end{align}

\subsection{Engineering Strain Components}

Conventional engineering strains:
\begin{align}
\varepsilon_{11} &= \frac{\partial u_1}{\partial x_1} \quad \text{(normal strain)} \\
\gamma_{12} &= 2\varepsilon_{12} = \frac{\partial u_1}{\partial x_2} + \frac{\partial u_2}{\partial x_1} \quad \text{(shear strain)}
\end{align}

\section{Compatibility Conditions}

\subsection{Saint-Venant Compatibility Equations}

For a strain field to correspond to a continuous displacement field, it must satisfy compatibility conditions:
\begin{equation}
\frac{\partial^2 \varepsilon_{ij}}{\partial x_k \partial x_l} + \frac{\partial^2 \varepsilon_{kl}}{\partial x_i \partial x_j} = \frac{\partial^2 \varepsilon_{ik}}{\partial x_j \partial x_l} + \frac{\partial^2 \varepsilon_{jl}}{\partial x_i \partial x_k}
\end{equation}

In index notation:
\begin{equation}
\varepsilon_{ij,kl} + \varepsilon_{kl,ij} = \varepsilon_{ik,jl} + \varepsilon_{jl,ik}
\end{equation}

\subsection{Simplified Forms}

For two-dimensional problems:
\begin{equation}
\frac{\partial^2 \varepsilon_{11}}{\partial x_2^2} + \frac{\partial^2 \varepsilon_{22}}{\partial x_1^2} = \frac{\partial^2 \gamma_{12}}{\partial x_1 \partial x_2}
\end{equation}

\section{Special Deformation States}

\subsection{Rigid Body Motion}

Pure translation: $\tena{x} = \tena{X} + \tena{c}(t)$
Pure rotation: $\tena{x} = \tenb{Q}(t) \scp \tena{X}$

For rigid motion: $\tenb{F} = \tenb{Q}$, $\tenb{C} = \tenb{I}$, $\tenb{E} = \tenb{0}$

\subsection{Homogeneous Deformation}

Deformation gradient is constant: $\tenb{F} = \text{constant}$

Examples:
\begin{itemize}
\item Simple extension
\item Simple shear
\item Pure shear
\end{itemize}

\subsection{Plane Strain and Plane Stress}

\textbf{Plane Strain}: $u_3 = 0$, $\varepsilon_{i3} = 0$

\textbf{Plane Stress}: $\sigma_{i3} = 0$ (typically for thin plates)

These kinematic measures provide the foundation for describing material motion and deformation, forming the basis for developing stress analysis and constitutive relations in continuum mechanics.

\section{Summary}

\begin{subox}[Summary]
This chapter developed the kinematic framework for describing motion and deformation in continuum mechanics:

\textbf{Configuration and Motion:}
\begin{itemize}
\item Material body mapped between reference configuration $\kappa_0$ and current configuration $\kappa_t$
\item Motion described by mapping $\tena{x} = \boldsymbol{\chi}(\tena{X}, t)$ that must be continuous, invertible, and differentiable
\item Jacobian determinant $J = \det(\partial \tena{x}/\partial \tena{X}) > 0$ ensures physical realizability
\item Lagrangian description follows material particles; Eulerian description observes fixed spatial points
\end{itemize}

\textbf{Velocity and Acceleration:}
\begin{itemize}
\item Velocity field: $\tena{v} = \partial \boldsymbol{\chi}/\partial t$ (material) or $\tena{v}(\tena{x}, t)$ (spatial)
\item Acceleration: $\tena{a} = \partial \tena{v}/\partial t + (\tena{v} \scp \nabla)\tena{v}$ includes convective acceleration
\item Material time derivative: $D\phi/Dt = \partial\phi/\partial t + \tena{v} \scp \nabla \phi$ tracks particle changes
\end{itemize}

\textbf{Deformation Gradient and Strain:}
\begin{itemize}
\item Deformation gradient $\tenb{F} = \partial \tena{x}/\partial \tena{X}$ characterizes local deformation
\item Maps line elements: $d\tena{x} = \tenb{F} \scp d\tena{X}$
\item Volume change: $dv = J \, dV$; Area change: $\dir{n} \, da = J \tenb{F}^{-T} \scp \Normal \, dA$
\item Right Cauchy-Green tensor: $\tenb{C} = \tenb{F}^T \tenb{F}$; Left: $\tenb{b} = \tenb{F} \tenb{F}^T$
\item Green-Lagrange strain: $\tenb{E} = \frac{1}{2}(\tenb{C} - \tenb{I})$; Almansi strain: $\tenb{e} = \frac{1}{2}(\tenb{I} - \tenb{b}^{-1})$
\end{itemize}

\textbf{Polar Decomposition:}
\begin{itemize}
\item Unique decomposition: $\tenb{F} = \tenb{R} \tenb{U} = \tenb{V} \tenb{R}$ separating rotation from stretching
\item $\tenb{R}$: orthogonal rotation tensor; $\tenb{U}, \tenb{V}$: symmetric stretch tensors
\item Principal stretches $\lambda_1, \lambda_2, \lambda_3$ are eigenvalues of $\tenb{U}$ and $\tenb{V}$
\item Provides clear physical interpretation of deformation mechanisms
\end{itemize}

\textbf{Velocity Gradient and Rate Tensors:}
\begin{itemize}
\item Velocity gradient: $\tenb{L} = \nabla \tena{v}$ characterizes rate of deformation
\item Decomposition: $\tenb{L} = \tenb{D} + \tenb{W}$ into symmetric and skew-symmetric parts
\item Rate of deformation tensor $\tenb{D}$: describes shape changes and strain rates
\item Vorticity tensor $\tenb{W}$: describes local rotation and relates to vorticity vector
\end{itemize}

\textbf{Infinitesimal Strain Theory:}
\begin{itemize}
\item Valid for small displacement gradients: $\|\nabla \tena{u}\| \ll 1$
\item Linearized strain tensor: $\boldsymbol{\varepsilon} = \frac{1}{2}(\nabla \tena{u} + (\nabla \tena{u})^T)$
\item Engineering strain components: normal strains $\varepsilon_{ii}$ and shear strains $\gamma_{ij} = 2\varepsilon_{ij}$
\item Compatibility conditions ensure strain field corresponds to continuous displacement
\end{itemize}

\textbf{Special Cases and Applications:}
\begin{itemize}
\item Rigid body motion: $\tenb{F} = \tenb{Q}$, $\tenb{E} = \tenb{0}$ (no deformation)
\item Homogeneous deformation: constant $\tenb{F}$ throughout material
\item Plane strain and plane stress approximations for specific geometries
\item Simple extension, simple shear, and pure shear as fundamental deformation modes
\end{itemize}

\textbf{Physical Significance:}
\begin{itemize}
\item Kinematics provides purely geometric description independent of forces and material properties
\item Essential foundation for stress analysis, constitutive modeling, and balance law formulation
\item Finite deformation theory handles large strains and rotations in modern applications
\item Rate tensors connect instantaneous kinematics to material time evolution
\end{itemize}

The kinematic framework developed here forms the cornerstone for all subsequent developments in continuum mechanics, providing the mathematical tools to describe how materials move and deform.
\end{subox}